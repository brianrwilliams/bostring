\section{Global sections of the factorization algebra} 
\brian{this should probably go earlier}

%First, consider the elliptic complex
%\ben
%\Omega^{0,*}(\Sigma ; \cV)
%\een
%of $(0,*)$-forms with values in $\cV$. The differential is simply the $\dbar$ operator. We consider its shift $\Omega^{0,*}(\Sigma ; \cV)[-1]$ as an abelian local dg Lie algebra. In this way, the associated cotangent theory of the elliptic complex is defined as 
%\ben
%T^{*}[-1] (\Omega^{0,*}(\Sigma ; \cV)) = \Omega^{0,*}(\Sigma ; V) \oplus \Omega^{1,*}(\Sigma ; T_\Sigma^{1,0}) \Omega^{1,*}(\Sigma ; \cV^\vee) \oplus \ .
%\een 
%We write the fields as $\gamma \in \Omega^{0,*}(\Sigma ; \cV)$ for the base direction and $\beta \in \Omega^{1,*}(\Sigma ; \cV^\vee)$ for the fiber direction. With this notation, the action functional is defined as
%\ben
%S (\beta, \gamma) = \int_\Sigma \<\beta, \dbar \gamma\>_\cV
%\een
%where $\<-,-\>_\cV$ denotes the evaluation pairing between $\cV$ and its dual. This theory is called the $\beta\gamma$ (or sometimes $\beta \gamma-bc$ \owen{what are the $bc$ fields?}) system with values in $\cV$. This is a free field theory and hence has a natural quantization. We will denote the factorization algebra of quantum observables on $\Sigma$ by $\Obs^{q}_{\cV}$. 

\brian{Discuss relationship to conformal blocks}

We wish to write the global observables of the holomorphic string in terms of the cohomology of natural holomorphic vector bundles on the Riemann surface. 

In Section \ref{rmk:bcbg} we have introduced the $bc\beta\gamma$ system defined on an Riemann surface $\Sigma$. 
This theory is free, and hence admits a canonical BV quantization.
Denote by $\Obs^{\q}_{free}$ be the corresponding factorization algebra.
In fact, Proposition 8.1.4.1 in \cite{fact1}, there is an isomorphism of the cohomology of the global observables of this factorization algebra:
\ben
H^*\left(\Obs^\q_{free}(\Sigma)\right) \cong \det \left(H^*(\Sigma ; \sO_\Sigma) \right)^{\tensor \dim(V)} \tensor \det \left(H^*(\Sigma ; T_\Sigma^{1,0})\right)^{-1} [d(\Sigma)] 
\een
where 
\ben
d(\Sigma) = \dim (V) \cdot \left(\dim H^0(\Sigma ; \sO_\Sigma) + \dim H^1(\Sigma ; \sO_\Sigma)\right) + \dim(H^0(\Sigma ; T_\Sigma^{1,0})) + \dim(H^0(\Sigma ; T_\Sigma^{1,0}))
\een

%\begin{lem}[\cite{fact1} Proposition 8.1.4.1]
%\label{lem bg global obs} 
%Let $b_i = \dim H^i (\Sigma ; \cV)$. The cohomology of the global observables of the $\beta\gamma$ system with values in $\cV$ is given by
%\ben
%H^*\left(\Obs^q_{\cV}(\Sigma)\right) \cong \det \left(H^*(\Sigma ; \cV) \right) [d(\cV)] 
%\een
%where $d(\cV) = -b_0 - b_1$. 
%\end{lem}

\owen{It would be good here to point out the GRR argument for identifying the first Chern class of this determinant line bundle. We can then point out that it resonates with our computation on the formal disk and give a reminder that this is a manifestation of ``Virasoro uniformization.'' This then leads into the string case: the holomorphic vector fields (rather the free $bc$ system) also contribute a determinant line (we should include that computation) and we can ask when those determinant lines tensor to a trivial line. Note that deforming the action from free $bc\beta\gamma$ system to holomorphic bosonic string doesn't affect the line bundles, since those are continuous parameters and Chern classes are discrete.}

\subsection{The holomorphic string on a Riemann surface}

We have constructed the quantization of the holomorphic string on $\CC$ and derived from its factorization algebra a vertex algebra. 
The only anomaly cancellation condition to have such a quantization was the restriction that the target be $13$ complex dimensional. 
Here, we show that our anomaly calculation above is actually enough to show the existence of a quantization on an arbitrary Riemann surface. 

Let 



Now, consider the global observables of the bosonic string $\Obs^\q(\Sigma)$. There is a spectral sequence converging to the cohomology of the global observables $H^*\Obs^\q(\Sigma)$ with $E_2$ page given by the cohomology of the global observables of the $\beta \gamma$ system with values in the holomorphic vector bundle $\cV = \cT_\Sigma [1] \oplus \CC^{13}$. By Lemma \ref{lem bg global obs} this $E_2$ page is concentrated in a single degree and is given by
\bestar
E_2 & \cong & \det\left(H^*(\Sigma ; \cT_\Sigma[1])\right) \tensor \det \left(H^*(\Sigma ; \cO_\Sigma)^{\oplus 13}\right) \\
& \cong & \det \left(H^1(\Sigma ; \cT_\Sigma) \right) \tensor \det \left(H^0(\Sigma ; \cT_\Sigma)\right)^{-1} \tensor \det \left(H^0(\Sigma ; K_{\Sigma}) \right)^{-13}
\eestar
where we have used the fact that $H^0(\Sigma ; \cO) \cong \CC$ for any $\Sigma$. 

Let $\Sigma_{g}$ be a surface of genus $g$. Then for $g=1$ the above simplifies to
\ben
\det \left(H^1(\Sigma_1 ; \cT_{\Sigma_1})\right) \tensor \det \left(H^0(\Sigma_1 ; K) \right)^{-14} .
\een 
If $g \geq 2$ one has
\ben
\det \left(H^1(\Sigma_1 ; \cT_{\Sigma_1})\right) \tensor \det \left(H^0(\Sigma_1 ; K) \right)^{-13} .
\een
It is clear that the spectral sequence degenerates at this page. Thus the above expressions give the global observables for the holomorphic string for genus $g =1$ and $g \geq 2$, respectively. 

\section{The string partition function on an elliptic curve}

\brian{Sketch how to see the mumford form}

\owen{If possible, it would be cool to explain how one can extract the differential equations (=flat connection) governing the partition function from our construction. This might be too hard right now \dots}
