
\section{Introduction}


Two intertwined goals govern our exposition.
First, we want to describe a two-dimensional field theory,
which we view as a holomorphic version of bosonic string theory,
and its perturbative quantization.
We'll see that this theory encodes the chiral sector of a bosonic string with linear target space,
justifying our interpretation.
Second, we want to use this theory as the running example for key ideas and techniques in the formalism for quantum field theory developed by Costello and collaborators \cite{CosBook, CG1,CG2, LL1, GG1, GLL, LiVA}.
We hope to give readers a feel for how to use this formalism by exhibiting it with a beautiful theory.

Our focus is thus on narrative rather than detailed argumentation.
That is, we work systematically according the natural flow of the formalism. 
Along the way we emphasize the motivations behind each step rather than the nitty-gritty computations. 
Precedence is given to communicating the essence of an argument, over spelling everything out.
We do give detailed citations where such arguments can be found in the literature,
but we defer some not-yet-extant details to a forthcoming work on this theory with curved target space~\cite{GWcurved}.

None of the results here about string theory is new, 
as the bosonic string has been under intensive study for several decades,
but this formalism recovers them in a single, systematic process,
often giving a novel argument or perspective.
It is compelling to have a direct path from the action functional to such sophisticated constructions as the semi-infinite cohomology of a vertex algebra.
In fact, since so many of these results are familiar,
the reader may see more clearly what's distinctive and illuminating about this approach to field theory.

There are many references on the bosonic string that have influenced us.
In the physics literature there are the classic sources \cite{GSW1, GSW2, polchinski} that explain perturbative string theory. 
In addition, there is an extensive mathematically-oriented treatment of perturbative string theory in \cite{DP}, as well as D'Hoker's notes in Volume II of \cite{IAScourse}.
Our approach, while intimately related, starts with a ``first-order" description of the bosonic string. 

Given the vastness of the string theory literature,
it should not be a surprise that there is already work along these lines,
notably by Losev, Marshakov, and Zeitlin \cite{LMZ}.
One could view this paper as attempting to communicate many of their insights to those with an intuition growing out of homotopical algebra and the functorial approach to geometry.
Again, we note that the formalism of Costello provides a mathematical articulation and verification of many ideas long known to physicists, such as the Wilsonian view of renormalization and the Batalin-Vilkovisky (BV) approach to gauge and gravity theories.\footnote{We also note that given the literature's size,
and our relative and unfortunate ignorance of much of it,
we have chosen to mention a reference when we feel its description is particularly useful for us, 
even if it is not the original or standard reference for the result.}
This machinery allows us to revisit such prior work in a manner particularly amenable to mathematicians.

\subsection{Overview} \label{sec:bvoverview}

The central figure of this paper is a holomorphic analogue of the bosonic string.
We proceed, as usual in physics, from the classical to the quantum.

Hence, we begin by introducing the classical theory, 
expressed both in the BV formalism and also in terms of an action functional.
We take some time to identify this theory as the chiral sector of a limit of the bosonic string,
where the K\"{a}hler metric of the target is made very large. 
We also interpret the theory in the language of derived geometry.

We then turn to analyzing the deformations of this classical theory,
which by Costello's work admits a nice description in terms of a type of Gelfand-Fuks cohomology.
This perspective naturally leads to a discussion of string backgrounds.

With a firm grip on the classical theory, we turn to constructing the perturbative quantization.
We first work with a disk or $\CC$ as the source manifold,
and we review relevant features of Costello's approach to renormalization.
The usual dimensional Weyl anomaly appears as an obstruction to satisfying the quantum master equation,
a key condition in the BV formalism.
At this stage, the anomaly appears as a computation with Feynman diagrams.

The next section describes the vertex algebra of the quantized theory,
using the machinery of factorization algebras of \cite{CG1, CG2}.
We find this piece of the formalism particularly illuminating,
as it lets a mathematician understand how to read off the OPE from path integral manipulations.

We then turn to the case of a compact Riemann surface as the source manifold.
Here we discuss how the formalism relates to the global approach to computing anomalies using, for instance, the Grothendieck-Riemann-Roch formula.
We also discuss conformal blocks in this formalism.

Finally, we sketch how to modify the approach here to allow a complex manifold as the target.
This paper can be viewed as an expository precursor to future work,
which pushes into new territory (particularly in describing the vertex algebra).

\subsection{Lessons to bear in mind}

Before turning to our example,
we want to expound some key ideas of the Costello formalism so that the reader is alert to them when proceeding through the text.
That is, we wish to articulate here the structural features of this BV/renormalization package that make the arguments below conceptual.

For instance, in a gauge theory we know that connections provide the ``naive'' fields and that one must identify connections that are gauge-equivalent.
A mathematician would say the true fields are a {\em stacky} quotient of the naive fields.
Similarly, the critical locus of the action functional $S$ is the zero locus of its differential $\d S$ (ignoring some subtleties of the variational set-up),
which is the intersection of $\d S$ with the zero section of the cotangent space of the fields.
But in mathematics it is better to take {\em derived} intersections.

\begin{lesson}[Part 1, \cite{CG2}]
The classical BV formalism is a method for computing the derived critical locus of the action functional on the derived stack of fields.
Ghosts appear to describe the direction along which one quotients---the stacky direction---while the antifields appear to describe the direction along which one intersects---the derived direction.
\end{lesson}

We will describe our theory in the usual way, involving fields and ghosts, 
but we will also sketch its meaning in terms of global derived geometry,
which we find illuminates the deep connections between string theory and algebraic geometry.

Path integral quantization amounts to trying to put a kind of measure or volume form on the derived stack of fields.
When the fields form a linear space, 
there is a natural quantization that is translation-invariant along the fields,
which is the analogue of the Lebesgue measure on an ordinary vector space.

\begin{lesson}[\cite{GH}]
Linear BV quantization is functorial, and it behaves much like a determinant functor.
Hence, when one takes the fiberwise quantization of a family of linear theories,
one typically obtains a determinant line bundle over the base.
\end{lesson}

This situation is relevant to us because the theory we study arises from a simple free theory,
the free $\beta\gamma$ system, which lives on any Riemann surface.
Hence the quantization of the free $\beta\gamma$ system makes sense over the moduli of Riemann surfaces and naturally produces a line bundle.

To be more specific, our classical theory of interest arises by gauging the natural action of holomorphic vector fields on the free $\beta\gamma$ system.
As holomorphic vector fields are infinitesimal biholomorphisms, 
one can say that we couple the $\beta\gamma$ system to holomorphic gravity.
But then we recognize a natural consequence of our prior lessons.

\begin{lesson}[\S 5.11, \cite{CosBook}]
Gauging a classical theory corresponds to taking a stacky quotient of the original fields. 
To quantize the gauged theory corresponds to descending the quantization to the quotient.
Hence, an anomaly that prevents quantization should be understood as an obstruction to descent.
\end{lesson}

The formalism of Costello makes this relationship manifest, 
as the anomaly that appears in trying to produce a BV quantization---which is a Feynman diagram construction---is a cocycle in a dg Lie algebra determined by the classical field theory.
Thus, the anomaly determines an element of a natural Lie algebra cohomology group (in this case, Gelfand-Fuks cohomology),
whose descent-theoretic meaning is typically easy to recognize. 
Here we will discover the famed Weyl, or conformal, anomaly, which requires the target space to be real 26-dimensional. 

Anomalies are often characteristic classes, and this BV/renormalization package offers a structural explanation.
Most classical field theories---at least most of broad interest---make sense on a class of manifolds,
and so the anomaly ought to be determined by the local geometry of this class.
In more mathematical language we have the following.

\begin{lesson}[\cite{GGW}]
If a classical theory determines a sheaf on some site of manifolds (such as the site of Riemann surfaces and local biholomorphisms), 
then to quantize the theory over the whole site, 
it suffices to check on a generating cover (typically given by disks with geometric structure) but compatibly with all automorphisms.
\end{lesson}

In particular, the BV anomaly is a cocycle for the Lie algebra of automorphisms of the {\em formal} disk equipped with such geometric structures.
In other words, it lives in some kind of Gelfand-Fuks cohomology, which gives deep and informative connections with foliation theory and topology.

So far, everything we have mentioned is well-known in field theory, 
albeit often expressed in a different dialect of mathematics.
We now turn to the main new notion of this framework:
factorization algebras, which provide an efficient and powerful way to organize the local-to-global structure of the observables of a field theory.

\begin{lesson}[\cite{CG1,CG2}]
Every BV theory produces a factorization algebra. 
The local structure encodes the OPE algebra, so that for a chiral CFT, one recovers a vertex algebra. 
On compact manifolds, the global structure often has finite-dimensional cohomology because solutions to the equations of motion are typically finite-dimensional.
For a chiral CFT, one recovers the conformal blocks in this way.
\end{lesson}

A technical result of \cite{CG1} gives a precise articulation of this lesson,
and we will apply it to identify the vertex algebra arising from our holomorphic version of the bosonic string.

\subsection{Acknowledgements}

We learned this approach to perturbative field theory as students of Kevin Costello.
He guided us towards this theory of the holomorphic string,  
and he pointed out key results and features visible through this BV/renormalization formalism. 
OG spent some time on this theory in graduate school,
partly in collaboration with Yuan Shen,
whom he thanks for illuminating discussions and computations.
The authors also wish to thank Si Li for his typical incisive comments and insight on CFT,
which clarified some of the trickier technical aspects.
Finally, we must express deep gratitude toward the Max Planck Institute for Mathematics,
which supported OG throughout his work on this project and which also allowed BW to visit repeatedly and thus contributed substantially to the efficacy of this collaboration.

