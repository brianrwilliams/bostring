\appendix

\section{Calculation of the anomaly} \label{sec:calculation}

In this section we compute the functionals $F[L]$ and $G[L]$ mentioned in the proof of Proposition \ref{prop anomaly}, hence completing the calculation of the anomaly. 

We have reduced the calculation to the weight of two wheel diagrams: A) with internal edges labeled by the $bc$ heat kernel and propagator, respectively. B) with internal edges labeled by the $\beta\gamma$ heat kernel and propagator, respectively.
The weight of A gives the functional we called $F[L]$, and the weight of B gives the functional we called $\dim_\CC(V) G[L]$. 

We will utilize the following version of Wick expansion to evaluate the integrals below. 

\begin{lem}
\label{lem wick} 
Let $\Phi$ be a smooth compactly supported function on $\CC$ and let $\tau > 0$. 
Then
\ben
\int_{\xi \in \CC} \Phi(\xi) e^{-\tau |\xi|^2/4}  = 4 \pi  \tau^{-1} \left(\exp\left(\tau^{-1} \frac{\partial}{\partial \xi} \frac{\partial}{\partial \xi} \Phi\right)_{\xi = 0}\right) .
\een
\end{lem}

Note that we suppress the term $\d^2 \xi$ from the integral, for brevity's sake.
As we are integrating over vector spaces here, 
one can recover the integrand by taking the Lebesgue measure for the variable labeled under the integral sign (e.g., $z \in \CC$ corresponds to~$\d^2 z$).

We now turn to the weight of diagram A. 
Use coordinates $z,w$ to denote the coordinates at each of the vertices.
Denote the inputs of the weight by the compactly supported vector fields $f(z) \partial_z$ and $g(w) \d \wbar \partial_w$.
(Note that the diagram is only nonzero if the total degree of the elements is $+1$.)
If $c(z) \partial_z$ is another vector field, the action by $f(z) \partial_z$ is given by 
\ben
[f(z) \partial_z, c(z) \partial_z] = f(z) \partial_z c(z) \partial_z - c(z) \partial_z f(z) \partial_z .
\een 
Thus, the weight of diagram $A$ can be written as the $\ell \to 0$ limit of
\be
\begin{array}{ccc}
\displaystyle
& & \int_{z,w} f(z) \partial_z P_{\ell}^L(z,w) g(w) \partial_w K_\ell(z,w) \\
&-& \int_{z,w} \partial_z f(z) P_{\ell}^L(z,w) g(w) \partial_w K_\ell (z,w) \\
&-& \int_{z,w} f(z) \partial_z P_\ell^L(z,w) \partial_w g(w) K_\ell (z,w) \\
&+& \int_{z,w} \partial_z f(z) P_\ell^L(z,w) \partial_w g(w) K_\ell (z,w) .
\end{array}
\ee
We label the integrals in each line above as I,II, III, IV, respectively. 

Using the form of the propagator in (\ref{propagator}) we see that line I is given by
\ben
{\rm I} = \frac{1}{(4 \pi)^2} \int_{(z,w) \in \CC \times \CC} \int_{t = \ell}^L f(z) g(w) \frac{1}{\epsilon^2} \frac{1}{t^3} \frac{(\zbar - \wbar)^3}{8} \exp \left(-\frac{1}{4}\left(\frac{1}{\ell} + \frac{1}{t}\right)|z-w|^2 \right)
\een
(we are omitting volume factors for simplicity). 
To evaluate this integral we change variables and apply the Wick expansion, Lemma \ref{lem wick} to one of the variables of integration. 
Indeed, introduce $\xi = z -w$, and notice that the integral simplifies to
\ben
{\rm I} = \frac{1}{(4 \pi)^2} \int_{w \in \CC} \int_{\xi \in \CC} \int_{t = \ell}^L f(\xi + w) g(w) \frac{1}{\epsilon^2} \frac{1}{t^3} \frac{\Bar{\xi}^3}{8} \exp \exp \left(-\frac{1}{4} \left(\frac{1}{\ell} + \frac{1}{t}\right)| \xi |^2 \right) .
\een
Applying Lemma \ref{lem wick} to the $\xi$-integral we see that this simplifies to
\ben
{\rm I} = \frac{1}{4 \pi} \int_{w \in \CC} \partial^3_w f(w) g(w) \int_{t = \ell}^L \frac{\ell^2 t}{(\ell + t)^4} + O(\ell)
\een
where the terms $O(\ell)$ are of order $\ell$ so are zero in the limit $\ell \to 0$. 
On the other hand, we can evaluate the remaining $t$-integral and see that in the limit $\ell \to 0$ Line I becomes
\ben
\lim_{\ell \to 0} \; {\rm I} = \frac{1}{4 \pi} \frac{1}{12} \int_{w\in \CC} \partial^3_w f(w) g(w) .
\een 

We evaluate II, III, and IV in a similar fashion.

After changing coordinates and performing the Wick type integral we obtain
\ben
{\rm II} = \frac{1}{4 \pi} \int_{w \in \CC} \partial^3_w f(w) g(w) \, \int_{t = \ell}^L \frac{\ell t}{(\ell + t)^3} + O(\ell) .
\een
Evaluating the remaining $t$ integral and taking $\ell \to 0$ this becomes 
\ben
\lim_{\ell \to 0} {\rm II} = \frac{1}{4 \pi} \frac{3}{8} \int_{w\in \CC} \partial^3_w f(w) g(w) .
\een 

Integral III is given by 
\ben
\frac{1}{4\pi} \int_{w \in \CC} \partial_w^3 f(w) g(w)\, \int_{t= \ell}^L \frac{\epsilon^2}{(\epsilon + t)^3}  + O(\ell) .
\een
In the limit $\ell \to 0$ we obtain
\ben
\lim_{\ell \to 0} \; {\rm III} = \frac{1}{4\pi} \frac{1}{8} \int_{w \in \CC}  \partial^3 f(w) g(w)  .
\een

Finally, integral IV is
\ben
\frac{1}{4\pi} \int_{w \in \CC} \partial_w^3 f(w) g(w) \, \int_{t= \ell}^L \frac{\epsilon}{(\epsilon + t)^2}  + O(\ell) .
\een
In the limit $\ell \to 0$ we obtain
\ben
\lim_{\ell \to 0} \; {\rm IV} = \frac{1}{4\pi} \frac{1}{2} \int_{w \in \CC}  \partial^3_w f(w) g(w) .
\een

In total, the functional $F[L]$ applied to $(f(z) \partial_z, g(w) \d \wbar \partial_w)$ is given by
\ben
F[L] (f(z) \partial_z, g(w) \d \wbar \partial_w) = - \frac{1}{4 \pi} \frac{13}{12} \int_{w \in \CC}  \partial^3_w f(w) g(w)  .
\een
Note that this functional is independent of $L$. 

Diagram B is similar to A, except the internal edges are labeled by the $\beta\gamma$ propagator. 
Applied to the input vector fields $(f(z) \partial_z, g(w) \d \wbar \partial_w)$ the weight is given by the dimension of $V$ times the integral we computed in $I$.
Thus
\ben
G[L](f(z) \partial_z, g(w) \d \wbar \partial_w) = \frac{1}{4 \pi} \frac{1}{12} \int_{w\in \CC} \partial^3_w f(w) g(w)  .
\een
The proposition follows.
