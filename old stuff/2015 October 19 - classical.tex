\documentclass[10pt]{amsart}

\usepackage{macros}

\def\brian{\textcolor{blue}}
\def\owen{\textcolor{red}}

\title{The holomorphic bosonic string: classical theory}
\begin{document}
\maketitle

\section{Sigma-models in the BV-formalism}

\subsection{Working over a dg-ring}
\brian{everything lives over $\Omega^*_X$.}

\section{The local Lie algebra}
\brian{Owen's description of local Lie algebra as a square zero
  extension of the derived quotient
  $\fg_X^S // \aut(S)_X$.}

\brian{Write down the classical field theory we consider}

Fix a source Riemann surface (possibly non-compact) $S$. 

The local Lie algebra is the following cochain complex
\[
\Dol{0}(S,T^{1,0}_S) \oplus \Dol{1}(S, T^{\ast 1,0}_S) \oplus \Dol{0}(S) \ot \fg_X \oplus \Dol{1}(S) \ot \fg_X^\vee,
\]
equipped with a Lie algebra structure that we now describe. The first summand is the Dolbeault complex of the holomorphic vector fields on $S$, and it is both a sub and a quotient Lie algebra. The second summand has no internal Lie structure but has the natural action of holomorphic vector fields. The third summand is a (complicated and curved) $L_\infty$ algebra, inherited from that for $\fg_X$; it also receives a natural action from the holomorphic vector fields. The final term is a module for both the first and third terms.

Let's introduce some notation to clarify the exposition. We write
\[
\aut(S) := \Dol{0}(S,T^{1,0}_S)
\]
for the Dolbeault complex of holomorphic vector fields on $S$, as these are the infinitesimal automorphisms of $S$ as a Riemann surface. We write
\[
\fg_X^S := \Dol{0}(S) \ot \fg_X
\]
for the curved $L_\infty$ algebra (over base ring $\Omega^\ast(X)$) that describes the formal neighborhood of the constant maps inside the space of all holomorphic maps. \owen{That description is perhaps a little unclear.} Note that $\aut(S)$ acts on $\fg_X^S$ by extending the natural action of holomorphic vector fields on holomorphic functions. We make the situation more symmetric with respect to $S$ and $X$ by working with 
\[
\aut(S)_X := \Omega^*(X) \otimes \aut(S),
\]
which is locally constant on $X$. 

Consider the Lie algebra
\[
\fg_X^S//\mathfrak{aut}(S)_X := \aut(S)_X \ltimes \fg_X^S. 
\]
which describes the derived quotient by holomorphic automorphisms of $S$ of the formal neighborhood of constant maps from $S$ to $X$. (What a mouthful!) Note that the automorphisms of $S$ are both a sub and quotient Lie algebra of this whole thing.

The Lie algebra describing the holomorphic bosonic string is the split square zero extension of this Lie algebra by its coadjoint module.

We want to quantize this \emph{cotangent theory}, which has several simplifying consequences. First, a cotangent quantization only involves one-loop contributions -- i.e., $\hbar$ additions to the differential -- and so it is much more computable if it exists. Second, the deformation complex as a cotangent theory is 
\[
C^*_{loc,red}(\fg_X^S//\mathfrak{aut}(S)_X),
\]
so that we can ignore the coadjoint summands.

The structure of this Lie algebra, as an extension of the holomorphic vector fields, induces a natural retract of commutative dg algebras
\[
\Omega^\ast_X \ot_\CC C^*_{loc}(\aut(S)) \to C^*_{loc}(\fg_X^S//\mathfrak{aut}(S)) \to \Omega^\ast_X \ot_\CC C^*_{loc}(\aut(S)).\footnote{Note that this is a retract on the nose. Note also that $C^*(\aut(S)_X) \cong \Omega^*(S) \ot C^*(\aut(S))$ by base change.}
\]
\owen{Is this true? I'm worried about how things interact with the curving \dots} Thus
\[
C^*_{loc}(\fg_X^S//\mathfrak{aut}(S)) \cong \Omega^\ast_X \ot_\CC C^*_{loc}(\aut(S)) \oplus C^*_{loc}(\aut(S)_X, C^*_{red}(\fg^S_X))
\]
Thus, we obtain a useful splitting:
\[
C^*_{loc,red}(\fg_X^S//\mathfrak{aut}(S)_X) \cong \Omega^\ast_X \ot_\CC C^*_{red,loc}(\aut(S)) \oplus C^*_{loc}(\aut(S)_X, C^*_{red}(\fg^S_X)).
\]
The first summand is where the obstruction for the string with flat target lives; in particularly, we will see here the same ``conformal anomaly'' appear.

We will now examine the second summand, where a dependence on $X$ appears.

\section{The classical field theory}

\section{An infinite volume limit}
\brian{Motivate our theory by considering infinite volume limit of the
  Polyakov action} 


\end{document}