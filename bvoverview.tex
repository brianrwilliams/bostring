\section{From classical to quantum: anomalies in the BV formalism}

\brian{A rapid overview of classical BV and effective quantizations. Stress how obstructions appear, where they live, and how to compute them.}

\owen{I think we should articulate here the structural features of our BV package that make the arguments below more conceptual. For instance:
\begin{itemize}
\item Linear BV quantization is determinantal, which explains why we'll produce determinant line bundles when we do free $\beta\gamma$ system.
\item ``Gauging'' a theory corresponds to a stacky quotient of the original fields. Hence, obstruction to quantizing a gauged theory corresponds to descending the quantization to the quotient.
\item If a classical theory makes sense on a class (=site) of manifolds, then to quantize the whole class, it suffices to check on a generating cover (typically given by disks with geometric structure) but compatibly with all automorphisms. This often explains the appearance of characteristic classes as anomalies.
\item Every BV theory produces a factorization algebra. The local structure encodes the OPE algebra (and hence recovers a vertex algebra in chiral CFT situation). On compact manifolds, solutions to EoM typically form finite-dimensional space, and the global observables encode a volume form on this space. (An example is conformal blocks for the free $bc\beta\gamma$ system.)
\end{itemize}
Please add others as you think of them!}

\owen{We might also add that we view the BV formalism as the analogue in field theory of derived geometry in geometry. That is, in ordinary algebraic geometry, one first builds geometry and then adds (sheaf) cohomology on top: in ordinary physics, one first builds field theories and then adds (BRST) cohomology on top. But derived geometry (respectively, BV formalism) builds the cohomological aspect into the foundations.}
