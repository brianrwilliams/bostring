\documentclass{conm-p-l}

\usepackage{macros}

\newtheorem{theorem}{Theorem}[section]
\newtheorem{lemma}[theorem]{Lemma}

\theoremstyle{definition}
\newtheorem{definition}[theorem]{Definition}
\newtheorem{example}[theorem]{Example}
\newtheorem{xca}[theorem]{Exercise}

\theoremstyle{remark}
\newtheorem{remark}[theorem]{Remark}

\numberwithin{equation}{section}



\def\brian{\textcolor{blue}{BW: }\textcolor{blue}}
\def\owen{\textcolor{red}{OG: }\textcolor{red}}

\renewcommand{\contentsname}{}

\begin{document}

\title{The holomorphic bosonic string}

\author{Owen Gwilliam}
\address{Max Planck Institute for Mathematics, Bonn, Germany}
\email{gwilliam@mpim-bonn.mpg.de}
%\thanks{}

\author{Brian Williams}
\address{Department of Mathematics, Northwestern University, Evanston, IL}
\subjclass[2010]{Primary 81T30, 81T70, Secondary 58D27, 17B69}

\date{}

\begin{abstract}
\owen{I modestly rewrote. Let me know what you think.}
We describe and analyze a holomorphic version of the bosonic string in the formalism of quantum field theory developed by Costello and collaborators, which provides a powerful combination of renormalization theory and the Batalin-Vilkovisky formalism. Our focus here is on the case in which the target space-time is a vector space. We identify the critical dimension as an obstruction to satisfying the quantum master equation, and when the obstruction vanishes, we construct a one-loop exact quantization. Moreover, we show how the factorization algebra of observables recovers the BRST cohomology of the string and use this perspective to give a new construction of its Gerstenhaber structure. Finally, we show how the factorization homology along closed manifolds encodes the determinant line bundle over the moduli space of Riemann surfaces.
An auxiliary goal of this paper is to give an exposition of this QFT formalism with the holomorphic bosonic string theory as the running example.
\end{abstract}

\maketitle


%\section*{Contents}
\setcounter{tocdepth}{1}

\tableofcontents



\section{Introduction}

Two intertwined goals govern this paper.
\owen{Can't quite get this first sentence right ...}
First, we want to describe a two-dimensional field theory,
which we view as a holomorphic version of bosonic string theory,
and its perturbative quantization.
We will see this theory encodes the chiral sector of a bosonic string with linear target space,
justifying our interpretation.
Second, we want to use this theory as the running example for key ideas and techniques in the formalism for quantum field theory developed by Costello and collaborators \cite{CG1,CG2, LL1, GG1, GLL, LiVA}.
We hope to give readers a feel for how to use this formalism by exhibiting it with a beautiful theory.

Our focus is thus on narrative rather than detailed argumentation.
That is, we work systematically in the natural order of the formalism. 
Along the way we emphasize the motivations behind each step rather than the nitty-gritty computations. 
Precedence is given to communicating the essence of an argument, over spelling everything out.
We do give detailed citations where such arguments can be found in the literature,
but we defer some not-yet-extant details to a forthcoming work on this theory with curved target space~\cite{GWcurved}.

None of the results here about string theory is new, 
as the bosonic string has been under intensive study for several decades,
but this formalism recovers them in a single, systematic process,
often giving a novel argument or perspective.
It is compelling to have a direct path from the action functional to such sophisticated constructions as the semi-infinite cohomology of a vertex algebra.
In fact, since so many of these results are familiar,
the reader may see more clearly what's distinctive and illuminating about this approach to field theory.

There are many references on the bosonic string that have influenced us.
In the physics literature there are the classic sources \cite{GSW1, GSW2, polchinski} that explain perturbative string theory. 
In addition, there is an extensive mathematically-oriented treatment of perturbative string theory in \cite{DP}, as well as D'Hoker's notes in Volume II of \cite{IAScourse}.
Our approach, while intimately related, starts with a ``first-order" description of the bosonic string. 

Given the vastness of the string theory literature,
it should not be a surprise that there is already work along these lines,
notably by Losev, Marshakov, and Zeitlin \cite{LMZ}.
One could view this paper as attempting to communicate many of their insights to those with an intuition growing out of homotopical algebra and the functorial approach to geometry.
Again, we note that the formalism of Costello provides a mathematical articulation and verification of many ideas long known to physicists, such as the Wilsonian view of renormalization and the Batalin-Vilkovisky (BV) approach to gauge and gravity theories.\footnote{We also note that given the literature's size,
and our relative and unfortunate ignorance of much of it,
we have chosen to mention a reference when we feel its description is particularly useful for us, 
even if it is not the original or standard reference for a given result.}
This machinery allows us to revisit such prior work in a manner particularly amenable to mathematicians.

\subsection{Overview} \label{sec:bvoverview}

The central figure of this paper is a holomorphic analogue of the bosonic string.
We proceed, as usual in physics, from the classical to the quantum.

Hence, we begin by introducing the classical theory, 
expressed both in the BV formalism and also in terms of an action functional.
We take some time to identify this theory as the chiral sector of a limit of the bosonic string,
where the K\"{a}hler metric of the target is made very large. 
We also interpret the theory in the language of derived geometry.

We then turn to analyzing the deformations of this classical theory,
which by Costello's work admits a nice description in terms of a type of Gelfand-Fuks cohomology.
This perspective naturally leads to a discussion of string backgrounds.

With a firm grip on the classical theory, we turn to constructing the perturbative quantization.
We first work with a disk or $\CC$ as the source manifold,
and we review relevant features of Costello's approach to renormalization.
The usual dimensional Weyl anomaly appears as an obstruction to satisfying the quantum master equation,
a key condition in the BV formalism.
At this stage, the anomaly appears as a computation with Feynman diagrams.

The next section describes the vertex algebra of the quantized theory,
using the machinery of factorization algebras of \cite{CG1, CG2}.
We find this piece of the formalism particularly illuminating,
as it lets a mathematician understand how to read off the OPE from path integral manipulations.

We then turn to the case of a compact Riemann surface as the source manifold.
Here we discuss how the formalism relates to the global approach to computing anomalies using, for instance, the Grothendieck-Riemann-Roch formula.
We also discuss conformal blocks in this formalism.

Finally, we sketch how to modify the approach here to allow a complex manifold as the target.
This paper can be viewed as an expository precursor to future work,
which pushes into new territory (particularly in describing the vertex algebra).

\subsection{Lessons to bear in mind}

Before turning to our example,
we want to expound some key ideas of the Costello formalism so that the reader is alert to them when proceeding through the text.
That is, we wish to articulate here the structural features of this BV/renormalization package that make the arguments below conceptual.

For instance, in a gauge theory we know that connections provide the ``naive'' fields and that one must identify connections that are gauge-equivalent.
A mathematician would say the true fields are a {\em stacky} quotient of the naive fields.
Similarly, the critical locus of the action functional $S$ is the zero locus of its differential $\d S$ (ignoring some subtleties of the variational set-up),
which is the intersection of $\d S$ with the zero section of the cotangent space of the fields.
But in mathematics it is better to take {\em derived} intersections.

\begin{lesson}[Part 1, \cite{CG2}]
The classical BV formalism is a method for computing the derived critical locus of the action functional on the derived stack of fields.
Ghosts appear to describe the direction along which one quotients---the stacky direction---while the antifields appear to describe the direction along which one intersects---the derived direction.
\end{lesson}

We will describe our theory in the usual way, involving fields and ghosts, 
but we will also sketch its meaning in terms of global derived geometry,
which we find illuminates the deep connections between string theory and algebraic geometry.

Path integral quantization amounts to trying to put a kind of measure or volume form on the derived stack of fields.
When the fields form a linear space, 
there is a natural quantization that is translation-invariant along the fields,
which is the analogue of the Lebesgue measure on an ordinary vector space.

\begin{lesson}[\cite{GH}]
Linear BV quantization is functorial, and it behaves much like a determinant functor.
Hence, when one takes the fiberwise quantization of a family of linear theories,
one typically obtains a determinant line bundle over the base.
\end{lesson}

This situation is relevant to us because the theory we study arises from a simple free theory,
the free $\beta\gamma$ system, which lives on any Riemann surface.
Hence the quantization of the free $\beta\gamma$ system makes sense over the moduli of Riemann surfaces and naturally produces a line bundle.

To be more specific, our classical theory of interest arises by gauging the natural action of holomorphic vector fields on the free $\beta\gamma$ system.
As holomorphic vector fields are infinitesimal biholomorphisms, 
one can say that we couple the $\beta\gamma$ system to holomorphic gravity.
But then we recognize a natural consequence of our prior lessons.

\begin{lesson}[\S 5.11, \cite{CosBook}]
Gauging a classical theory corresponds to taking a stacky quotient of the original fields. 
To quantize the gauged theory corresponds to descending the quantization to the quotient.
Hence, an anomaly that prevents quantization should be understood as an obstruction to descent.
\end{lesson}

The formalism of Costello makes this relationship manifest, 
as the anomaly that appears in trying to produce a BV quantization---which is a Feynman diagram construction---is a cocycle in a dg Lie algebra determined by the classical field theory.
Thus, the anomaly determines an element of a natural Lie algebra cohomology group (in this case, Gelfand-Fuks cohomology),
whose descent-theoretic meaning is typically easy to recognize. 
Here we will discover the famed Weyl, or conformal, anomaly, which requires the target space to be real 26-dimensional. 

Anomalies are often characteristic classes, and this BV/renormalization package offers a structural explanation.
Most classical field theories---at least most of broad interest---make sense on a class of manifolds,
and so the anomaly ought to be determined by the local geometry of this class.
In more mathematical language we have the following.

\begin{lesson}[\cite{GGW}]
If a classical theory determines a sheaf on some site of manifolds (such as the site of Riemann surfaces and local biholomorphisms), 
then to quantize the theory over the whole site, 
it suffices to check on a generating cover (typically given by disks with geometric structure) but compatibly with all automorphisms.
\end{lesson}

In particular, the BV anomaly is a cocycle for the Lie algebra of automorphisms of the {\em formal} disk equipped with such geometric structures.
In other words, it lives in some kind of Gelfand-Fuks cohomology, which gives deep and informative connections with foliation theory and topology.

So far, everything we have mentioned is well-known in field theory, 
albeit often expressed in a different dialect of mathematics.
We now turn to the main new notion of this framework:
factorization algebras, which provide an efficient and powerful way to organize the local-to-global structure of the observables of a field theory.

\begin{lesson}[\cite{CG1,CG2}]
Every BV theory produces a factorization algebra. 
The local structure encodes the OPE algebra, so that for a chiral CFT, one recovers a vertex algebra. 
On compact manifolds, the global structure often has finite-dimensional cohomology because solutions to the equations of motion are typically finite-dimensional.
For a chiral CFT, one recovers the conformal blocks in this way.
\end{lesson}

A technical result of \cite{CG1} gives a precise articulation of this lesson,
and we will apply it to identify the vertex algebra arising from our holomorphic version of the bosonic string.

\subsection{Acknowledgements}

We learned this approach to perturbative field theory as students of Kevin Costello.
He guided us towards this theory of the holomorphic string,  
and he pointed out key results and features visible through this BV/renormalization formalism. 
OG spent some time on this theory in graduate school,
partly in collaboration with Yuan Shen,
whom he thanks for illuminating discussions and computations.
The authors also wish to thank Si Li for his typical incisive comments and insight on CFT,
which clarified some of the trickier technical aspects.


\section{The classical holomorphic bosonic string}

\brian{First define the holomorphic theory we will work with. Then show how it's related to more familiar models for the string, eg the Polyakov action. Level of detail depending on the space we have.}

There is a basic format for a string theory, at least in the perturbative approach. 
One starts with a nonlinear $\sigma$-model, 
whose fields are smooth maps from a Riemann surface to a target manifold $X$;
in this setting we want the theory to make sense for an arbitrary Riemann surface as the source manifold.
In the usual bosonic string theory, 
this nonlinear $\sigma$-model picks out the harmonic maps from a Riemannian 2-manifold to a Riemannian manifold.
In our holomorphic setting,
the nonlinear $\sigma$-model picks out holomorphic maps from a Riemann surface to a complex manifold.
One then quotients the space of fields (and solutions to the equations of motion) with respect to reparametrization.
\owen{This description is a bit opaque. We should find a better one.}
In the usual bosonic string,
one quotients by diffeomorphisms, which can thus change the metric on the source.
In our setting,
we quotient by diffeomorphisms as well, which can thus change the complex structure on the source.

In this section we begin by describing our theory in the BV formalism.
We do not expect the reader to find the action functional immediately clear,
so we devote some time to analyzing what it means and how it arises from concrete questions.
We then turn to interpreting this classical BV theory using dg Lie algebras and derived geometry
(i.e., we identify the moduli space it encodes).
Finally, we conclude by sketching how our theory appears as the chiral sector of a degeneration of the usual bosonic string when the target is a complex manifold with a Hermitian metric.
Our theory thus does provide insights into the usual bosonic string;
moreover, it clarifies why so many aspects of the bosonic string,
like the anomalies or $B$-fields, 
have holomorphic analogues.

\subsection{The theory we study} 

Let $V$ denote a complex vector space (the target),
and let $\langle-,-\rangle_V$ denote the evaluation pairing between $V$ and its linear dual~$V^\vee$.
Let $\Sigma$ denote a Riemann surface (the source).
Let $T_\Sigma^{1,0}$ denote the holomorphic tangent bundle on $\Sigma$, 
let $\langle-,-\rangle_T$ denote the evaluation pairing between $T_\Sigma^{1,0}$ and its vector bundle dual~$T_\Sigma^{1,0*}$. 
\owen{Correct terminology?}
These are the key geometric inputs.

In a BV theory, the fields are $\ZZ$-graded;
we call this the {\em cohomological grading}.
We have four kinds of fields:
\[
\begin{array}{ccccc}
\text{field} & -1 & 0 & 1 & 2\\
\hline
\gamma & & \Omega^{0,0}(\Sigma) \otimes V & \Omega^{0,1}(\Sigma) \otimes V & \\
\beta & & \Omega^{1,0}(\Sigma) \otimes V^\vee & \Omega^{1,1}(\Sigma) \otimes V^\vee & \\
c & \Omega^{0,0}(\Sigma, T^{1,0}_\Sigma) & \Omega^{0,1}(\Sigma, T^{1,0}_\Sigma) & \\
b & & & \Omega^{1,0}(\Sigma, T^{1,0 *}_\Sigma) & \Omega^{1,1}(\Sigma, T^{1,0 *}_\Sigma)
\end{array}
\]
More accurately, we have eight different kinds of fields, 
but we view each row as constituting a single type 
since each given row consists of the Dolbeault forms of a holomorphic vector bundle.
For instance, the field $\gamma$ is a $(0,*)$-form with values in the trivial bundle with fiber~$V$,
and the field $b$ is a $(0,*)$-form with values in the bundle $T^{1,0 *} \otimes~T^{1,0 *}$.

To orient oneself it is helpful to start by examining the fields of cohomological degree zero,
since these typically have a manifest physical meaning.
For instance, the degree zero $\gamma$ field is a smooth $V$-valued function
and hence the natural field for the nonlinear $\sigma$-model into~$V$.
The degree zero $c$ field is a smooth $(0,1)$-form with values in vector field ``in the holomorphic direction,''
and hence encodes an infinitesimal change of complex structure of~$\Sigma$.
They thus constitute the obvious fields to introduce for a holomorphic version of the bosonic string.
The fields a$\beta$ and $b$ are less obvious but appear as ``partners'' (or antifields)
whose role is clearest once we have the action functional and hence equations of motion.

The action functional is
\begin{equation}
S(\gamma,\beta,c,b) = 
\int_\Sigma \langle \beta, \dbar \gamma \rangle_V 
+ \int_\Sigma \langle b, \dbar c \rangle_T 
+ \int_\Sigma \langle \beta, [c,\gamma] \rangle_V 
+ \int_\Sigma \langle b, [c,c] \rangle_T.
\end{equation}
(We discuss below how to think about fields with nonzero cohomological degrees as inputs.)
The equations of motion are thus
\begin{alignat*}{2}
0 &= \dbar \gamma + [c,\gamma] & \quad\quad  0 &= \dbar \beta + [c,\beta] \\
0  &= \dbar c + \tfrac{1}{2} [c,c] & \quad\quad  0 &= \dbar b + [c,b] 
\end{alignat*}
Note that these equations are familiar in complex geometry.
For instance, the equation purely for $c$ encodes a deformation of complex structure on $\Sigma$; concretely, it modifies the $\dbar$ operator to $\dbar + c$.
The other equations then amount to solving for holomorphic sections (of the relevant bundle) withe respect to this deformed complex structure.
For instance, the equation in $\gamma$ picks out holomorphic maps from $\Sigma$,
with the $c$-deformed complex structure, to~$V$.

\owen{Add something about how to understand the degrees. E.g., does $b$ ever appear?}

\owen{Add explanation of writing BV theory from ordinary action.}

\begin{rmk}
\label{rmk:bcbg}
Just looking at this action functional, one might notice that if one drops the last two terms,
which are cubic in the fields, then one obtains a free theory
\[
S_{free}(\gamma,\beta,c,b) = 
\int_\Sigma \langle \beta, \dbar \gamma \rangle_V 
+ \int_\Sigma \langle b, \dbar c \rangle_T,
\]
which is known as the {\em free $bc\beta\gamma$ system}.
Thus, one may view the holomorphic bosonic string as a deformation of this free theory
by ``turning on'' those interaction terms.
We will repeatedly try a construction first with this free theory before tackling the string itself,
as it often captures important information with minimal work.
For instance, we will examine the vertex algebra for the free theory before seeing how the interaction affects the operator products.
Similarly, one can identify the anomaly already at the level of the free theory.
\end{rmk}

\begin{rmk}
\label{rmk:curved}
It is easy to modify this action functional to allow a curved target,
i.e., one can replace the complex vector space $V$ with an arbitrary complex manifold~$X$. 
The fields $b,c$ remain the same.
The degree 0 field $\gamma$ still encodes smooth maps into $X$, but now the degree 1 field is a section of $\Omega^{0,1}(\Sigma, \gamma^*T^{1,0}_X)$.
Similarly, $\beta$ is now a section of $\Omega^{1,*}(\Sigma, \gamma^*T^{1,0*}_X)$.
The action is then
\begin{equation}
S(\gamma,\beta,c,b) = 
\int_\Sigma \langle \beta, \dbar \gamma \rangle_{T_X}
+ \int_\Sigma \langle b, \dbar c \rangle_{T_\Sigma} 
+ \int_\Sigma \langle \beta, [c,\gamma] \rangle_{T_X}
+ \int_\Sigma \langle b, [c,c] \rangle_{T_\Sigma}.
\end{equation}
In Section \ref{sec:curved} we will indicate how the results with linear target generalize to this situation.
\end{rmk}

\subsection{From the perspective of derived geometry}

We would like to explain what this theory is about in more conceptual terms,
rather than simply by formulas and equations.
Thankfully this theory is amenable to such a description.
We will be informal in this section and not specify a particular geometric context (e.g., derived analytic stacks),
except when we specialize to the deformation-theoretic situation (i.e., perturbative setting) that is our main arena.

\def\Maps{\operatorname{Maps}}

Let $\cM$ denote the moduli space of Riemann surfaces,
so that a surface $\Sigma$ determines a point in~$\cM$.
Let $\Maps_{\dbar}(\Sigma,V)$ denote the space of holomorphic maps from $\Sigma$ to $V$,
and hence a bundle $\Maps_{\dbar}(-,V)$ over $\cM$ by varying~$\Sigma$.
For our equations of motion, the $\gamma$ and $c$ fields of a solution determine a point in this bundle~$\Maps_{\dbar}(-,V)$.

\def\RS{{\mathcal RS}}

This construction makes sense on noncompact Riemann surfaces as well.
Let $\RS$ denote the category whose objects are Riemann surfaces and whose morphisms are holomorphic embeddings.
There is a natural site structure: a cover is a collection of maps $\{S_i \to \Sigma\}_i$ such that the union of the images is all of~$\Sigma$.
Then $\Maps_{\dbar}(-,V)$ defines a sheaf of spaces over~$\RS$.
The observables for the classical theory is, in essence, the {\em co}\/sheaf of commutative algebras~$\cO(\Maps_{\dbar}(-,V))$,
and hence provides a factorization algebra.

In fact, it is better to work with the derived version of these spaces.
One important feature of derived geometry is that the appropriate version of a tangent space at a point is, in fact, a cochain complex.
In our setting, a point $(c,\gamma)$ in $\Maps_{\dbar}(-,V)$ determines a a complex structure $\dbar + c$ on $\Sigma$---we denote this Riemann surface by $\Sigma_c$---and $\gamma$ a $V$-valued holomorphic function on~$\Sigma_c$.
The tangent complex of $\Maps_{\dbar}(-,V)$ at $(c,\gamma)$ is precisely 
\[
\Omega^{0,*}(\Sigma_c,T^{1,0})[1] \oplus \Omega^{0,*}(\Sigma_c,V).
\]
The first summand is the usual answer from the theory of the moduli of surfaces 
(recall, for example, that the ordinary tangent space is the sheaf cohomology $H^1(\Sigma,\cT_\Sigma)$ of the holomorphic tangent sheaf),
and the second is usual elliptic complex encoding holomorphic maps.

\begin{rmk}
It is useful to bear in mind that the degree zero cohomology of the tangent complex will recover the ``naive'' tangent space. 
In our case, we have 
\[
H^1(\Sigma_c,\cT_{\Sigma_c}) \oplus H^0(\Sigma_c,V),
\]
which encodes deformations of complex structure and holomorphic maps.
Negative degree cohomology of the tangent complex detects infinitesimal automorphisms (and automorphisms of automorphisms, etc) of the space.
For instance, here we see $H^0(\Sigma_c,\cT_{\Sigma_c})$ appear in degree -1, 
since a holomorphic vector field is an infinitesimal automorphism of a complex curve.
These negative directions are called ``ghosts'' (or ghosts for ghosts, etc) in physics.
The positive degree cohomology detects infinitesimal relations (and relations of relations, etc).
For instance, here we see $H^1(\Sigma_c,V)$, the cokernel of~$\dbar+c$. \owen{???}
\end{rmk}

Note that the underlying graded spaces of this tangent complex are the $c$ and $\gamma$ fields from the BV theory described above.
We emphasize that the tangent complex is only specified up to quasi-isomorphism,
but it is compelling that a natural representative is the BV theory produced by the usual physical arguments.
This behavior, however, is typical of the relationship between derived geometry and BV theories:
when physicists write down a classical BV theory, 
the underlying free theory is essentially always the tangent complex of a nice derived stack.

The reader has probably noticed that, yet again, we have postponed discussing the $\beta$ and $b$ fields.
From a derived perspective, the full BV theory describes the shifted cotangent bundle $\TT^*[-1]\Maps_{\dbar}(-,V)$.
At the level of a tangent complex, the shifted cotangent direction contributes
\[
\Omega^{1,*}(\Sigma_c,T^{1,0*})[-1] \oplus \Omega^{1,*}(\Sigma_c,V^\vee),
\]
whose underlying graded spaces are the $\beta$ and $b$ fields.
These ``antifields'' are added so that the overall space of fields has a 1-shifted symplectic structure  when $\Sigma$ is closed, and a shifted Poisson structure when $\Sigma$ is open.

\subsection{Relationship to the Polyakov action functional}

This holomorphic bosonic string has a natural relationship with the usual bosonic string.
We sketch it briefly, only considering a linear target.

We begin with a bosonic string theory where the source is a 2-dimensional smooth oriented manifold $\Sigma$ and the target is a Hermitian vector space~$(V,h)$. 
The ``naive'' action functional is
\ben
S^{naive}_{Poly}(\varphi, g) = \int_\Sigma h(\varphi, \Delta_{g} \varphi)\, \dvol_g
\een
where the field $g$ is a Riemannian metric on $\Sigma$ and the field $\varphi$ is a smooth map from $\Sigma$ to~$V$.
The notation $\Delta_g$ denotes the Laplace-Beltrami operator on~$\Sigma$. 

Note that $S^{naive}_{Poly}$ is invariant under the diffeomorphism group ${\rm Diff}(\Sigma)$ and under rescalings of the metric
(i.e., the theory is classically conformal).
Typically we express rescaling as $g \mapsto e^{f} g$ with $f \in~C^\infty(\Sigma)$.
As we are interested in a string theory, we want to gauge these symmetries.
In geometric language, we want to think about the quotient stack 
obtained by taking solutions to the equations of motion and quotienting by these symmetry groups.

\owen{It might be better to explain the first-order description of sigma model before entering into the perturbative \& BV discussion.}

Our focus is perturbative, so that we want to study the behavior of this action near a fixed solution to the equations of motion
(e.g., the Taylor expansion of the true action near some solution).
\owen{Might be good here to leverage the derived discussion earlier:
it's easy to see what the tangent complex looks like \dots which leads to fields we work with.}
Hence, we fix a metric $g_0$ on $\Sigma$ and substitute for the field $g$,
the term $g_0+\alpha$ where $\alpha \in \Gamma(\Sigma,\Sym^2(T_\Sigma))$.
That is, we simply consider deformations of~$g_0$.
As $\varphi$ is linear, we just consider expanding around the zero map.
Thus our initial fields are $\varphi \in C^\infty(\Sigma,V)$ and $\alpha \in \Gamma(\Sigma,\Sym^2(T_\Sigma))$.

There are also ghost fields associated to the symmetries we gauge.
First, there are infinitesimal diffeomorphisms,  which are described by vector fields on~$\Sigma$.
We denote this ghost field by $X \in \Gamma(\Sigma,T_\Sigma))$.
It acts on the initial fields by the transformation 
\[
(\varphi,\alpha) \mapsto (\varphi + X \cdot \varphi, \alpha + L_X \alpha), 
\]
where $L_X$ denotes the Lie derivative on tensors.
Second, there are infinitesimal rescalings such as $\alpha \mapsto \alpha + f \alpha$, 
with ghost field $f \in~C^\infty(\Sigma)$.
The rescaling does not affect $\varphi$.
The two symmetries are compatible: 
given $f$ and $X$, then $L_{X} (f \alpha) = X(f) \alpha + f L_X \alpha$ for any $\alpha \in \Sym^2(T_\Sigma)$.

To summarize, we have the following fields:
\[
\begin{array}{cccc}
-1 &0 & 1 & 2\\
\hline \\
& \Omega^{0}(\Sigma) \tensor V & \Omega^2(\Sigma) \tensor V & \\
{\rm Vect}(\Sigma) \oplus C^\infty(\Sigma) & \Sym^2(T_\Sigma) & & \\
& & \Omega^2(\Sigma ; \Sym^{2}(T^*_\Sigma)) & \Omega^2(\Sigma ; T^*_\Sigma) \oplus \Omega^2(\Sigma) .
\end{array}
\]
The BV action functional is then
\[
S_{Poly}(\phi,\alpha,X,f) = \int_\Sigma h(\varphi, \Delta_{g_0 + \alpha} \varphi)\, \dvol_{g_0+\alpha} + \text{MORE STUFF}
\]
\owen{Finish writing action}
This situation is quite a bit more complicated than our holomorphic bosonic string,
but it admits a succinct description in geometric terms \owen{add statement about quotient of mapping stack} 

***********

In this section we start with a description of the classical Polyakov model for the bosonic string as a classical BV theory. 
This is the ordinary $\sigma$-model of maps $\Sigma \to V$ coupled to a metric on $\Sigma$. 
More precisely, this is a perturbative model for the Polyakov string, since we only look at deformations of the fixed metric $g_0$. We will show that after a reparametrization of the space of fields that it makes sense to take a certain ``infinite volume limit" as $h \to \infty$. In this limit we will show that the Polyakov model splits into a certain holomorphic theory plus its complex conjugate. The holomorphic theory is what we call the {\em holomorphic bosonic string}.

\begin{rmk} 
A similar analysis has appeared in \cite{GGW} where one does not consider deformations of the metric: the infinite volume limit of the bare $\sigma$-model of maps $\Sigma \to V$ splits into the free $\beta\gamma$ system plus its complex conjugate. In the case of the string we find an interacting theory that can be thought of as a deformation of a $\beta\gamma$ system. 
\end{rmk}

\owen{Give explanation of what this section will be about: writing down a holomorphic theory that appears as the chiral part of a large volume limit of the usual bosonic string. We should advertise that we start with conventional ways of writing a theory and explain the algorithm by which one extracts a BV action.}


We recall the most familiar form of the classical Polyakov string and show how to write it down in terms of a classical BV theory. The fields of the Polyakov model consist of a $C^\infty$ function $\varphi : \Sigma \to V$ and a metric $g$ on $\Sigma$. Since we are doing perturbation theory, we assume that $g$ is infinitesimally close to the fixed metric $g_0$ in the space of all metrics on $\Sigma$. There is an identification of the tangent space of the space of all metrics $T_{[g_0]} {\rm Met}(\Sigma) \cong \Sym^2(T_\Sigma)$. Thus, we can take the metric $g$ to be of the form $g = g_0 + \alpha$ where $\alpha \in \Sym^2(T_\Sigma)$. \owen{Should we include comments about "formal (derived) spaces"?}
 In the definition of a classical BV theory we must prescribe the data of a $(-1)$-shifted symplectic pairing on the BRST complex together with an interaction which is a local functional on the complex. The pairing can be described as follows. If $\varphi \in \Omega^0(\Sigma ; V)$ and $\psi \in \Omega^2(\Sigma ; V)$ then
\ben
\<\varphi, \psi \> = \int h(\varphi, \psi) .
\een 
The fields $(X, f) \in \Vect(\Sigma) \oplus C^\infty(\Sigma)$ pair with the conjugate fields $(X', f') \in  \Omega^2(\Sigma ; T^*_\Sigma) \oplus \Omega^2(\Sigma)$ via
\ben
\<(X,f), (X',f')\> = \int \ev(X, X') + \int f f' 
\een
where $\ev$ denotes the evaluation pairing between the tangent and cotangent bundles. 

\brian{Start with Polyakov action and explain how the chiral theory emerges in the infinite volume limit. There should also be an explanation for the theory we write down as a twist of 2d supergravity (in the same way that CDO's are a twist of a $(0,2)$ theory), not sure if you want to get into that.}
\owen{I don't know anything about the supergravity thing you mention. It sounds interesting.}
%
%\owen{When we write out the whole BV shebang, we should point out how it relates to the usual physics description. Namely, the physicists do the following:
%\begin{itemize}
%\item write a free $bc\beta\gamma$ system as a $\ZZ/2$-graded theory
%\item lift to a $\ZZ$-grading such that the $bc$ fields are ghosts/antighosts for holomorphic vector fields (with no action on $\beta\gamma$ fields or themselves)
%\item deform the action to encode the action of vector fields on functions etc.
%\end{itemize}
%We should then see our BV action on the nose.
%(I think this is correct, but we should double-check, of course.)
%One nice thing about this observation is that it verifies the identification with semi-infinite homology, which is often explained in these kinds of terms. 
%(See the nice, short, readable Voronov note I've put in our folder.)}


\section{Deformations of the theory and string backgrounds}
\label{sec: moduli}

\owen{Recall that Eugene said he felt we should say something about (1) why we bother to analyze the deformations and (2) what we mean by string backgrounds.}

\owen{Here's my attempt.}

Whenever one is studying a theory,
it is helpful to understand how it can be modified 
and how features of the theory change as one adjusts natural parameters of the theory,
such as coupling constants of the action functional.
Such manipulations often force one to think about the theory in more structural and qualitative terms.
In mathematical language, one is studying the theory in the moduli space of classical theories.

In this example, we will uncover natural parameters that are holomorphic analogs of parameters that appear in the usual bosonic string.
For instance, it is apparent that we could adjust the metric on the target,
but versions of the dilaton and $B$-fields also become manifest as we examine how we could vary the action.
A choice of values for these parameters is sometimes called a {\em string background},
as it picks out another classical string theory.
(Such features become crucial when one wants to study curved targets.)
These parameters have also spurred interesting developments in mathematics,
by animating abstract-seeming objects inside a physical problem.

The BV formalism provides an explicit tool for identifying these parameters,
called the {\em deformation complex} of the theory.
(A systematic discussion can be found in Chapter 5 of~\cite{CosBook}.)
This tool is homological because the BV formalism is homological.
In mathematical language, we recognize that the moduli space of classical BV theories is derived,
and so the tangent space of the theory (i.e., first-order deformations) becomes a tangent complex.

As a gloss, the underlying graded vector space of this deformation complex consists of the local functionals on the jets of fields, i.e., Lagrangian densities.
(Note that we allow local functionals of arbitrary cohomological degree.) 
There is also a shifted Lie bracket $\{-,-\}$, 
which arises from the pairing $\int_\Sigma \langle-,-\rangle$ on the fields.
It is, in essence, the shifted Poisson bracket corresponding to that shifted symplectic pairing on the fields.
The differential on the local functionals is then $\{S,-\}$, where $S$ is the classical action. 
All together, the deformation complex forms a shifted dg Lie algebra. 
Observe that if we find a degree zero element $I$ such that
\[
0=\{S +I,S +I\}=2\{S,I\}+\{I,I\},
\]
then $I$ is a shifted Maurer-Cartan element and 
hence determines a new classical BV theory whose action functional is $S + I$. 
In particular, degree 0 cocycles determine first-order deformations of the classical BV theory. Cocycles in degree -1 encode local symmetries of the classical theory; 
and obstructions to satisfying the quantum master equation end up being degree 1 cocycles.

In this section, we will explain why the deformation complex $\Def_{\rm string}$ of the holomorphic string 
can be expressed in terms of Gelfand-Fuks cohomology~\cite{Fuks}. 
Along the way we will show how the usual backgrounds for the bosonic string appear as elements in this complex of local functionals and hence as deformations of the classical action. 

Right now, we will focus on the case $\Sigma = \CC$, 
and in Section \ref{sec: conformalblock} we will consider arbitrary Riemann surfaces.
We restrict ourselves to examining {\em translation-invariant} local functionals (which will allow us to descend to a theory defined on an elliptic curve).
Unpacking what this means will lead swiftly to Gelfand-Fuks cohomology.

\subsection{Deformations for the classical theory}

As a local functional is given by integration of a Lagrangian density,
translation invariance requires the density to be the Lebesgue measure $\d^2 z$, up to rescaling,  
and requires the Lagrangian to be specified by its behavior at one point.
Hence, a translation-invariant local functional on $\CC$ is determined by a function of the jet (i.e., Taylor expansion) of the fields at the origin in~$\CC$. 

It is particularly easy to understand what we mean in the case of the free $bc\beta\gamma$ system.
For instance, the $\gamma$ fields live in the Dolbeault complex $\Omega^{0,*}(\CC ; V)$,
and their jets at the origin are $(V [[z,\zbar]] [\d \zbar] , \dbar)$,
where $\dbar$ is the formal Dolbeault differential. 
An example of an element is thus $\widehat{\gamma} = \sum_{m,n} \frac{1}{m!n!}g_{mn} z^m \zbar^n$,
which is just a formal power series with values in $V$.
An example of a functional is
\[
F(\widehat{\gamma}) = g_{10} + g_{21} = \left( \partial_z \widehat{\gamma}\right)|_0 + \left( \partial_z^2 \partial_{\zbar} \widehat{\gamma}\right)|_0,
\]
which corresponds to the local functional
\[
F(\gamma) = \int_\CC \partial_z \gamma + \partial_z^2 \partial_{\zbar} {\gamma}\,\d^2 z.
\]
We call the first kind of term a {\em chiral} interaction, as it only depends on holomorphic derivatives.

By the $\dbar$-Poincar\'{e} lemma, 
this complex $(V [[z,\zbar]] [\d \zbar] , \dbar)$ is quasi-isomorphic to $V[[z]]$, concentrated in degree zero. 
This observation is actually quite concrete:
it simply says that for a solution $\gamma$ to the equation of motion $\dbar \gamma = 0$, 
its Taylor expansion is just a power series in $z$ and it is independent of $\zbar$.
In consequence, if we consider translation-invariant Lagrangians depending only on the $\gamma$ field, then up to quasi-isomorphism these are $\Sym(V^\vee[z^\vee])$.
In other words, only chiral interactions yield distinct modifications of the action,
when one takes into account the equation of motion.

Note that we have chosen to work with functionals of the fields
that are polynomials built out of continuous linear functionals $V^\vee[z^\vee]$ of the jets.
This choice is the standard and natural one for variational problems.
We note as well that constant functionals are irrelevant,
so we want to use $\Sym^{>0}(V^\vee[z^\vee])$ to describe translation-invariant local functionals.

An analogous argument applies to the $c$ field. 
It shows there is a quasi-isomorphism of dg Lie algebras 
between the jet at the origin of the Dolbeault complex $\Omega^{0,*}(\CC ; T^{1,0}_\CC)$ of holomorphic vector fields 
and the Lie algebra of formal vector fields $\wone = \CC[[z]]\partial_z$.
The translation-invariant Lagrangians depending only on the $c$ field 
are thus quasi-isomorphic to $\cred^*(\wone)$,
by which we mean the (reduced) {\em continuous} Lie algebra cohomology,
often known as the Gelfand-Fuks cohomology 
Similar arguments work for the $\beta$ and $b$ fields.

If we take all the fields into account together and consider the full equations of motion 
for the holomorphic string,
which couple the $c$ field to the others,
then these arguments yield the following.

\begin{lem}\label{lem: gf1}
There is a quasi-isomorphism 
\[
\Def_{\rm string}(\CC,V)^\CC \simeq \cred^*(\wone, \Sym(V^\vee[z^\vee] \oplus V[z^\vee] \d z^\vee \oplus W_1^{\rm ad}[2])) [2]
\]
between the deformation complex of translation-invariant local functionals for the holomorphic string and a certain Gelfand-Fuks cochain complex.
\end{lem}

This lemma already substantially simplifies our lives, 
as one can invoke the literature on Gelfand-Fuks cohomology.
But before we do,
we will take advantage of another symmetry condition to simplify the situation.

\subsection{Dilating cotangent fibers}

We have already seen how to think of the holomorphic bosonic string theory 
as corresponding to the shifted cotangent bundle $\TT^*[-1]{\rm Maps}_{\dbar}(-, V)$, 
as a bundle over the moduli of Riemann surfaces. 
There is a natural action of the group $ \CC^\times$ on this space
by scaling the shifted cotangent fibers,
and we will use the notation $\CC^\times_{\rm cot}$ to indicate this appearance of the multiplicative group.

This group action can be seen on the level of the field theory as follows: 
we give the $\gamma$ and $c$ fields---the base of the cotangent bundle---weight $0$ and give the $\beta$ and $b$ fields---the cotangent fiber---weight~$1$. 
Note that, in consequence, the pairing $\langle-,-\rangle$ on fields thus has weight -1.
In these terms, the classical action functional is weight 1. 
Thus, we focus on weight 1 deformations of the action for the holomorphic bosonic string,
as we are interested in local functionals of the same kind.
That means we consider the subcomplex of weight 1 local functionals inside the deformation complex.

\begin{rmk}\label{rmk: classical weights}
Although this action $S$ has weight 1, 
its role in the cochain complex of classical observables is to define the differential $\{S,-\}$.
Observe that the shifted Poisson bracket $\{-,-\}$ has weight -1, 
because it is determined by the pairing, 
and so the differential has weight 0. 
\end{rmk}

This subcomplex admits a nice description in terms of the geometry of the target space.

\def\wt{{\rm wt}}

\begin{lem}
\label{lem: def complex wt zero} 
There is a $\GL(V)$-equivariant quasi-isomorphism
\[
\Def_{\rm string}(\CC)^{\CC, \wt(1)} \simeq \Sym(V^*) \tensor V[1]
\]
between the weight 1, translation-invariant deformation complex 
and the polynomial vector fields on $V$, placed in degree~-1.
\end{lem}

Concretely, this result says that there are every weight one interaction is trivialized by an automorphism of the theory.
This claim is a consequence of the fact that the zeroth cohomology group vanishes.
On the other hand, this lemma says the theory admits a large group of symmetries,
namely diffeomorphisms of the target, 
which appears as the degree -1 cohomology.

The $\GL(V)$ equivariance takes into account the natural symmetries of the target. 
It also is the first step in the approach to studying the deformation complex with general curved target. 
We will discuss this further in the section on string backgrounds. 

\subsection{Interaction terms that appear at one loop}

As we will see in Section~\ref{sec: quantization}, 
the quantization of the holomorphic string only involves local functionals of weight zero for this $\CC^\times_{\rm cot}$-action.
(Concretely, this restriction appears because the one-loop Feynman diagrams only have external legs for $c$ and $\gamma$ fields.)
Hence, it behooves us to compute the weight zero subcomplex of the deformation complex as well.

\begin{lem}
\label{lem: def complex wt zero} 
There is a $\GL(V)$-equivariant quasi-isomorphism
\[
\Def_{\rm string}(\CC)^{\CC, \wt(0)} \simeq \CC[-1] \oplus \Omega^2_{cl}(V)[1] \oplus \Omega^1(V) \oplus \Omega^1_{cl}(V)[-1] 
\]
between the weight 0, translation-invariant deformation complex 
and natural complexes related to the geometry of the target.
\end{lem}

Before explaining the key steps of the proof, 
we remark that there is another, more structural way to see that only weight zero local functionals should be relevant.
A quick physical argument would say that we want the path integral measure $\exp(-S/\hbar)$ to be weight zero,
which forces $\hbar$ to have weight one to cancel out with the weight of the classical action.
But the one-loop term $I_1$ in the quantized action $S^\q = S + \hbar I_1 + \cdots$ must then have weight zero.

There is a BV analogue of this argument.
It notes that the differential of the quantum observables has the form $\{S^\q,-\} + \hbar \Delta$,
where $\Delta$ denotes the BV Laplacian.
(See Section \ref{subsec: QME} for a discussion of these objects.)
As the BV Laplacian has weight -1 because it is determined by the bracket,
we must give $\hbar$ weight 1 to ensure the total differential has weight zero.
Again the one-loop interaction is forced to have weight zero.

\subsubsection{Sketch of proof}

We have already seen in Lemma \ref{lem: gf1} that we can identify the full translation invariant deformation complex with a certain Gelfand-Fuks cohomology. 
In terms of this Gelfand-Fuks cohomology we see that the cotangent weight zero piece is identified with the subcomplex
\ben
\Def_{\rm string}(\CC)^{\CC, \wt(0)} = \cred^*\left(\wone ; \Sym(V^\vee[z^\vee]) \right) [2]. 
\een 
We will drop the overall shift by $2$ until the end of the proof. 

Any symmetric algebra has a natural maximal ideal: 
for any vector space $W$,
\[
\Sym(W) = \CC \oplus \Sym^{\geq 1}(W).
\] 
Thus, we can decompose our complexes as
\ben
 \cred^*\left(\wone ; \Sym(V^\vee[z^\vee]) \right) =   \cred^*(\wone) \oplus \clie^*\left(\wone ; \Sym^{\geq 1} (V^\vee[z^\vee]) \right) .
 \een
The first summand is the reduced Gelfand-Fuks cohomology of formal vector fields with values in the trivial module.
It is well-known, see \cite{Fuks}, that $H^3_{\rm red} (\wone) \cong \CC[-3]$, 
i.e., this cohomology is one-dimensional and concentrated in degree~$3$. 

We now proceed to computing the second summand. 
Denote by $\{L_n = z^{n+1} \partial_z\}$ the standard basis for the Lie algebra of formal vector fields $\wone$. 
Notice that the Euler vector field $L_0 = z \partial_z$ induces a grading on $\wone$,
that we will call {\em conformal dimension}.
Note that $L_n$ has conformal dimension $n$. 
This grading extends naturally to the Chevalley-Eilenberg complex of $\wone$ with coefficients in any module. 

Let $\lambda_n \in \wone^\vee$ be the dual vector to $L_n$. 
(We work with the continuous dual vector space, as in the setting of Gelfand-Fuks cohomology.) 
An arbitrary element of $V [[z]]$ is a linear combination of vectors of the form $v \tensor z^k$. 
Write $\zeta_k$ for the dual element $(z^k)^\vee$. 
Thus an element of $(V [[z]])^\vee$ is a linear combination of the vectors of the form $v^\vee \tensor \zeta_k$. 

The following general fact is useful.

\begin{lem} \label{lem: gf}
Let $M$ be any $\wone$-module. 
The inclusion of the subcomplex of conformal dimension zero elements
\ben
\clie^*(\wone ; M)^{\wt(0)} \xto{\simeq} \clie^*(\wone ; M)
\een
is a quasi-isomorphism.
\end{lem}

\owen{Eugene asks if the superscript means weight or conformal dimension.}
\brian{Fudge, we seem to have convoluted the notations for weight and conformal dimension. In this lemma we really mean conformal dimension. I suggest we reserve the superscript for the conformal dimension.}
\owen{Sounds good. Can you fix this notation systematically, as I think you have the deformation stuff clearer in your head and hence are less likely to introduce new mistakes!}

\begin{proof} 
For each $p-1 \geq 0$, 
define the operator $\iota_{L_0} : \clie^{p}(\wone ; M) \to \clie^{p-1}(\wone ; M)$ by sending a cochain $\varphi$ to the cochain
\ben
(\iota_{L_0}\varphi)(X_1,\ldots,X_p) = \varphi(L_0, X_1,\ldots,X_p) .
\een 
Let $\d$ be the differential for the complex $\clie^*(\wone ; M)$. 
It is easy to check that the difference $\d \iota_{L_0} + \iota_{L_0} \d$ is equal to the \owen{identity minus?} projection onto the dimension zero subspace. 
\end{proof}

Returning now to our case of interest, we see that the underlying graded vector space of this conformal dimension zero subcomplex splits as follows:
\be\label{splitting}
\clie^{\#}(\wone)^{\wt(0)} \tensor \left(\Sym^{\geq 1}\left(V [[z]]\right)^\vee \right)^{\wt(0)} \oplus \clie^{\#}(\wone)^{\wt(1)} \tensor \left(\Sym^{\geq 1}\left(V [[z]]\right)^\vee\right)^{\wt(-1)}
\ee
In the first component, the purely dimension zero part of the reduced symmetric algebra is simply $\Sym^{\geq 1}(V^\vee)$, i.e., power series on $V$ with no constant term.
We denote this algebra concisely as $\cO_{red}(V)$, for reduced functions on $V$.
Similarly, in the second component, 
the dimension one part of $\Sym^{\geq 1}\left(V[[z]]\right)^\vee$ is of the form ${\rm Sym}(V^\vee) \tensor z^\vee V^\vee$, which is naturally identified with the polynomial 1-forms~$\Omega^1_{\rm alg}(V)$. 

The differential in this Gelfand-Fuks complex has the form
\[
\xymatrix{
\overset{\ul{0}}{1 \otimes \sO_{red}(V)} \ar[rd]^{\d_{dR}} & \overset{\ul{1}}{\lambda^0 \otimes \sO_{red}(V)} \ar[r] \ar[rd]^{\d_{dR}} & \overset{\ul{2}}{\lambda^{-1} \wedge \lambda^1 \otimes \sO_{red}(V)} & \overset{\ul{3}}{\lambda^{-1} \wedge \lambda^1 \wedge \lambda^0 \otimes \sO_{red}(V)} \\
 & \lambda^{-1} \otimes \Omega^1(V) \ar[r] & \lambda^{-1} \wedge \lambda^0 \otimes\Omega^1(V) &
}
\]
The top line comes from the first summand in (\ref{splitting}) and the bottom line corresponds to the second summand.
The top horizontal map sends $\lambda^0$ to $2 \cdot \lambda^{-1} \wedge \lambda^1$, 
and the bottom horizontal map sends $\lambda^{-1}$ to $\lambda^{-1} \wedge \lambda^0$ (both are the identity on $V$). 
The diagonal maps are given by the de Rham differential $\d_{dR} : \sO_{red}(V) \to \Omega^1(V)$. 
This complex is quasi-isomorphic to 
\[
\xymatrix{
1 \otimes \sO_{red}(V) \ar[rd]^{\d_{dR}} & & & \lambda^{-1} \wedge \lambda^1 \wedge \lambda^0 \otimes \sO_{red}(V) \\
 & \lambda^{-1} \otimes \Omega^1(V) & \lambda^{-1} \wedge \lambda^0 \otimes\Omega^1(V) &
}
\]
which, in turn, is identified with $\Omega^{2}_{cl}(V)[-1] \oplus \Omega^1(V)[-2] \oplus \Omega^1_{cl}(V)[-3]$. 
After accounting for the overall shift by $2$, 
we arrive at the identification of the $\CC_{\rm cot}^\times$-weight zero component of the translation-invariant deformation complex.

\subsection{Interpretation as string backgrounds}

We now discuss, in light of the calculations above, how to interpret string backgrounds in our approach. 
Since $V$ is flat,
we will see that the following deformations will be trivializable. 
These trivializations are {\em not}, however, equivariant for the obvious $\GL(V)$ action.
(More generally, for non-flat targets, they are not equivariant for general diffeomorphisms of the target.) 
Thus, these deformations are relevant for the case of a curved target, and we can give an interpretation of them in terms of the usual perspective of {\em string backgrounds}. 

As discussed above, we should think of the $\CC^\times_{\rm cot}$ weight $1$ local functionals as deformations of the classical theory as a cotangent theory;
we simply add such terms to the action functional.
These weight one deformations of cohomological degree zero are parametrized by $H^1(V ; T_V)$. 
We extract an explicit local functional from an element $\mu \in H^1(V ; T_V)$
by the formula
\ben
\int_\Sigma \<\beta, \mu(\gamma)\>_V .
\een 
On the other hand, this element $\mu$ also parametrizes a deformation of the complex structure of $V$.
Thus, we see how a deformation of complex structure is encoded in a modification of the action. 
We interpret this phenomenon as an appearance of the ordinary curved background in bosonic string theory from the perspective of the holomorphic model we are studying here.

There are interesting deformations that go outside of the world of cotangent theories. 
Consider the cohomological degree zero part of the weight 0 complex. 
There is a term of the form $H^1(V ; \Omega^2_{cl}(V))$.
It is shown in Part 2, Section 8.5 of \cite{GGW}, 
how closed holomorphic two-forms determine local functionals of the $\beta\gamma$ system with curved target. 
A sketch of this construction goes as follows.
Locally we can write a closed holomorphic 2-form as $\d \theta$ for some holomorphic one-form $\theta \in \Omega^1(V)$. 
If $\gamma : \Sigma \to V$ is a map of the $\sigma$-model, 
there is an induced map (when $\gamma$ satisfies the equations of motion) $\gamma^* : \Omega^1(V) \to \Omega^1(\Sigma)$. 
We can then integrate $\gamma^* \theta$ along any closed cycle $C$ in $\Sigma$, 
and one should think of this function as a residue along~$C$. 
In \cite{GGW} we write down a local functional that realizes this residue, 
and one can show that it only depends on the corresponding class in $H^1(V ; \Omega^2_{cl}(V))$. 
We posit that this example corresponds to the $B$-field deformation of the ordinary bosonic string. 

In future work we aim to study how our description of holomorphic string backgrounds compares to the approaches of string backgrounds in the physics literature. 
See, for instance, \cite{CFMP} for an overview.



\section{Quantizing the holomorphic bosonic string on a disk} 
\label{sec: quantization}

For us, quantization will mean that we use perturbative constructions in the setting of the BV formalism.
Concretely, this means that we enforces the gauge symmetries using the homological algebra of the BV formalism and that we use Feynman diagrams and renormalization to obtain an approximation for the desired, putative path integral. 
%\owen{Be more careful about saying path integral. It's an approximation.}
There are toy models for this approach where one can see very clearly how it gives asymptotic expansions for finite-dimensional integrals \cite{GJF}.
In particular, these toy models show that this approach need not recover the true integral
but does know important information about it;
a similar relationship should hold between this quantization method and the putative path integral, 
but in this case there is no {\em a priori} definition of the true integral in most cases.

This notion of quantization applies to any field theory arising from an action functional,
and the algorithm one applies to obtain a quantization is the following:
\begin{enumerate}
\item Write down the integrals labeled by Feynman diagrams arising from action functional.
\item Identify the divergences that appear in these integrals and add ``counterterms'' to the original action that are designed to cancel divergences.
\item Repeat these steps until no more divergences appear in Feynman diagrams.
We call this the ``renormalized action.''
\item Check if the renormalized action satisfies the quantum master equation. 
If it does, you have a well-posed BV quantum theory, and we call the result a {\em quantized action}. If not, guess a way to adjust the renormalized action and begin the whole process again.
\end{enumerate}
It should be clear that along the way, one makes many choices;
hence if a quantization exists, it may not be unique.
It is also possible that a BV quantization may not exist.

In this section we will apply the algorithm in the case of $\Sigma~=~\CC$.
For this theory we are lucky, however:
at one-loop the integrals that appear in our quantization from the Feynman diagrams do not have divergences,
so that renormalized action is easy to compute.
This aspect is the subject of the first part of this section.
(In Section \ref{sec: conformalblock} we will provide an argument based on deformation theory as to why quantizations exist on arbitrary Riemann surfaces.)
Moreover, it is easy to check whether the quantum master equation is satisfied,
and the answer is simple.
This aspect is the subject of the second part.
The results can be summarized as follows.

\begin{prop}
The holomorphic bosonic string with source $\CC$ and target $\CC^d$ admits a BV quantization
if $d = 13$.
This quantized action only has terms of order $\hbar^0$ and $\hbar$ (i.e., it quantizes at one loop).
\end{prop}

\subsection{The Feynman diagrams}

Let us describe the combinatorics of the Feynman diagrams that appear here
before we describe the associated integrals.

\subsubsection{}

The procedure constructs graphs out of a prescribed type of vertices and edges;
we must consider all graphs with such local structure.
The classical action functional determines the allowed kinds of vertices and edges.
The quadratic terms of the action tell us the edges;
each quadratic term yields an edge whose boundary is labeled by the two fields appearing in the term.
For us there are thus two types of edges: 
an edge that flows from $\beta$ to $\gamma$, 
and an edge that flows from $b$ to~$c$ displayed in Figure \ref{fig:props}.
\begin{figure}
\begin{tikzpicture}
		\draw[fermion] (-4,0) -- (-2,0);
		\draw[vector] (2,0) -- (4,0);
		\draw (-4.2, 0) node {$\beta$};
		\draw (-1.8,0) node {$\gamma$};
		\draw (1.8,0) node {$b$};
		\draw (4.2,0) node {$c$};
\end{tikzpicture}
\caption{The $\beta\gamma$ and $bc$ propagators}
\label{fig:props}
\end{figure}

The nonquadratic terms tell us the vertices:
each $n$-ary term yields a vertex with $n$ legs,
and the legs are labeled by the $n$ types of fields appearing in the term.
For us there are thus two types of trivalent vertices:
a vertex with two $c$ legs and a $b$ leg, 
and a vertex with a $c$ leg, a $\gamma$ leg, and a $\beta$ leg.
It helpful to picture these legs as directed,
so that $c$ and $\gamma$ legs flow into a vertex
and $b$ and $\beta$ legs flow out. 
These vertices are displayed in Figure \ref{fig:verts}.

\begin{figure}
\begin{tikzpicture}[line width=.2mm, scale = 1]
		\draw[vector](-4.7,1)--(-4,0);
		\draw[fermion](-3.3,1)--(-4,0);
		\draw[fermion](-4,0)--(-4,-1);
		\filldraw[color=black]  (-4,0) circle (.1);
		\draw (-4.75,1.2) node {$c$};
		\draw (-3.25,1.2) node {$\gamma$};
		\draw (-4,-1.3) node {$\beta$};	
		
		\draw[vector](3.3,1)--(4,0);
		\draw[vector](4.7,1)--(4,0);
		\draw[vector](4,0)--(4,-1);
		\filldraw[color=black]  (4,0) circle (.1);
		\draw (3.25, 1.2) node {$c$};
		\draw (4.75,1.2) node {$c$};
		\draw (4, -1.3) node {$b$};
\end{tikzpicture}
\caption{The trivalent vertices for $\int \langle \beta, [c,\gamma] \rangle$ and $\int \langle b, [c,c] \rangle$}
\label{fig:verts}
\end{figure}

The kinds of graphs one can build with such vertices and edges are limited.
We focus on connected graphs, since an arbitrary graph is just a union of connected components.

A tree (i.e., a connected graph with no loops) must have at most one outgoing leg,
which must be either a $b$ or a~$\beta$;
the other legs are incoming, so each must be labeled by a $c$ or a~$\gamma$. 
An example of such a tree is given in Figure \ref{fig:tree}.

\begin{figure}
\begin{tikzpicture}[line width=.2mm, scale = 1]
		\draw[vector] (-2,3) -- (-1,2.5);
		\draw[vector] (-2, 2) -- (-1, 2.5);
		\draw[vector] (-1,2.5) --(0, 1.5);
		\draw (-2.25, 3) node {$c$};
		\draw (-2.25, 2) node {$c$};
		\draw(-.4, 2.25) node {$P$};
		\filldraw[color=black] (-1,2.5) circle (.075);
		\draw[vector] (-2, 1) -- (-1, .5);
		\draw[fermion] (-2, 0) -- (-1,.5);
		\draw[fermion] (-1,.5) -- (0,1.5);
		\draw (-2.25, 1) node {$c$};
		\draw (-2.25, 0) node {$\gamma$};
		\draw (-.4, .75) node {$P$};
		\filldraw[color=black] (-1,.5) circle (.075);
		\draw[fermion] (0,1.5) -- (1,1.5);
		\draw (1.25, 1.5) node {$\beta$};
		\filldraw[color=black] (0,1.5) circle (.075);
\end{tikzpicture}
\caption{An example of a tree with four inputs and one output}
\label{fig:tree}
\end{figure}
		
Note that there are two types of trees.
If there is a $\gamma$ leg, then there is a $\beta$ leg,
and there is a chain of $\gamma\beta$ edges connecting them;
all other external legs are of $c$~type.
If there is a $b$ leg, then the only other legs are $c$~type.

A one-loop graph will consist of a wheel (i.e., a sequence of edges that form an overall loop)
with trees attached.
The outer legs are all of $c$~type.
Every edge along a wheel will have the same type.
It is not possible to build a connected graph with more than one loop.
This combinatorics is the essential reason that we can quantize at one loop.
For an example of such a wheel see Figure \ref{fig:wheel}.

\begin{figure}
\begin{tikzpicture}[line width=.2mm, scale=1.5]
		\draw[vector](145:1) -- (145:.3cm);
			\node at (145:1.15) {$c$};
		\draw[vector](215:1) -- (215:.3cm);
			\node at (215:1.2) {$c$};
		\draw[vector](35:1) -- (35:.3cm);
			\node at (35:1.15) {$c$};
		\draw[vector](-35:1) -- (-35:.3cm);
			\node at (-35:1.2) {$c$};
		\draw[fill=black] (0,0) circle (.3cm);
		\draw[fill=white] (0,0) circle (.29cm);
	    	\clip (0,0) circle (.3cm);
\end{tikzpicture}
\caption{An example of a wheel with four inputs}
\label{fig:wheel}
\end{figure}

We write ${\bf Graph}_{\rm string}$ for the collection of connected graphs just described,
namely the directed trees and 1-loop graphs allowed by the string action functional.
Let ${\bf Graph}_{\rm string}^{(0)}$ denote the 0-loop graphs (i.e., trees) and let ${\bf Graph}_{\rm string}^{(1)}$ denote the 1-loop graphs (i.e., wheels with trees attached).


%\begin{figure}
%\begin{tikzpicture}[decoration={markings,mark=at position 0.6cm with {\arrow[black,line width=.4mm]{stealth}}}];
%\draw[thick, postaction=decorate, decoration={markings,mark=at position 0.6cm with {\arrow[black,line width=.4mm]{stealth}}},
%decoration={snake, segment length=5, amplitude=1}](-.7,1) --(0,0);
%%\draw[postaction=decorate, line width=.2mm] (-0.7, 1) -- (0,0);
%\draw[postaction=decorate, line width=.2mm] (0.7,1) -- (0,0);
%\draw[postaction=decorate, line width=.2mm] (0,0) -- (0,-1);
%\filldraw[color=black]  (0,0) circle (.1);
%\draw (1.2,-0.1) node {$I_k = t^n \partial_t$};
%\draw (-0.75,1.2) node {$c$};
%\draw (0.75,1.2) node {$\gamma$};
%\draw (0,-1.3) node {$\beta$};
%\end{tikzpicture}
%\caption{The trivalent vertex for $\int \langle \beta, c \gamma\rangle$}
%\label{fig:vertex}
%\end{figure}

%\begin{figure}
%\begin{tikzpicture}[decoration={markings,
%   mark=at position 1.2cm with {\arrow[black,line width=.4mm]{stealth}}}];
%\draw[postaction=decorate, line width=.2mm] (-4,0) -- (-2,0);
%\draw (-4.2,0) node {$\gamma$};
%\draw (-1.8,0) node {$\beta$};
%\draw[->,decorate,decoration=snake] (2,0) -- (4,0);
%\draw (1.8,0) node {$c$};
%\draw (4.2,0) node {$b$};
%\draw (0, 0.4) node {$P$};
%\end{tikzpicture}
%\caption{The propagators as directed from $\gamma$ to $\beta$}
%\label{fig:prop}
%\end{figure}

%\begin{figure}
%\begin{center}
%\feynmandiagram [horizontal=a to b]{
%a [particle=\(\gamma\)] -- [charged boson] b [particle=\(\beta\)]
%};
%\end{center}
%\end{figure}
%
%\feynmandiagram[horizontal=a to b]{
%a -- [charged boson] b,
%i1 -- [fermion] a -- [fermion] i2,
%f1 -- [fermion] b -- [fermion] f2,
%};

%\begin{figure}
%\begin{center}
%\begin{tikzpicture}[scale=0.8,decoration={markings,mark=at position 0.9cm with {\arrow[black,line width=.4mm]{stealth}}}];
%\draw[postaction=decorate, line width=.2mm] (2,0) -- (0,0);
%\draw (1, -0.4) node {$P$};
%\draw[postaction=decorate, line width=.2mm] (0,0) -- (0,2);
%\draw (-0.4, 1) node {$P$};
%\draw[postaction=decorate, line width=.2mm] (0,2) -- (2,2);
%\draw (1, 2.4) node {$P$};
%\draw[postaction=decorate, line width=.2mm] (2,2) -- (2,0);
%\draw (2.4, 1) node {$P$};
%\draw[postaction=decorate, line width=.2mm] (-1.4,0) -- (0,0);
%\draw (-1.6,0) node {$\gamma$};
%\draw[postaction=decorate, line width=.2mm] (0,-1.4) -- (0,0);
%\draw (0,-1.6) node {$\gamma$};
%\draw[postaction=decorate, line width=.2mm] (-1.12,-1.12) -- (0,0);
%\draw (-1.3,-1.3) node {$\gamma$};
%\draw[postaction=decorate, line width=.2mm] (-1.12,3.12) -- (0,2);
%\draw (-1.3,3.3) node {$\gamma$};
%\draw[postaction=decorate, line width=.2mm] (3.4,2.5) -- (2,2);
%\draw (3.6,2.6) node {$\gamma$};
%\draw[postaction=decorate, line width=.2mm] (2.5,3.4) -- (2,2);
%\draw (2.6,3.6) node {$\gamma$};
%\draw[postaction=decorate, line width=.2mm] (3.4,-0.5) -- (2,0);
%\draw (3.6,-.6) node {$\gamma$};
%\draw[postaction=decorate, line width=.2mm] (2.5,-1.4) -- (2,0);
%\draw (2.6,-1.6) node {$\gamma$};
%\filldraw[color=black]  (0,0) circle (.1);
%\filldraw[color=black]  (0,2) circle (.1);
%\filldraw[color=black]  (2,0) circle (.1);
%\filldraw[color=black]  (2,2) circle (.1);
%\end{tikzpicture}
%\caption{A wheel with four vertices}
%\label{fig:wheel}
%\end{center}
%\end{figure}

\subsubsection{}

These graphs describe linear maps associated to the field.
More precisely, a graph with $k$ legs describes a linear functional on the $k$-fold tensor product of the space of fields.
One builds this linear functional out of the data of the action functional.

As an example, a $k$-valent vertex corresponds to a $k$-ary term in the action,
which manifestly takes in $k$ copies of the fields and outputs a number.
Thus, the vertex labels an element of a (continuous) linear dual of the $k$-fold tensor product of fields.
In fact, one restricts to {\em compactly-supported} fields,
since the action functional is rarely well-defined on all fields when the source manifold is non-compact.
(Note this domain of compactly-supported fields is all one needs for making variational arguments or for constructing a BV quantization.)

An edge corresponds an element $P$ of the 2-fold tensor product of the space of fields,
often called a {\em propagator}.
More precisely, the edge should correspond to
the Green's function for the linear differential operator 
appearing in the associated quadratic term of the action;
hence the propagator is an element of the {\em distributional completion} of the 2-fold tensor product.
For us the $\beta\gamma$ leg should be labeled by $\dbar^{-1} \otimes {\rm id}_V$,
where $\dbar^{-1}$ denotes an inverse to the Dolbeault operator on functions.
The $bc$ leg should be labeled by $\dbar^{-1}_T$, 
the inverse of the Dolbeault operator on the bundle~$T^{1,0}$.

Given a graph~$\Gamma$, one should contract the tensors associated to the vertices and edges.
We denote the linear functional for this graph by~$w_\Gamma(P,I)$,
where $w$ stands for ``weight,'' the term $P$ indicates we label edges by the propagator~$P$,
and the term $I$ indicates we label vertices by the ``interaction'' term of the action~$S$ 
(i.e., the terms that are cubic and higher).

This contraction is not always well-posed, unfortunately.
Each vertex labels a distributional section of some vector bundle on~$\Sigma$,
and each edge labels a distributional section of a vector bundle on~$\Sigma^2$.
Thus the desired contraction can be written {\em formally} as an integral over the product manifold~$\Sigma^{v}$,
where $v$ denotes the number of vertices.
In most situations this contraction is ill-defined, 
since one cannot (usually) pair distributions.
Concretely, one sees that the integral expression is divergent.

Thus, to avoid these divergences, one labels the edges by a smooth replacement of the Green's functions. 
(Imagine replacing a delta function $\delta_0$ by a bump function.)
Since one can pair smooth functions and distributions,
each graph yields a linear functional on fields using these mollified edges.
Thus we have {\em regularized} the divergent expression.

But now this linear functional depends on the choice of mollifications.
Hence the challenge is to show that 
if one picks a sequence of smooth replacements that approaches the Green's function,
there is a well-defined limit of the linear functionals.

\subsubsection{}

We will now sketch one method well-suited to complex geometry
that allows us to see that no divergences appear for the holomorphic bosonic string.
Our approach is an example of the renormalization method developed by Costello in ~\cite{CosBook},
which applies to many more situations.

Our primary setting in this section is $\Sigma=\CC$.
For this Riemann surface, 
a standard choice of Green's function for the $\dbar$ that acts on functions is
\[
P(z,w) = \frac{1}{2 \pi i} \frac{\d z + \d w}{z-w}.
\]
It is a distributional one-form on $\CC^2$ that satisfies $\dbar \otimes 1(P) = \delta_\Delta$, 
where $\delta_\Delta$ is the delta-current supported along the diagonal $\Delta: \CC \hookrightarrow \CC^2$ and providing the integral kernel for the identity.
In terms of our discussion above,
we view this one-form as a distributional section of the fields $\gamma$ and~$\beta$: 
for example, for fixed $w$, the one-form $\d z/(z - w)$ is a $\beta$ field in the $z$-variable 
as it is a $(1,0)$-form.
(This propagator is for the $\beta\gamma$ fields---and one must tensor with a kernel for the identity on $V$---but a similar formula provides a propagator for the $bc$ fields.)

\subsubsection{}

We will now describe the integral associated to a simple diagram.
For simplicity, we assume $V = \CC$ so that the $\gamma$ and $\beta$ fields are simply functions and $1$-forms on $\CC$, respectively.
Consider a ``tadpole'' diagram, Figure \ref{fig:tadpole}, $\Gamma_{\rm tad}$ whose outer legs are $c$~fields 
(i.e., vector fields on~$\CC$).

\begin{figure}
\begin{tikzpicture}[line width=.2mm, scale=1.5]
		\draw[vector](180:1) -- (180:.3cm);
			\node at (180:1.15) {$c$};
		\draw[fill=black] (0,0) circle (.3cm);
		\draw[fill=white] (0,0) circle (.29cm);
	    	\clip (0,0) circle (.3cm);
\end{tikzpicture}
\caption{The tadpole diagram $\Gamma_{\rm tad}$}
\label{fig:tadpole}
\end{figure}


There is only one vertex here, corresponding to the cubic function on fields
\[
w_{\Gamma_{\rm tad}}(P,I_{\rm string}) = \int_{z \in \CC} \beta \wedge c\gamma.
\]
If the field $c$ is of the form $f(z) \d \zbar \partial_z$,
with $f$ compactly supported, 
then our integral is
\[
\int_{z \in \CC} \beta \wedge f(z)(\partial_z\gamma) \d \zbar.
\]
(Note that a general cubic function could be described as an integral over $\CC^3$,
but our function is supported on the small diagonal $\CC \hookrightarrow \CC^3$.)
The linear functional for this tapole diagram should be given by inserting the propagator $P$ in place of the $\beta$ and $\gamma$ fields. 
Hence it ought to be given by the following integral over~$\CC$:
\[
\int_{z \in \CC} c(z)P(z,w)|_{z = w}  
= \int_{z \in \CC} f(z) \partial_z \left(  \frac{1}{2 \pi i} \frac{\d z + \d w}{z-w}\right)|_{z = w}\, \d \zbar.
\]
This putative integral is manifestly ill-defined,
since the distribution is singular along the diagonal.

\subsubsection{}

We smooth out the propagator $P$ using familiar tools from differential geometry.
Fix a Hermitian metric on $\Sigma$, 
which then associates provides an adjoint $\dbar^*$ to the Dolbeault operator~$\dbar$.
For the usual metric on $\CC$, we have
\[
\dbar^* = -2 \frac{\partial}{\partial (\d \zbar)} \frac{\partial}{\partial z}.
\]
In physics one calls a choice of the operator $\dbar^*$ a {\em gauge-fix}.
The commutator $[\dbar,\dbar^*]$, which we will denote $D$, 
is equal to $\tfrac{1}{2} \Delta$, where $\Delta$ is the Laplace-Beltrami operator for this metric.
In the physics literature, explicit gauge fixes for the bosonic string can be found in \cite{Bochicchio}.

%We can thus call upon Hodge theory and many nice results about finding partial inverses to the Laplacian.

We introduce a smoothed version of the propagator using the heat kernel~$e^{-tD}$,
which is a notation that denotes a solution to the heat equation $\partial_t f(t,z) + D f(t,z) = 0$.
For $\CC$ with the Euclidean metric, the standard heat kernel is
\[
e^{-tD}(z,w) =  \frac{1}{4\pi t} e^{-|z-w|^2/4t} (\d z - \d w) \wedge (\d\zbar - \d\wbar) . 
\]
For $0 < \ell < L < \infty$, we define
\[
P_\ell^L = \dbar^* \int_{\ell}^L e^{-tD}\d t.
\]
We compute
\[
\dbar P_\ell^L = D \int_{\ell}^L e^{-tD}\d t =  \int_{\ell}^L \frac{d}{dt} e^{-tD}\d t = e^{-LD} - e^{-\ell D}.
\]
In the limit as $\ell \to 0$ and $L \to \infty$, the operator $P_\ell^L$ goes to a propagator (or Green's function) $P$ for~$\dbar$.
To see this, consider an eigenfunction $f$ of $D$ where $Df=\lambda f$ where $\lambda$ is a non-negative real number. 
Then
\[
(\dbar P_\ell^L) f = (e^{-L\lambda} - e^{-\ell \lambda})f, 
\]
which goes to $f$ as $L \to \infty$ and $\ell~\to~0$.
Thus, if one works with the correct space of functions, 
$P_\ell^L$ is almost an inverse to $\dbar$;
moreover, it is a smooth function on $\Sigma~\times~\Sigma$. 

%\brian{I've fixed our convention for eigenvalues and some small typos above. Is there something you want to be careful above?}

\subsubsection{}

We now return to the tadpole diagram and put $P_\ell^L$ on the edge instead of~$P$.
(We again assume $V = \CC$ for simplicity.)
The propagator is
\begin{align}\label{propagator}
P_\ell^L(z,w) &= \int_{\ell}^L \d t \, \frac{\partial}{\partial (\d \zbar)} \frac{\partial}{\partial z}\left( \frac{1}{4\pi t} e^{-|z-w|^2/4t} (\d z - \d w) \wedge (\d\zbar - \d\overline{w})\right)\\
&= \int_{\ell}^L \d t \frac{1}{4\pi t} \frac{\zbar - \overline{w}}{2t} e^{-|z-w|^2/4t} (\d z - \d w).
\end{align}
Note that it is smooth everywhere on~$\CC^2$.
The integral for the tadpole diagram is 
\begin{align*}
w_{\Gamma_{\rm tad}}(P_\ell^L,I_{\rm string})
&= \int_{z \in \CC} c(z)P_\ell^L(z,w)|_{z = w}  \\
&= \int_{z \in \CC} \int_{\ell}^L \d t f(z) \partial_z \left(\frac{1}{4\pi t} \frac{\zbar - \overline{w}}{2t} e^{-|z-w|^2/4t} (\d z - \d w) \right)|_{z = w}\, \d \zbar\\
&= \int_{z \in \CC} \int_{\ell}^L \d t f(z) \left(\frac{1}{4\pi t} \left(\frac{\zbar - \overline{w}}{2t}\right)^2 e^{-|z-w|^2/4t} (\d z - \d w) \right)|_{z = w}\, \d \zbar\\
&= 0,
\end{align*}
since the integrand vanishes along the diagonal.
Note that this integral is independent of $\ell$ and $L$ and hence the limit is zero.

\subsubsection{}

By explicitly analyzing the $\ell \to 0$ limit for the integral associated to every Feynman diagram,
we find the following result.

\begin{prop}
For any graph $\Gamma \in {\bf Graph}_{\rm string}$ allowed by the combinatorics of the string action functional and for any $L > 0$,
there is a well-defined limit $\lim_{\ell \to 0} w_{\Gamma}(P_{\ell}^L,I_{\rm string})$.
\end{prop}

We denote this limit by~$w_{\Gamma}(P_{0}^L,I_{\rm string})$. 
The necessary manipulations and inequalities referenced below are very close to those used in~\cite{wg2, GGW}.
%We recommend looking at \owen{exact location} for model arguments. 

\begin{proof}[Outline of proof]
When $\Gamma$ is a tree, there is never an issue with divergences; 
we could even use the Green's function $\dbar^{-1}$ on each edge.
To see this, note that one can view a tree as having a distinguished root,
given by the leg that is either of $\beta$ or $b$~type.
One can then see the tree as describing a multilinear map from the leaves (i.e., legs that are not roots) to the root.
Indeed, one can view each cubic vertex as such an operator.
For instance, $\langle b, [c,c]\rangle$ corresponds to the Lie bracket of vector fields,
since we view $\langle b,-\rangle$ as an element of the $c$~fields.
For a tree, one can then input arbitrary elements into the leaves, 
apply the operations labeled by the vertices,
apply the operator labeled by the edge, and so on,
until one reaches the root.
The composite multilinear operator sends smooth sections to smooth sections,
even if the edges are labeled by distributional sections,
since the associated operator sends smooth sections to smooth sections.

When $\Gamma$ is a one-loop graph, it consists of a wheel with trees attached to the outer legs.
By the preceding argument, we know those trees do not introduce singularities;
hence any divergences are due solely to the wheel.
It thus suffices to consider pure wheels (i.e., those with no trees attached).

Let the wheel have $n$ vertices. 
The $k$th vertex has a coordinate $z_k$ on $\CC$;
the $k$th external leg has input $c_k = f_k(z_k,\zbar_k) \d\zbar_k\, \partial_{z_k}$, 
where $f_k$ is a compactly-supported smooth function.
Then the integral has the form
\[
\int_{(z_1,\ldots,z_n) \in \CC^n}\d^n \zbar \,(f_1  \partial_{z_1} P_\ell^L(z_1,z_n))(f_2 \partial_{z_2} P_\ell^L(z_2,z_1)) \cdots (f_n\partial_{z_n} P_\ell^L(z_n,z_{n-1})),
\] 
since the $k$th input will act on one of the propagators entering the $k$th vertex.
One needs to show that this expression has a finite $\ell \to 0$ limit.

Let us prove this limit exists for the case $n=2$.
Then we have
\begin{align*}
\int_{z_1,z_2 \in \CC} \d\zbar_1\d\zbar_2 \int_{\ell}^L \d t_1 \int_{\ell}^L \d t_2\, 
& f_1(z_1)f_2(z_2) 
\partial_{z_1} \left(\frac{1}{4\pi t_1} \frac{\zbar_1 - \zbar_2}{2t_1} e^{-|z_1-z_2|^2/4t_1} (\d z_1 - \d z_2) \right)\\
& \times \partial_{z_2} \left(\frac{1}{4\pi t_2} \frac{\zbar_1 - \zbar_2}{2t_2} e^{-|z_1-z_2|^2/4t_2} (\d z_2 - \d z_1) \right),
\end{align*}
which is already a bit lengthy.
As our focus is on showing a limit exists, we will throw out unimportant factors and simplify the expression.
First, note that taking the partial derivative $\partial_{z_i}$ will simply multiply the integrand by $(\zbar_1 - \zbar_2)/2t_i$.
Moreover, we change coordinates to $u = z_1 - z_2$ and $v = z_2$. 
Then the integral is proportional to
\[
\int_{\ell}^L \d t_1 \int_{\ell}^L \d t_2\int_{\CC^2} \d^2 u \, \d^2 v \, f_1 f_2 \frac{\overline{u}^4}{t_1^3 t_2^3} e^{-|u|^2(\tfrac{1}{t_1} + \tfrac{1}{t_2})}.
\]
We take the integral over $v$ last;
it will be manifestly well-behaved after we take the other integrals.

Thus consider the integral just over $u \in \CC$,
so that we are computing the expected value of $F=f_1 f_2$ against a Gaussian measure 
whose variance is determined by $t_1$ and~$t_2$.
(Namely, the variance is~$t_1 t_2/(t_1+t_2)$.)
We might as well focus on values of $t_i$ that are very small, 
as those would be the source of divergences when $\ell \to 0$.
For small $t_i$, we only care about the behavior of $F$ near the origin as the measure is concentrated near the origin.
Thus, consider a partial Taylor expansion of $F$.
The polynomial part can be computed quickly since the expected values of monomials against a Gaussian measure (i.e., the moments) have a simply expression in terms of the variance.
The first nonzero contribution would come from the $u^4$ term in the Taylor expansion of $F$,
and it contributes a factor of the form $(t_1 t_2/(t_1+t_2))^5$,
up to constant that we ignore.
We are left with
\[
\int_{\ell}^L \d t_1 \int_{\ell}^L \d t_2 \frac{(t_1t_2)^{3}}{(t_1 + t_2)^{5}} 
\leq \int_{\ell}^L \d t_1 \int_{\ell}^L \d t_2 \, 2^{-5}\sqrt{t_1t_2} 
= 2^{-5}(L^{3/2} - {\ell}^{3/2})^2,
\]
where we use the arithmetic-geometric mean inequality $\sqrt{t_1t_2}/(t_1+t_2)\leq 1/2$ in the middle.
This expression has a finite limit as $\ell \to 0$.
The higher terms in the Taylor expansion contribute bigger powers of the variance 
and hence have $\ell \to 0$ limits.
Finally, the expected value of the error term of our partial Taylor expansion, 
which vanishes to some positive order at the origin,
can be bounded in such as way that an $\ell \to 0$ limit exists.
\end{proof}

We can now define the effective theory that we consider for the string. 

\begin{dfn}
The {\em renormalized action functional} at scale~$L$ for the holomorphic bosonic string is
\[
I[L] = \sum_{\Gamma \in {\bf Graph}_{\rm string}^{(0)}} w_{\Gamma}(P_{0}^L,I_{\rm string}) + \hbar\sum_{\Gamma \in {\bf Graph}_{\rm string}^{(1)}} w_{\Gamma}(P_{0}^L,I_{\rm string}).
\]
We denote the first summand---the tree-level expansion---by $S_0[L]$ 
and the second summand---the one-loop expansion---by~$S_1[L]$.
We use the notation $S[L] = S_{free} + I[L]$ where $S_{free}$ is the classical free part of the action functional. 
\end{dfn}

\begin{rmk} For any functional $J$, let $w(P_{\ell}^L, J)$ denote the sum over all graphs as above with the smooth propagator $P_{\ell}^L$ placed at the edges and $J$ placed at the vertices. 
Then, the family $\{I[L]\}$ satisfies the {\em homotopy RG equation}
\ben
I[L] = w(P_\ell^L , I[\ell]).
\een
The operator $w(P_\ell^L,-)$ defines a homotopy equivalence between the theory at scale $\ell$, defined using $S[\ell]$, and the theory at scale $S$, defined using $S[L]$. 
\end{rmk}

\subsection{The quantum master equation}
\label{subsec: QME}

In the BV formalism the basic idea is to replace integration against a path integral measure $e^{-S(\phi)/\hbar} \cD \phi$ with a cochain complex.
In this cochain complex, we view a cocycle as defining an observable of the theory,
and its cohomology class is viewed as its expected value against the path integral measure.
For toy models of finite-dimensional integration, see \cite{GJF};
these examples are always cryptomorphically equivalent to a de Rham complex,
which is a familiar homological approach to integration.

Hence the content of the path integral, in this approach, is encoded in the differential. 
A key idea is that the differential is supposed to behave like a divergence operator for a volume form:
recall that given a volume form $\mu$ on a manifold, 
its divergence operator maps vector fields to functions by the relationship
\[
{\rm div}_\mu({\mathcal X}) \mu = L_{\mathcal X} \mu.
\] 
This relationship, in conjunction with Stokes lemma, 
implies that if a function~$f$ is a divergence ${\rm div}_\mu({\mathcal X})$,
then $\int f \mu = 0$,
i.e., its expected value against the measure~$\mu$ is zero.
%(The toy models extend the divergence operator of a true volume form on a finite-dimensional manifold to an operator on polyvector fields.)
The BV formalism axiomatizes general properties of divergence operators;
a putative differential must satisfy these properties to provide a BV quantization.

When following the algorithm of Section~\ref{sec:bvoverview},
we want the renormalized action
\[
S = S^{\rm cl} + \hbar S_1 + \hbar^2 S_2 + \cdots
\]
to determine a putative differential $\d^q_S$ on the graded vector space of observables.
To explain this operator, we need to describe further algebraic properties on the observables
that the BV formalism uses.

First, in practice, the observables are the symmetric algebra generated by the continuous linear duals to the vector spaces of fields.
There is also a pairing on fields that is part of the data of the classical BV theory,
between each field and its ``anti-field.''
(This pairing is a version of the action of constant vector fields on functions in the toy models.)
In our case, there is the pairing between $b$ and $c$ and between $\beta$ and~$\gamma$, respectively.
It behaves like a ``shifted symplectic'' pairing as it has cohomological degree~$-1$,
and hence it determines a degree~1 Poisson bracket $\{-,-\}$ on the graded algebra of observables.
Finally, the pairing also determines a second-order differential operator $\Delta_{BV}$ on the algebra of observables by the condition that
\[
\Delta_{BV}(FG) = (\Delta_{BV}F)G + (-1)^F F(\Delta_{BV}G) + \{F,G\}.
\]
(This equation is a characteristic feature of divergence operators with respect to the product of polyvector fields.)

With these structures in hand, we can give the formula
\[
\d^q_S=\{S,-\} + \hbar\Delta_{BV}
\]
for the putative differential.
As $S$ has cohomological degree~0, the operator $\{S,-\}$ has degree~1.
We remark that modulo~$\hbar$, one recovers the differential $\{S^{\rm cl},-\}$ on the classical observables;
the zeroth cohomology of the classical observables is functions on the critical locus of the classical action~$S^{\rm cl}$.

By construction, this putative differential $\d_S^q$ satisfies the conditions of behaving like a divergence operator.
The only remaining condition to check is that it is square-zero.
This condition ends up being equivalent to $S$ satisfying the {\em quantum master equation}
\begin{equation}
\hbar \Delta_{BV} S + \frac{1}{2}\{S,S\} = 0.
\end{equation}
More accurately, $\d^q_S$ is a differential if and only if the right hand side is a constant.


\subsubsection{}

We now turn to examining this condition in our setting.
It helps to understand it is diagrammatic terms.

As the bracket is determined by a linear pairing,
it admits a simple diagrammatic description as an edge.
For instance, given an observable $F$ that is a homogeneous polynomial of arity~$m$
and an observable $G$ of arity~$n$, 
then $\{F,G\}$ has arity~$m+n-2$.
It can be expressed as a Feynman diagram 
%\brian{pic}
where the edge connecting $F$ and $G$ is labeled by a 2-fold tensor~$K$.

The BV Laplacian acts by attaching an edge labeled by~$K$ as a loop in all possible ways.
%\owen{Add picture.}
This diagrammatic behavior corresponds to the fact that $\Delta_{BV}$ is a constant-coefficient second-order differential operator.

The tensor~$K$ determined by the pairing on fields is distributional.
As one might expect from our discussion of divergences above,
these diagrammatic descriptions of the BV bracket and Laplacian are thus typically ill-defined.
In other words, the quantum master equation is {\em a priori} ill-posed for the same reason that the initial Feynman diagrams are ill-defined.
We can apply, however, the same cure of mollification.

\subsubsection{}

Costello's framework \cite{CosBook} provides an approach to renormalization built to be compatible with the BV formalism.
A key feature is that for each ``length scale''~$L>0$, 
there is a BV bracket $\{-,-\}_L$ and BV Laplacian $\Delta_L$.
The scale~$L$ renormalized action $S[L]$ satisfies the scale~$L$ quantum master equation~(QME)
\[
\hbar \Delta_{L} S[L] + \frac{1}{2}\{S[L],S[L]\}_L = 0
\]
if and only if $S[L']$ satisfies the scale~$L'$ quantum master equation for every other scale~$L'$, see Lemma 9.2.2 in \cite{CosBook}.
Hence, we say a renormalized action satisfies the quantum master equation if its solves the scale~$L$ equation for some~$L$.

Thus it remains for us to describe the scale~$L$ bracket and BV Laplacian in our setting,
so that we can examine whether the renormalized action satisfies the quantum master equation.

\begin{dfn}
The {\em scale~$L$ bracket} $\{-,-\}_L$ is given by pairing with the scale~$L$ heat kernel
\[
K_L(z,w) = \frac{1}{4\pi L} e^{-|z-w|/4L} (\d z - \d w) \wedge (\d\zbar - \d\overline{w}). 
\]
The {\em scale~$L$ BV Laplacian} $\Delta_L$ is given by the contraction~$\partial_{K_L}$.
\end{dfn}

These definitions mean that testing the quantum master equation leads to diagrams whose integrals are similar to those we encountered earlier.
We explain the diagrammatics and sketch the relevant integrals in the proof of the following result,
which characterizes when the string action admits a BV quantization.

We emphasize that up to now, we have not indicated explicitly which vector space $V$ is the target space for our string.
But the action functional explicitly depends on this choice,
so here we will write $S_V$ for the action with target~$V$.

\begin{prop} \label{prop anomaly}
The obstruction to satisfying the quantum master equation is the functional
\[
Ob_V[L] = \hbar \Delta_{L} S_V[L] + \frac{1}{2}\{S_V[L],S_V[L]\}_L.
\]
It has the form
\[
Ob_V[L] = \hbar (\dim_\CC(V) - 13) F[L],
\]
where $F[L]$ is a functional independent of~$V$.
\end{prop}

In short, the failure to satisfy the QME is a linear function of the dimension of the target space~$V$.
In particular, when $V \cong \CC^{13}$, 
the obstruction vanishes and the renormalized action {\em does} satisfy the QME, 
giving us an immediate corollary.
(Note that we do {\em not} need to know $F[L]$ to recognize that the obstruction vanishes!)

\begin{cor}
When the target vector space is 13-dimensional (i.e., has 26 real dimensions),
the holomorphic bosonic string admits a BV quantization.
\end{cor}

\begin{proof}
It is a general feature of Costello's formalism that the tree-level term $S_0[L]$ of the renormalized action satisfies the scale~$L$ equation
\[
\{S_0[L],S_0[L]\}_L = 0,
\]
known as the classical master equation.
Hence the first obstruction to satisfying the QME can only appear with positive powers of~$\hbar$.
We can also see quickly that no terms of~$\hbar^2$ appear:
the one-loop term $S_1[L]$ is only a function of the $c$~field, 
so 
\[
\{S_1[L],S_1[L]\}_L = 0 \quad\text{and}\quad \Delta_L S_1[L] = 0.
\]
Hence the obstruction to satisfying the QME is precisely
\[
\hbar\left( \{S_0[L],S_1[L]\} + \Delta_L S_0[L] \right).
\]
Thus we see that the obstruction is a multiple of~$\hbar$.
For simplicity, we will divide out that factor and let $Ob_V$ denote the term inside the parenthesis.

Consider the term $\{S_0[L],S_1[L]\}_L$. 
Diagrammatically, it corresponds to attaching a tree with a $b$ ``root'' to a wheel using an edge labeled by~$K_L$.
Arguments similar to Lemma 16.0.3 of \cite{wg2} carry over to account for the vanishing of this term in the $L \to 0$ limit. 

%\owen{Need to reference/explain the results of Si and KC.}

Now consider the term $\Delta_L S_0[L]$. 
Diagrammatically, it corresponds to turning a tree into a wheel by using an edge---labeled by $K_L$---to attach the root to an incoming leaf.
There are thus two kinds of wheels that appear, since there are two kinds of trees.
There are the wheels where the $K$ edge is for $bc$ fields.
Note that these wheels are the same for every choice of target $V$
as they only depend on the $bc$ fields, i.e., are independent of the $\beta\gamma$ fields.
These will contribute a term $F[L]$ to the obstruction.
On the other hand, there are the wheels where the $K$ edge is for $\beta\gamma$ fields.
These depend on $V$ but in a very simple way: 
the distribution $K$ is just the heat kernel tensored with the identity on~$V$, 
and hence the contraction amounts to taking $\dim_\CC(V)$ copies of the $V = \CC$ value.
In other words, the $\beta\gamma$ wheels contribute a term $\dim_\CC(V) G[L]$ to the obstruction,
where $G[L]$ is the value for $V = \CC$.
The last part of the proof of the theorem is a direct calculation of the functionals $F[L]$ and $G[L]$. 
So as to not diverge from our track of thought we include this calculation in Appendix \ref{sec:calculation} where we show that $F[L], G[L]$ are both independent of $L$ and satisfy $F = -13 G$, thus completing the proof.
\end{proof}

\begin{rmk} One can consider coupling the $\beta\gamma$ system to any tensor bundle on the Riemann surface. 
For instance, suppose $\gamma$ is a section of $T_\Sigma^{\tensor n}$ and hence $\beta$ is a section of $T^{* \otimes {n+1}}_\Sigma$.
In this case, one can show that the part of the obstruction with internal edges labeled by the $\beta\gamma$ propagators contributes a factor $(6n^2 + 6n + 1)G$, with $G$ the same functional above. 
\end{rmk}






\section{OPE and the string vertex algebra}

Vertex algebras are mathematical objects that axiomatize the behavior of local observables 
(i.e., point-like observables) of a chiral conformal field theory (CFT),
such as the $bc\beta\gamma$ system or the holomorphic bosonic string.
The vertex operator of a vertex algebras encodes the operator product expansions (OPE) for local observables,
which is of central interest in understanding a chiral CFT.
(We will not review vertex algebras here
as there are many nice expositions \owen{cite some}.)

In this section we will explain how to extract the vertex algebra of the holomorphic bosonic string,
using machinery developed in~\cite{CG1,LiVA,CDO}.
The answer we recover is precisely the chiral sector of the usual bosonic string.
\owen{Is that true?}

\subsection{A reminder on the chiral algebra of the string}

We provide a brief background on the vertex algebra for the chiral sector of the bosonic string. 
For a detailed reference we refer the reader to the series of papers \cite{LZ1,LZ2}. 
It is easiest to introduce this as a {\em differential graded vertex algebra}. 
This is simply a vertex algebra internal to the category of chain complexes. 
The underlying graded vertex algebra has state space of the form
\ben
\cV_{\beta \gamma}^{\tensor 13} \tensor \cV_{bc}
\een
where $\cV_{\beta\gamma}$ and $\cV_{bc}$ are the $\beta\gamma$ and $bc$ vertex algebras, respectively. 
The $\beta$ and $\gamma$ generators are in grading degree zero, the $c$ generator is in grading degree $-1$, and the $b$ is in grading degree $+1$. 
In the physics literature this is referred to as the {\em BRST} grading.

Forgetting the cohomological (or BRST) grading, this vertex algebra is a conformal vertex algebra of central charge zero (by construction). 
In particular, this means that the vertex algebra has a stress energy tensor. 
Explicitly, it is of the form
\ben
T_{\rm string} (z) = \left(\sum_{i = 1}^{13} \beta_i (z) \partial_z \gamma_i (z) + \partial_z \beta_i(z) \gamma_i (z) \right) + \left(b(z) \partial_z c(z) + 2 \partial_z b(z) c(z) \right) . 
\een
Note that $T_{\rm string}$ is of cohomological degree zero. 
The first parenthesis is interpreted as the stress energy tensor of the vertex algebra $\cV_{\beta \gamma}^{\tensor 13}$ and the second term is the stress energy tensor of $\cV_{bc}$. 

We have not yet described the differential on the graded vertex algebra. 
The BRST differential is defined to be the vertex algebra derivation obtained by taking the following residue
\ben
Q^{BRST} = \oint c(z) T_{\rm string}(z) .
\een 
By construction this operator satisfies $(Q^{BRST})^2 = 0$. 

\begin{dfn} The {\em string vertex algebra} is the dg vertex algebra 
\ben 
\cV_{\rm string} = \left(\cV_{\beta \gamma}^{\tensor 13} \tensor \cV_{bc}, Q^{BRST}\right)  .
\een
\end{dfn}


\subsection{Some context}

In the BV formalism one constructs a cochain complex of observables,
for both the classical and the quantized theory, if it exists.
The cochain complexes are local on the source manifold of a theory:
on each open set $U$ in that manifold~$\Sigma$,
one can pick out the observables with support in~$U$ by asking for the observables that vanish on fields with support outside~$U$.
It is the central result of~\cite{CG1,CG2} that the observables also satisfy a local-to-global property,
akin to the sheaf gluing axiom,
and hence form a {\em factorization algebra} on~$\Sigma$.

We will not need that general notion here.
Instead, we will use vertex algebras.
Theorem~\owen{???} of~\cite{CG1} explains how a factorization algebra~$F$ on $\Sigma = \CC$ 
(satisfying certain hypotheses) yields a vertex algebra~$\Vert(F)$. 
It assures us that the observables of a chiral CFT determine a vertex algebra.

In particular, Section~\owen{???} of~\cite{CG1} examines the free $\beta\gamma$ system in great detail.
Its main result is that the well-known $\beta\gamma$ vertex algebra is recovered by the two-step process of BV quantization, which yields a factorization algebra, and then the extraction of a vertex algebra.

The exact same arguments apply to the free $bc\beta\gamma$ system,
and we obtain the following.

\begin{prop}
\owen{State the $bc\beta\gamma$ VOA. Mention the classical (=associated graded) observables are a Poisson vertex algebra.}
\end{prop}

\subsection{The case of the string}

The holomorphic bosonic string is a chiral CFT and so the machinery of~\cite{CG1} applies to it.
One can extract a vertex algebra directly by this method.

But there is a slicker approach, using Li's work~\cite{Li},
which studies chiral deformations of free chiral BV theories such as the $bc\beta\gamma$ system.
Recall, a deformation of a classical field theory is given by a local functional. 
We have seen that this is essentially the data of a Lagrangian density, which is a density valued multilinear functional that depends on (arbitrarily high order) jets of the fields. 
In other words, for a field $\varphi$, a Lagrangian density is of the form
\ben
\cL(\varphi) = \sum (D_{k_1} \varphi) \cdots (D_{k_m} \varphi) \cdot {\rm vol}_\Sigma
\een 
for $C^\infty(\Sigma)$-valued differential operators $D_{k_i}$.
By a {\em chiral} Lagrangian density we mean a Lagrangian for which the differential operators $D_{k_i}$ are all holomorphic. 
For instance, on $\Sigma = \CC$, we require $D_{k_i}$ to be a sum of operators of the form $f(z) \partial_z^n$ where $f(z)$ is a holomorphic function. 
On $\Sigma = \CC$ we will also require the chiral Lagrangian to be translation invariant. 
This means that all differential operators $D_{k_i}$ are of the form $\partial_z^n$. 
Thus, a {\em translation invariant chiral deformation} is a local functional of the form
\ben
I(\varphi) = \sum \int (\partial^{k_1}_z \varphi) \cdots (\partial^{k_m} \varphi) \d^2 z .
\een

One of his main results is that for a free chiral BV theory with action $S_{\rm free}$ and associated vertex algebra $\cV_{\rm free}$, one has the following:
\owen{Check that this is an accurate summary of his work}
\begin{itemize}
\item For any chiral interaction~$I$, the action $S_I = S_{\rm free} + I$ needs no counterterms, 
and yields a renormalized action~$\{S_I[L]\}$.
\item If the renormalized action $\{S_I[L]\}$ satisfies the quantum master equation,
then it determines a vertex algebra derivation $D_I$ of~$\cV_{\rm free}$.
\item The vertex algebra $\cV_I$ for such an action $\{S_I[L]\}$ is isomorphic to the vertex algebra~$\cV_{\rm free}$ via the automorphism~$\exp(D_I)$.
\end{itemize}
The holomorphic bosonic string with target $V=\CC^{13}$ provides a concrete example of this situation.
The free theory is the $bc\beta\gamma$ system, 
and we have seen that the renormalized action satisfies the QME.
Hence we obtain the following.

\begin{prop} 
Let $\Obs^q_{\rm string}$ be the factorization algebra on $\Sigma = \CC$ of the holomorphic bosonic string with target~$V = \CC^{13}$. 
There is an isomorphism of vertex algebras $\cV_{\rm string} \cong \Vert(\Obs^q_{\rm string})$.
Moreover, this vertex algebra is isomorphic to the chiral sector of the bosonic string \owen{cite some description}.
\end{prop}

\owen{Compare with Lian-Zuckerman. We need to see that our differential on the dg vertex algebra is the BRST operator they write down.}

\owen{describe classical observables \& why we get a Poisson vertex algebra}


\brian{Write down vertex algebra from quantization above. Possibly state the relationship to semi-infinite cohomology}

\owen{I believe these vertex algebras are cohomologically graded, unless we're lucky and the cohomology all sits in degree zero. In which case, we should point out this miracle. Perhaps better would be to extract the dg vertex algebra.}

\owen{Do you know a citation where the string vertex algebra is already written down? Of course it's almost explicit in any discussion of the ``modern''/BRST quantization of the bosonic string, where they write down $Q$, which ought to be the differential of the dg vertex algebra using our construction.}

\owen{Si's theorem gives us a nice approach: we conjugate the free $bc\beta\gamma$ vertex algebra by the appropriate VOA automorphism to recover the string. This may be close to what physicists do anyway.}

\brian{How deformations discussed in Section 3 gives explicit deformations of the the vertex algebra.}

\section{The holomorphic string on closed Riemann surfaces} 
\label{sec: conformalblock}

Thus far we have discussed the local behavior of the holomorphic string,
such as its quantization on a disk and the concomitant vertex algebra.
Now we turn to its global behavior, 
particularly the observables on a closed Riemann surface,
and the relationship with certain natural holomorphic vector bundles on the moduli space of Riemann surfaces.
This local-to-global transition is where the BV/factorization package really shines.
On the one hand, the theory of factorization algebras provides a conceptual characterization of the local-to-global relationship,
much like the understanding of sheaf cohomology as the derived functor of global sections.
On the other hand, the examples from BV quantization provide computable, convenient models for the global sections,
much as the de Rham or Dolbeault complexes do for the cohomology of sheaves that arise naturally in differential or complex geometry.

As we will explain, the answers we recover for the holomorphic string can be related quite cleanly to natural determinant lines on the moduli of Riemann surfaces,
hence providing a bridge from the Feynman diagrammatic anomaly computations to the index-theoretic computations.

\subsection{The global observables in the free case}

Before jumping to the holomorphic string, 
we will work out the global observables in the simpler case of the free $bc\beta\gamma$ system,
introduced in Remark \ref{rmk:bcbg}. 
The global {\it classical}\/ observables on a Riemann surface $\Sigma$ are given by the symmetric algebra on the continuous linear dual to the fields,
\[
\Sym\left(\Omega^{0,*}(\Sigma,V)^\vee \oplus \Omega^{1,*}(\Sigma,V^\vee)^\vee \oplus \Omega^{0,*}(\Sigma,T[1])^\vee \oplus \Omega^{1,*}(\Sigma,T^*_\Sigma[-2])^\vee \right),
\]
with the differential $\dbar$ extended as a derivation.
Hence the cohomology is
\[
\Sym\left(H^*(\Sigma,V)^\vee \oplus H^*(\Sigma,\omega \otimes V^\vee)^\vee \oplus H^*(\Sigma,T[1])^\vee \oplus H^*(\Sigma,\omega^{\otimes 2}[-2])^\vee\right),
\]
where $\omega$ denotes the canonical bundle.
Although this expression might look complicated, 
it can be readily unpacked in the setting of algebraic geometry, 
particularly when $\Sigma$ is closed.
In that case, this graded commutative algebra is a symmetric algebra on a finite-dimensional graded vector space,
which encodes the derived tangent space of the moduli of Riemann surfaces at $\Sigma$ and of holomorphic functions from $\Sigma$ to~$V$ at the zero map. 

As this theory is free, it admits a canonical BV quantization.
Denote by $\Obs^{\q}_{free}$ the corresponding factorization algebra.
One can compute its global sections on $\Sigma$ by using a spectral sequence whose first page is the global classical observables.
The result of Theorem 8.1.4.1 of \cite{CG1} states that the cohomology of this free theory along a closed Riemann surface with values in {\em any} line bundle is one-dimensional and concentrated in a certain cohomological degree. 
In our case, it the calculation implies that we get a shifted determinant of the cohomology of the fields:
\ben
H^*\left(\Obs^\q_{free}(\Sigma)\right) \cong \det \left(H^*(\Sigma ; \sO_\Sigma) \right)^{\tensor \dim(V)} \tensor \det \left(H^*(\Sigma ; T_\Sigma^{1,0})\right)^{-1} [d(\Sigma)] 
\een
where 
\begin{align*}
d(\Sigma) = & \dim (V)  \left(\dim H^0(\Sigma ; \sO_\Sigma) + \dim H^1(\Sigma ; \sO_\Sigma)\right) \\
&+ \dim(H^0(\Sigma ; T_\Sigma^{1,0})) - \dim(H^1(\Sigma ; T_\Sigma^{1,0})).
\end{align*}
(The meaning of this shift is not completely clear to us.)

\begin{rmk}
The shift $d(\Sigma)$ here likely looks funny.
In this case at least, the meaning can be unpacked pretty straightforwardly. 
The BV complex for an ordinary finite-dimensional vector space is equivalent to the de Rham complex shifted down by the dimension of the vector space, 
so that the top forms are in degree 0.
(Abstracting this situation is one way to ``invent'' the BV formalism.)
For the $\sigma$-model, the global solutions to the equations of motion are $H^0(\Sigma,\sO) \otimes V$ for the $\gamma$ fields and $H^0(\Sigma,\omega) \otimes V^\vee$ for the $\beta$ fields.
For $\Sigma$ closed, these are finite-dimensional, and thus we get the shift
\[
 \dim (V)  \left(\dim H^0(\Sigma ; \sO_\Sigma) + \dim H^1(\Sigma ; \sO_\Sigma)\right).
\]
For the ghost system (the $bc$ fields), 
the BV complex recovers the Euler characteristic 
\[
\dim(H^0(\Sigma ; T_\Sigma^{1,0})) - \dim(H^1(\Sigma ; T_\Sigma^{1,0}))
\]
as it encodes the de Rham complex on the formal quotient stack $B\fg = \ast/\fg$ for the Lie algebra of symmetries~$\fg$.
\end{rmk}

The computation here works for any Riemann surface $\Sigma$ and, indeed, for any family of Riemann surfaces.
Hence it implies that the global observables of the free $bc\beta\gamma$ system determine a determinant line bundle on the moduli $\cM$ of Riemann surfaces.
We will identify {\em which} line bundle we get after examining the global observables of the holomorphic string.

\subsection{The global observables for the holomorphic string}

The cohomology of the global observables $\Obs^\q(\Sigma)$ of the holomorphic string on a closed surface~$\Sigma$ is also surprisingly easy to compute.

Consider the filtration on the quantum observables induced by the polynomial degree of the functional. 
\owen{Eugene: This is not a filtration, since BV Laplacian lowers symmetric degree. Perhaps use homological perturbation lemma?}
\brian{Instead of a spectral sequence, let's use the harmonic representative for the factorization homology. The interaction restricts to the harmonic subcomplex and I think for degree reasons can't do anything to the cohomology.}
\owen{I think you're implicitly using a spectral sequence argument then. We should discuss this point.}
There is a spectral sequence abutting to the cohomology of the global observables $H^*\Obs^\q(\Sigma)$ with $E_1$ page given by the cohomology of the global observables of the free $bc\beta \gamma$ system which we have already computed:
\bestar
E_2 & \cong & \det\left(H^*(\Sigma ; T_\Sigma[1])\right) \tensor \det \left(H^*(\Sigma ; \cO_\Sigma)^{\oplus 13}\right) \\
& \cong & \det \left(H^1(\Sigma ; T_\Sigma) \right) \tensor \det \left(H^0(\Sigma ; T_\Sigma)\right)^{-1} \tensor \det \left(H^0(\Sigma ; K_{\Sigma}) \right)^{-13}
\eestar
where we have used the fact that $H^0(\Sigma ; \cO) \cong \CC$ for any $\Sigma$. 
Since this page is concentrated in a single line, we see that the spectral sequence degenerates at this page.

Let $\Sigma_{g}$ be a surface of genus $g$. 
For $g=1$, the above expression simplifies to
\ben
\det \left(H^1(\Sigma_1 ; T_{\Sigma_1})\right) \tensor \det \left(H^0(\Sigma_1 ; K) \right)^{-14} .
\een 
For $g \geq 2$, one has
\ben
\det \left(H^1(\Sigma_1 ; T_{\Sigma_1})\right) \tensor \det \left(H^0(\Sigma_1 ; K) \right)^{-13} .
\een
Thus the above expressions give the global observables for the holomorphic string for genus $g =1$ and $g \geq 2$, respectively. 
Compare these formulas to Witten's analysis of the bosonic string in Section 2.1 of~\cite{WitString}.

\subsection{Identifying the determinant lines}

We now work out the first Chern class of this determinant line bundle using the Grothendieck-Riemann-Roch theorem.
Consider the universal Riemann surface $\pi \colon C \to \cM$ over the moduli space, 
and consider the bundles $\sO_C \otimes V$ and the relative tangent sheaf $\sT_\pi = \sT_{C/\cM}$.
(These encode the universal $\gamma$ fields and $c$ fields, respectively.)
The first Chern class of the derived pushforward $R\pi_*(\sO_C \otimes V)$ is given by the first Chern class of $\det(H^*(\sO_C \otimes V)) \cong \det(\sO_C)^{\otimes \dim V}$, 
since the first Chern class of a vector bundle is the first Chern class of its determinant bundle.
The Grothendieck-Riemann-Roch (GRR) theorem states that for a cochain complex of coherent sheaves $\cF^\bullet$ on $C$, 
the Chern character $\ch(R\pi_* \cF)$ of its derived pushforward $R\pi_* \cF$  is given by 
\def\Td{{\rm Td}}
\[
\pi_*( \ch(\cF^\bullet)\Td(\sT_\pi)) = \pi_*\left( \left(\sum_{i} (-1)^i \ch (\cF^i) \right) \Td(\sT_\pi)\right).
\]
Unraveling the definitions, we see that the class to be pushed forward expands~to
\[
\left(\sum_i (-1)^i ({\rm rk}(\cF^i) + c_1(\cF^i) + \frac{1}{2}(c_1(\cF^i)^2) + \cdots)\right) \left(1 +\frac{1}{2}c_1(\sT_\pi) + \frac{1}{12} c_1(\sT_\pi)^2 + \cdots \right).
\]
%since $\sT_\pi$ is the tangent line bundle of a Riemann surface.
When we pushforward, we integrate out the fiber direction, so along a Riemann surface.

We are interested in the first Chern class of the pushforward $R\pi_* \cF$,
which is the component of cohomological degree~2.
Thus we want to take the fiberwise integral of the degree 4 component of $\pi_*( \ch(\cF^\bullet)\Td(\sT_\pi))$,
which we do by using the expansion.
For instance, if $\cF = \cF^0$ is concentrated in degree zero, the relevant expression simplifies~to
\ben
\frac{1}{12} {\rm rk}(\cF) c_1(\sT_\pi)^2 + \frac{1}{2} c_1(\cF) c_1(\sT_\pi) + \frac{1}{2} c_1(\cF)^2 .
\een  
As another example, if $\cF = \sT^{\tensor n}_{\pi}$, the expression for the first Chern class~is 
\[
\frac{1 + 6n + 6n^2}{12} c_1(\sT_\pi)^2.
\]
And if $\cF = \cO \tensor V$, we simply get~$\dim(V)/12$. 

For the free $bc\beta\gamma$ system,
we have $\cF = \sT[1] \oplus (\cO \tensor V)$.
We know that the global observables $H^*\left(\Obs^\q_{free}(C)\right)$ provide a determinant line,
and the computations then imply
\[
c_1\left(H^*\left(\Obs^\q_{free}(\Sigma)\right)\right) = \frac{1}{12} (\dim(V) - 13) c_1(\sT_\pi)^2 .
\]
(Note the sign change due to shifting the relative tangent bundle.)

It is worthwhile to point out that the above argument based on GRR for identifying the first Chern class of this determinant line bundle resonates with our computation of the anomaly of the bosonic string on the disk. 
%Indeed, this is a manifestation of ``Virasoro uniformization.'' 
Also, notice that the above calculation assumed that there was no deformation, so that we were working with a free theory. 
However, deforming the action from free $bc\beta\gamma$ system to holomorphic bosonic string should not affect the line bundles, 
since varying the action involves adjusting continuous parameters (the coupling constants) and Chern classes are discrete invariants.

\subsection{The anomaly and moduli of quantizations on an arbitrary Riemann surface}

We have already seen that the holomorphic string {\it on a disk} admits a BV quantization if and only if the target is a complex vector space of dimension 13.
Here we will explain why this anomaly calculation is actually enough to show the existence of a quantization on an {\it arbitrary} Riemann surface. 
An argument using the GRR theorem was given in the preceding section. 
In this section we give a proof using only the perspective of BV quantization.
One can view this approach as giving a proof of (a piece of) the GRR theorem using Feynman diagrams (and will be the topic of future work). 

Our diagrammatic arguments show that only wheels with $c$ legs appear in the anomaly,
and these arguments did not depend on the choice of $\Sigma$. 
Hence the anomaly will be purely a functional on the $c$ fields.
We thus restrict ourselves to the relevant piece of the deformation complex,
the component only involving such fields.
By arguments analogous to those in Section~\ref{sec: moduli}, 
when $\Sigma$ is the disk,
this component is quasi-isomorphic to $\cred^*(W_1)[2]$,
whose cohomology is $\CC$ concentrated in degree~1.
More generally, the deformation complex is a sheaf of cochain complexes on $\Sigma$, 
and Proposition 5.3 of \cite{BWvir} shows that this sheaf of complexes is quasi-isomorphic to the constant sheaf $\CC_\Sigma[-1]$ concentrated in degree~1. 

Since the construction of BV quantization is manifestly {\em local-to-global} on spacetime, 
anomalies inherit this property: 
the anomaly computed on an open set $U \subset \Sigma$ is equal to the anomaly of the theory on $\Sigma$ restricted to $U$. 
In our case, the anomaly on some Riemann surface $\Sigma$ must match with the anomaly we have already computed diagrammatically, if we take $U$ to be a disk in $\Sigma$.
This global anomaly is thus a 1-cocycle for the derived global sections of the shifted constant sheaf $\CC_\Sigma[-1]$.
Because of the shift, this cocycle is determined by a constant function on $\Sigma$.
Thus, it suffices to compute the anomaly on an arbitrary open,
and in particular, it suffices to compute it on a flat disk. 
But this context is precisely where we computed the anomaly in Section~\ref{sec: quantization}, 
so we know the anomaly is simply the dimension of the target vector space.
Thus, a quantization of the holomorphic string exists on any Riemann surface provided $\dim_{\CC}(V) = 13$. 

Now we ask how many such quantizations are possible,
i.e., what is the moduli of quantized theories.
By the calculation in Section~\ref{sec: moduli}, 
we know that, up to equivalence of BV theories, 
the possible one-loop terms in the quantized action functional are parametrized by
\ben
H^0(\Sigma) \tensor \Omega^1(V) \oplus H^1(\Sigma) \tensor \Omega^2_{cl}(V).
\een 
(That is, these vector spaces are the zeroeth cohomology group of the relevant deformation complex.)
This space of deformations corresponds to continuous parameters that we can vary in the action functional.
As the isomorphism classes of line bundles form a discrete set, 
varying these continuous parameters will not change the class of the line bundle of global observables. 
In conclusion, no matter what one-loop quantization we choose, 
the cohomology of the global observables will be the same.


\section{Looking ahead: curved targets}
\label{sec:curved}

\owen{I think here we can mention our CDO work and assert that it's compatible with the discussion here. Then we state the corresponding theorems.}

\brian{State the quantization condition for curved target.} 

\appendix

\section{Calculation of the anomaly} \label{sec:calculation}

\owen{Sometimes integrals have top form components and sometimes not. I don't care if we have them or not, but we should be consistent (and explain at the beginning what we're suppressing if we do drop them).}

In this section we compute the functionals $F[L]$ and $G[L]$ mentioned in the proof of Proposition \ref{prop anomaly}, hence completing the calculation of the anomaly. 

We have reduced the calculation to the weight of two wheel diagrams: A) with internal edges labeled by the $bc$ heat kernel and propagator, respectively. B) with internal edges labeled by the $\beta\gamma$ heat kernel and propagator, respectively.
The weight of A gives the functional we called $F[L]$, and the weight of B gives the functional we called $\dim_\CC(V) G[L]$. 

We will utilize the following version of Wick expansion to evaluate the integrals below. 

\begin{lem}
\label{lem wick} 
Let $\Phi$ be a smooth compactly supported function on $\CC$ and let $\tau > 0$. 
Then
\ben
\int_{\xi \in \CC} \Phi(\xi) e^{-\tau |\xi|^2/4}  = 4 \pi  \tau^{-1} \left(\exp\left(\tau^{-1} \frac{\partial}{\partial \xi} \frac{\partial}{\partial \xi} \Phi\right)_{\xi = 0}\right) .
\een
\end{lem}

Note that we suppress the term $\d^2 \xi$ from the integral, for brevity's sake.
As we are integrating over vector spaces here, 
one can recover the integrand by taking the Lebesgue measure for the variable labeled under the integral sign (e.g., $z \in \CC$ corresponds to~$\d^2 z$).

We now turn to the weight of diagram A. 
Use coordinates $z,w$ to denote the coordinates at each of the vertices.
Denote the inputs of the weight by the compactly supported vector fields $f(z) \partial_z$ and $g(w) \d \wbar \partial_w$.
(Note that the diagram is only nonzero if the total degree of the elements is $+1$.)
If $c(z) \partial_z$ is another vector field, the action by $f(z) \partial_z$ is given by 
\ben
[f(z) \partial_z, c(z) \partial_z] = f(z) \partial_z c(z) \partial_z - c(z) \partial_z f(z) \partial_z .
\een 
Thus, the weight of diagram $A$ can be written as the $\ell \to 0$ limit of
\be
\begin{array}{ccc}
\displaystyle
& & \int_{z,w} f(z) \partial_z P_{\ell}^L(z,w) g(w) \partial_w K_\ell(z,w) \\
&-& \int_{z,w} \partial_z f(z) P_{\ell}^L(z,w) g(w) \partial_w K_\ell (z,w) \\
&-& \int_{z,w} f(z) \partial_z P_\ell^L(z,w) \partial_w g(w) K_\ell (z,w) \\
&+& \int_{z,w} \partial_z f(z) P_\ell^L(z,w) \partial_w g(w) K_\ell (z,w) .
\end{array}
\ee
We label the integrals in each line above as I,II, III, IV, respectively. 

Using the form of the propagator in (\ref{propagator}) we see that line I is given by
\ben
{\rm I} = \frac{1}{(4 \pi)^2} \int_{(z,w) \in \CC \times \CC} \int_{t = \ell}^L f(z) g(w) \frac{1}{\epsilon^2} \frac{1}{t^3} \frac{(\zbar - \wbar)^3}{8} \exp \left(-\frac{1}{4}\left(\frac{1}{\ell} + \frac{1}{t}\right)|z-w|^2 \right)
\een
(we are omitting volume factors for simplicity). 
To evaluate this integral we change variables and apply the Wick expansion, Lemma \ref{lem wick} to one of the variables of integration. 
Indeed, introduce $\xi = z -w$, and notice that the integral simplifies to
\ben
{\rm I} = \frac{1}{(4 \pi)^2} \int_{w \in \CC} \int_{\xi \in \CC} \int_{t = \ell}^L f(\xi + w) g(w) \frac{1}{\epsilon^2} \frac{1}{t^3} \frac{\Bar{\xi}^3}{8} \exp \exp \left(-\frac{1}{4} \left(\frac{1}{\ell} + \frac{1}{t}\right)| \xi |^2 \right) .
\een
Applying Lemma \ref{lem wick} to the $\xi$-integral we see that this simplifies to
\ben
{\rm I} = \frac{1}{4 \pi} \int_{w \in \CC} \partial^3_w f(w) g(w) \int_{t = \ell}^L \frac{\ell^2 t}{(\ell + t)^4} + O(\ell)
\een
where the terms $O(\ell)$ are of order $\ell$ so are zero in the limit $\ell \to 0$. 
On the other hand, we can evaluate the remaining $t$-integral and see that in the limit $\ell \to 0$ Line I becomes
\ben
\lim_{\ell \to 0} \; {\rm I} = \frac{1}{4 \pi} \frac{1}{12} \int_{w\in \CC} \partial^3_w f(w) g(w) .
\een 

We evaluate II, III, and IV in a similar fashion.

After changing coordinates and performing the Wick type integral we obtain
\ben
{\rm II} = \frac{1}{4 \pi} \int_{w \in \CC} \partial^3_w f(w) g(w) \, \int_{t = \ell}^L \frac{\ell t}{(\ell + t)^3} + O(\ell) .
\een
Evaluating the remaining $t$ integral and taking $\ell \to 0$ this becomes 
\ben
\lim_{\ell \to 0} {\rm II} = \frac{1}{4 \pi} \frac{3}{8} \int_{w\in \CC} \partial^3_w f(w) g(w) .
\een 

Integral III is given by 
\ben
\frac{1}{4\pi} \int_{w \in \CC} \partial_w^3 f(w) g(w)\, \int_{t= \ell}^L \frac{\epsilon^2}{(\epsilon + t)^3}  + O(\ell) .
\een
In the limit $\ell \to 0$ we obtain
\ben
\lim_{\ell \to 0} \; {\rm III} = \frac{1}{4\pi} \frac{1}{8} \int_{w \in \CC}  \partial^3 f(w) g(w)  .
\een

Finally, integral IV is
\ben
\frac{1}{4\pi} \int_{w \in \CC} \partial_w^3 f(w) g(w) \, \int_{t= \ell}^L \frac{\epsilon}{(\epsilon + t)^2}  + O(\ell) .
\een
In the limit $\ell \to 0$ we obtain
\ben
\lim_{\ell \to 0} \; {\rm IV} = \frac{1}{4\pi} \frac{1}{2} \int_{w \in \CC}  \partial^3_w f(w) g(w) .
\een

In total, the functional $F[L]$ applied to $(f(z) \partial_z, g(w) \d \wbar \partial_w)$ is given by
\ben
F[L] (f(z) \partial_z, g(w) \d \wbar \partial_w) = - \frac{1}{4 \pi} \frac{13}{12} \int_{w \in \CC}  \partial^3_w f(w) g(w)  .
\een
Note that this functional is independent of $L$. 

Diagram B is similar to A, except the internal edges are labeled by the $\beta\gamma$ propagator. 
Applied to the input vector fields $(f(z) \partial_z, g(w) \d \wbar \partial_w)$ the weight is given by the dimension of $V$ times the integral we computed in $I$.
Thus
\ben
G[L](f(z) \partial_z, g(w) \d \wbar \partial_w) = \frac{1}{4 \pi} \frac{1}{12} \int_{w\in \CC} \partial^3_w f(w) g(w)  .
\een
The proposition follows.


\bibliographystyle{amsalpha}

\bibliography{string}

\end{document}