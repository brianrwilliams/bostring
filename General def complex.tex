\documentclass{article}

\usepackage{macros}

\usepackage{tikz,tikz-cd}
\usetikzlibrary{arrows,shapes}
\usetikzlibrary{trees}
\usetikzlibrary{matrix,arrows}
\usetikzlibrary{positioning}
\usetikzlibrary{calc,through}
\usetikzlibrary{decorations.pathreplacing}
\usepackage{pgffor}

\title{Deformations of the holomorphic string}
\author{}
\date{2018-4-19}
\begin{document}
\maketitle

In this section $\fg$ will denote an arbitrary curved $L_\infty$ algebra defined over a dg ring $R$.

\begin{prop}
Let $\Sigma$ be a Riemann surface and $\fg$ any curved $L_\infty$ algebra.
Then, there is a quasi-isomorphism of sheaves on $\Sigma$,
\ben
\cloc^*(\sT_\Sigma \ltimes \fg_\Sigma) \simeq \Omega^*_\Sigma \tensor \left(\CC[-1] \oplus \Omega^2_{cl}(B \fg)[1] \oplus \Omega^1(B \fg) \oplus \Omega^1_{cl}(B \fg)[-1]\right).
\een
\end{prop}

\begin{proof}
\ben
\cloc^*(\sT_\Sigma \ltimes \fg_\Sigma) \simeq \Omega^*\left(\Sigma , \cred^*(J \sT_\Sigma \ltimes J \fg_\Sigma)\right)[2]
\een
The right-hand side is quasi-isomorphic to the de Rham complex
\ben
\Omega^* \left(\Sigma, \cred^*(J^{hol} T^{1,0} \Sigma \ltimes J^{hol}(\ul{\CC}_\Sigma \tensor \fg)\right)[2] .
\een

\begin{lem}
There is a quasi-isomorphism between the local cohomology of $\sT_\Sigma \ltimes \fg_\Sigma$ and (a shift of) the Gelfand-Kazhdan descent of the $(\W_1, \CC^\times)$-module $\cred^*(\W_1 \ltimes \fg[[z]])$,
\ben
\cloc^*(\sT_\Sigma \ltimes \fg_\Sigma) \simeq \bdesc_X \left( \cred^*(\W_1 \ltimes \fg[[z]]) \right)[2] .
\een
\end{lem}

We compute the Gelfand-Fuks cohomology in the following lemma.

\begin{lem}
There is a quasi-isomorphism
\ben
\cred^*({\rm W}_1 \ltimes \fg[[z]]) \simeq \cred^*({\rm W}_1) \oplus \Omega^{2}_{cl}(B \fg)[-1] \oplus \Omega^1(B \fg)[-2] \oplus \Omega^1_{cl}(B \fg) [-3] .
\een
In the case $\fg = \fg_n$ the quasi-isomorphism still holds provided we replace $\Omega^i(B \fg)$ by the formal de Rham forms $\hOmega^i_n$.
Moreover, in this case the quasi-isomorphism is $(\Vect, \GL_n)$-equivariant.
\end{lem}

\begin{proof}

Note that we can rewrite 
\ben
\cred^*(\W_1 \ltimes \fg[[z]]) \cong \cred^*(\W_1) \oplus \clie^*(\W_1 ; \cred^*(\fg[[z]])) .
\een
We will invoke the following special result about the cohomology of $\W_1$ with coefficients in any module.
Recall that the vector field $L_0 = z \frac{\partial}{\partial z} \in \W_1$ determines a grading by conformal dimension on $\W_1$ and any $\W_1$-module.
Indeed, if $m \in M$, we say that $m$ has {\em conformal dimension} $k$ if $L_0 m = k m$. 
 
\begin{lem} Let $M$ be any $\W_1$-module. 
Then, the inclusion of the dimension zero subcomplex
\ben
\begin{tikzcd}
\clie^*(\W_1 ; M)^{(0)} \arrow[r,hook,"\simeq"] & \clie^*(\W_1 ; M)
\end{tikzcd}
\een
is a quasi-isomorphism.
\end{lem}

\end{proof}

Finally, to finish the proof of the proposition we observe the following fact.

\begin{lem}
Suppose $M$ is any $\fg$-module.
Then, the $\fg$-module structure on the Chevalley-Eilenberg complex $\clie^*(\fg ; M)$ is homotopically trivial.
\end{lem}

\end{proof}

\end{document}

