\documentclass{amsart}

\usepackage{fullpage,color,mathrsfs,amssymb,xypic}

\setlength{\parindent}{0pt} 
\setlength{\parskip}{2ex}% plus 0.5ex minus 0.2ex}

\def\bgs{\beta\gamma}
\def\CC{\mathbb C}
\def\NN{\mathbb N}
\def\RR{\mathbb R}
\def\xbar{\overline{x}}
\def\ybar{\overline{y}}
\def\d{{\rm d}}
\def\cX{\mathcal X}
\def\fg{\mathfrak g}
\def\cL{\mathcal L}
\def\ot{\otimes}
\def\Td{\rm Td} % Todd class
\def\ch{\rm ch} % Chern character
\def\sk{\rm sk} % skeleton
\def\Grav{\mathscr G} % symbol for holomorphic vector fields as local Lie algebra for ``gravity''
\def\ob{\Theta} % symbol for the obstruction to quantization
\def\sO{\mathscr O}
\def\gl{\frak{gl}}
 

\def\owen{\textcolor{red}}
\def\brian{\textcolor{blue}}

\newtheorem{prop}{Proposition}
\newtheorem{lemma}{Lemma}
\newtheorem{cor}{Corollary}
\newtheorem{theorem}{Theorem}

\begin{document}

\owen{Comments by Owen look like this.}

\brian{Comments by Brian look like this.}

We will outline how to describe the deformation complex for the holomorphic bosonic string with target a complex manifold $X$. (Following Costello, we will interpret $X$ as an $L_\infty$ space $B\fg_X$, so our computations will include a larger world of examples, but \emph{not} the instantonic contributions to the nonlinear $\sigma$-model.) Our source manifold will be $\CC$, although we will discuss what changes when working with an arbitrary Riemann surface $S$.

\section{Background}

Let $S$ denote a Riemann surface.

The deformation complex is always given by the local functionals modulo constants.  (\emph{Local} here refers to the source $S$.) Alternatively, it is the reduced local Chevalley-Eilenberg cochains on a local dg Lie algebra. For us, the local Lie algebra is the following cochain complex
\[
\Dol{0}(S,T^{1,0}_S) \oplus \Dol{1}(S, T^{\ast 1,0}_S) \oplus \Dol{0}(S) \ot \fg_X \oplus \Dol{1}(S) \ot \fg_X^\vee,
\]
equipped with a Lie algebra structure that we now describe. The first summand is the Dolbeault complex of the holomorphic vector fields on $S$, and it is both a sub and a quotient Lie algebra. The second summand has no internal Lie structure but has the natural action of holomorphic vector fields. The third summand is a (complicated and curved) $L_\infty$ algebra, inherited from that for $\fg_X$; it also receives a natural action from the holomorphic vector fields. The final term is a module for both the first and third terms.

Let's introduce some notation to clarify the exposition. We write
\[
\aut(S) := \Dol{0}(S,T^{1,0}_S)
\]
for the Dolbeault complex of holomorphic vector fields on $S$, as these are the infinitesimal automorphisms of $S$ as a Riemann surface. We write
\[
\fg_X^S := \Dol{0}(S) \ot \fg_X
\]
for the curved $L_\infty$ algebra (over base ring $\Omega^\ast(X)$) that describes the formal neighborhood of the constant maps inside the space of all holomorphic maps. \owen{That description is perhaps a little unclear.} Note that $\aut(S)$ acts on $\fg_X^S$ by extending the natural action of holomorphic vector fields on holomorphic functions. We make the situation more symmetric with respect to $S$ and $X$ by working with 
\[
\aut(S)_X := \Omega^*(X) \otimes \aut(S),
\]
which is locally constant on $X$. 

Consider the Lie algebra
\[
\fg_X^S//\mathfrak{aut}(S)_X := \aut(S)_X \ltimes \fg_X^S. 
\]
which describes the derived quotient by holomorphic automorphisms of $S$ of the formal neighborhood of constant maps from $S$ to $X$. (What a mouthful!) Note that the automorphisms of $S$ are both a sub and quotient Lie algebra of this whole thing.

The Lie algebra describing the holomorphic bosonic string is the split square zero extension of this Lie algebra by its coadjoint module.

We want to quantize this \emph{cotangent theory}, which has several simplifying consequences. First, a cotangent quantization only involves one-loop contributions -- i.e., $\hbar$ additions to the differential -- and so it is much more computable if it exists. Second, the deformation complex as a cotangent theory is 
\[
C^*_{loc,red}(\fg_X^S//\mathfrak{aut}(S)_X),
\]
so that we can ignore the coadjoint summands.

The structure of this Lie algebra, as an extension of the holomorphic vector fields, induces a natural retract of commutative dg algebras
\[
\Omega^\ast_X \ot_\CC C^*_{loc}(\aut(S)) \to C^*_{loc}(\fg_X^S//\mathfrak{aut}(S)) \to \Omega^\ast_X \ot_\CC C^*_{loc}(\aut(S)).\footnote{Note that this is a retract on the nose. Note also that $C^*(\aut(S)_X) \cong \Omega^*(S) \ot C^*(\aut(S))$ by base change.}
\]
\owen{Is this true? I'm worried about how things interact with the curving \dots} Thus
\[
C^*_{loc}(\fg_X^S//\mathfrak{aut}(S)) \cong \Omega^\ast_X \ot_\CC C^*_{loc}(\aut(S)) \oplus C^*_{loc}(\aut(S)_X, C^*_{red}(\fg^S_X))
\]
Thus, we obtain a useful splitting:
\[
C^*_{loc,red}(\fg_X^S//\mathfrak{aut}(S)_X) \cong \Omega^\ast_X \ot_\CC C^*_{red,loc}(\aut(S)) \oplus C^*_{loc}(\aut(S)_X, C^*_{red}(\fg^S_X)).
\]
The first summand is where the obstruction for the string with flat target lives; in particularly, we will see here the same ``conformal anomaly'' appear.

We will now examine the second summand, where a dependence on $X$ appears.

\section{The disk as source}

Let $S$ be $\CC$. Let $\cT$ denote the sheaf of holomorphic vector fields, which possesses a canonical inclusion $\cT \hookrightarrow \aut$. In this setting, it is a quasi-isomorphism on every open. Similarly, we have an inclusion and quasi-isomorphism $\sO \ot \fg_X \hookrightarrow \fg_X^S$. This induces a quasi-isomorphism
\[
C^*_{loc}(\aut(S)_X, C^*_{red}(\fg^S_X)) \to C^*_{loc}(\cT, C^*{red}(\sO \ot \fg_X)).
\]
We will focus on the right hand term.

\owen{Is there a characterization of $C^*_{loc}$ as simply the Lie algebra cohomology in the sheaf-theoretic setting? I think this might make some formal arguments cleaner. Note that a map of sheaves between $C^\infty$ modules is always a differential operator, by Peetre's theorem, so I think this might be true. (There's some fine print to worry about there.)}

\owen{I now want to take the invariants with respect to the Lie subalgebra $\CC z \partial_z = \CC L_0$. If everything were purely algebraic, then everything would nicely decompose as a sum/product over weights, but on an arbitrary open in $\CC$, this isn't really true. What are the magic words here?}

\subsection{Identifying the weight zero subcomplex}

The role of $\fg_X$ here is bookkeeping, so we will simply denote it $\fg$. We will use the coordinate $z$ on $\CC$, and let $\partial$ denote $\partial/\partial z$/

For the Lie algebra $\CC[[z]]\partial$, we have a natural basis $L_n = iz^{n+1} \partial$, with $n \geq -1$. Note that 
\[
[L_m,L_n] = (m-n) L_{m+n}.
\]
Then $L_0$ induces a natural weight filtration on everything in sight. For instance, $z^k$ has weight $k$, $\partial = L_{-1}$ has weight $-1$, and $\d z$ has weight $1$.

We want to compute Lie algebra cohomology, so we work with the dual spaces. We denote the dual basis vector to $L_n$ by $\lambda^n$ and the dual to $z^k$ by $\zeta_k$ (which can also be written as $(1/k!)\partial^k\delta_0$). Note that $L_m \zeta_n = (n-m) \zeta_{n-m}$. For concision's sake, if $\xi$ is an element of $\fg^\vee$, we write $\xi \zeta_k$ for the element $\xi \ot \zeta_k$. 

Since $\zeta_k$ has weight $-k$, we see that every $L_0$-eigenvector in $\fg^\vee \otimes \CC[[z]]^\vee$ has nonpositive weight. Similarly, every element $\lambda^k$ has weight $1-k$. Thus, the weight zero subspace of
\[
C^*(\CC[[z]]\partial, C^*_{red}(\fg[[z]])),
\] 
as a graded vector space, is
\begin{align*}
C^\sharp(\CC[[z]]&\partial)_{\wt=0} \ot C^\sharp_{red}(\fg[[z]])_{\wt=0} \\
&\oplus C^\sharp(\CC[[z]]\partial)_{\wt=1} \ot C^\sharp_{red}(\fg[[z]])_{\wt=-1}
\end{align*}
A basis for the weight 0 component for the vector fields is $1$, $\lambda^{0}$, $\lambda^{-1} \wedge \lambda^{1}$,  and $\lambda^{-1} \wedge \lambda^0 \wedge \lambda^{1}$. A basis for the weight 1 component for the vector fields is $\lambda^{-1}$ and $\lambda^{-1} \wedge \lambda^0$.  A term in the weight 0 component for the module is given by sums of elements like
\[
(\xi_1 \zeta_0) \wedge (\xi_2 \zeta_0) \wedge \cdots \wedge ( \xi_k \zeta_0),
\]
where the $\xi_j$s denote arbitrary elements of $\fg^\vee$.  (The important point is that every such term is paired with $\zeta_0$.) A term in the weight -1 component for the module is given by sums of elements like
\[
(\xi_1 \zeta_1) \wedge (\xi_2 \zeta_0) \wedge \cdots \wedge ( \xi_k \zeta_0),
\]
where the $\xi_j$s denote arbitrary elements of $\fg^\vee$.  (The important point is that precisely one copy of $\zeta_1$ appears, and the rest are $\zeta_0$s.)

Let's analyze now the differential on this complex. There is the internal differential of the module $C^*_{red}(\fg[[z]])$, which preserves the weight grading. We make the following observation.

\begin{lemma}
We have
\[
C^\ast_{red}(\fg[[z]])_{\wt=0} \cong C^*_{red}(\fg).
\]
In other words, the weight zero component of the module is simply the reduced Lie algebra cohomology $\fg$.

Similarly,
\[
C^\ast_{red}(\fg[[z]])_{\wt=-1} \cong C^*(\fg,\fg^\vee).
\]
\end{lemma}

With respect to the interpretation of $\fg$ as providing a formal space $B\fg$, we see that the weight zero component is $\sO_{red}(B\fg)$ and the weight -1 component is $\Omega^1(B\fg)[-1]$.

The interesting contribution arises from how the vector fields act, both on themselves and on the module. 

\subsubsection{The main diagram} 

\owen{I seem to keep screwing up this part of the computation. I'll draw the main parts of the diagram and then sketch how I'm confused.}

\[
\xymatrix{
1 \otimes \sO_{red}(B\fg) \ar[rd]^{\d_{dR}} & \lambda^0 \otimes \sO_{red}(B\fg) \ar[r]^-2 \ar[rd]^{\d_{dR}} & \lambda^{-1} \wedge \lambda^1 \otimes \sO_{red}(B\fg) & \lambda^{-1} \wedge \lambda^1 \wedge \lambda^0 \otimes \sO_{red}(B\fg) \\
 & \lambda^{-1} \otimes \Omega^1(B\fg) \ar[r]^-1 & \lambda^{-1} \wedge \lambda^0 \otimes\Omega^1(B\fg) &
}
\]

The top row is the part of the total complex coming from the weight 0
by weight 0 terms and the bottom row comes from the weight 1 by weight
-1 terms. Each component has the internal differential coming from the
``geometry'' of $B\fg$. The diagonal terms should be the de Rham
differential on $B\fg$. The horizontal arrows arise from the action of
vector fields on themselves.

The above complex is quasi-isomorphic to
\[
\xymatrix{
1 \otimes \sO_{red}(B\fg) \ar[rd]^{\d_{dR}} & & & \lambda^{-1} \wedge \lambda^1 \wedge \lambda^0 \otimes \sO_{red}(B\fg) \\
 & \lambda^{-1} \otimes \Omega^1(B\fg) & \lambda^{-1} \wedge \lambda^0 \otimes\Omega^1(B\fg) &
}
\]
Keeping track of the degrees this complex is isomorphic to
\[
\Omega^2_{cl}(B \fg)[1] \oplus \Omega^1(B \fg) \oplus \Omega^1_{cl}(B
\fg)[-1] .
\]

\owen{If this diagram is correct (about which I'm dubious), then we're kind of fucked. The complex would be quasi-isomorphic to $\sO_{red}(B\fg) \oplus \sO_{red}(B\fg)[-3]$. I don't think that gives us what we want.}

***************** GARBAGE BELOW HERE ******************

Let's examine first how the vector fields act on themselves. Because $[L_{-1},L_1] = 2 L_0$, we see that the differential sends $\lambda^0$ to $2 \lambda^{-1} \wedge \lambda^1$.  Similarly, we know $[L_{-1},L_0] = L_{-1}$, so that the differential sends $\lambda^{-1}$ to $\lambda^{-1} \wedge \lambda^0$.

Now consider how the vector fields act on the module. We know that
\[
L_{-1} \cdot (y_0 z)y_1 \cdots y_k = i y_0 y_1 \cdots y_k,
\]
where the $y_j$ are elements of $\fg$, so that $(y_0 z)y_1 \cdots y_k$ has weight $1$ and $y_0 y_1 \cdots y_k$ has weight 0. Dualizing, we find that in our total complex,
\[
\d (1 \ot \gamma_0 \cdots \gamma_k) = i \lambda^{-1} \ot \sum \pm (\gamma_j \zeta_0) \gamma_1 \cdots \widehat{\gamma_j} \cdots \gamma_k + \lambda^0 \ot \d_\fg(\gamma_0 \cdots \gamma_k).
\]
Reinterpreting via the geometric language, we are witnessing 
\[
\sO_{red}(B\fg) \overset{\d_{dR}}{\longrightarrow} \Omega^1(B\fg).
\]
We also know that
\[
L_{0} \cdot (y_0 z)y_1 \cdots y_k = i(y_0z) y_1 \cdots y_k.
\]
Hence
\begin{align*}
\d (\lambda^0 \ot \gamma_0 \cdots \gamma_k) = &i \lambda^0 \wedge \lambda^{-1} \ot \sum \pm (\gamma_j \zeta_0) \gamma_1 \cdots \widehat{\gamma_j} \cdots \gamma_k \\
&+ 2 \lambda^{-1} \wedge \lambda^{1} \ot \gamma_0 \cdots \gamma_k \\
&+ \lambda^0 \ot \d_\fg(\gamma_0 \cdots \gamma_k).
\end{align*}

We also know that
\[
[L_{-1},L_0] = L_{-1} \quad\text{and}\quad sdf
\]

\section{An arbitrary source manifold}




\end{document}