\documentclass{amsart}

\usepackage{fullpage,bwog_macros,amssymb,stmaryrd,xypic,color}

%formatting
\setlength{\parindent}{0pt} 
\setlength{\parskip}{2ex}% plus 0.5ex minus 0.2ex}

\def\owen{\textcolor{red}}
\def\brian{\textcolor{blue}}
\def\zbar{\Bar{z}}
\def\dbar{\Bar{\partial}}

\title{Obstruction for flat string}
\begin{document}
\maketitle

Obstruction calculation for the flat string. 
Throughout we work on $\Sigma = \CC$. 
Let $V$ denote the target vector space.

\section{Deformation complex}

The full deformation complex is
\ben
{\rm Def} = \left(\scr{O}_{\rm loc}(\scr{E}_{\rm string}), \Bar{\partial} +
\{I,-\} \right)
\een
There is a subcomplex ${\rm Def}' \subset {\rm Def}$ consisting of
those functionals that only depend on $\scr{E}_{\rm grav}$. 

\begin{prop} The obstruction lies in ${\rm Def}'$. 
\end{prop}

\subsection{Heat Kernels}
\brian{this might fit better in an appendix}
The heat kernel is a degree $1$ element $\mathbb{K}_t \in (\scr{E}_{\rm string})^{\tensor 2}$
that depends smoothly on $t \geq 0$. It splits as 
\ben
\mathbb{K}_t = \left(\mathbb{K}_t^{\beta
      \gamma}\tensor 1 \right)
\oplus \left(1 \tensor \mathbb{K}_t^{\rm grav}\right) \in
\left(\left(\scr{E}_{\beta \gamma}^{\oplus 13}\right)^{\tensor 2} \tensor 1\right) \oplus
\left( 1 \tensor \scr{E}_{\rm grav} \right)
\een
For they single free $\beta\gamma$-theory we have the explicit form
\ben
\mathbb{K}_t^{\beta\gamma} (z,w) = 
\een
\section{Anomaly cancelation}
The machinery of [C1] allows us to define a pre-theory
\ben
I[L] := \lim_{\epsilon \to 0} W(\mathbb{P}_\epsilon^L, I - I^{\rm
  CT}(\epsilon)) .
\een

\begin{lemma} The counterterms vanish. In other words, the limit
\ben
\lim_{\epsilon \to 0} W(\mathbb{P}_\epsilon^L, I) 
\een
exists, and we denote it by $I[L]$. 
\end{lemma}

The obstruction has the form
\ben
\Theta [L] := \dbar I[L] + \{I[L],I[L]\}_L + \Delta_L I[L] .
\een 

\begin{lemma} 
The obstruction is computed by
\ben
\Theta[L] = \lim_{\epsilon} W(\mathbb{P}_{\epsilon}^L,
\mathbb{K}_\epsilon, I)  .
\een
\end{lemma}

\begin{lemma} Only wheels (of valency 3) contribute to the obstruction. Further,
  tadpole and wheels with vertices greater than 2 don't contribute. That
  leaves two wheels of with two vertices each. One of them has inputs
  $\beta \gamma$ the other has inputs vector fields. 
\end{lemma}

Thus, the obstruction is
\bestar
\Theta[L] &=& \lim_{\epsilon \to 0} W_{\gamma_2}(\mathbb{P}_{\epsilon}^L,
\mathbb{K}_\epsilon, I) \\ 
&=& \dim_\CC(V) \cdot \lim_{\epsilon \to 0}  W_{\gamma_2}(\mathbb{P}_{\epsilon}^{\beta\gamma,L},
\mathbb{K}^{\beta\gamma}_\epsilon, I) + \lim_{\epsilon \to 0}
W_{\gamma_2}(\mathbb{P}_{\epsilon}^{{\rm string},L},
\mathbb{K}^{\rm string}_\epsilon, I)  .
\eestar

We already know from the DEFORMATION section 
that these functionals are only nonzero on vector fields. 
Denote by $\Theta^{\beta\gamma}_\epsilon[L] (\xi_0,\xi_1)$ 
the first term, which uses the $\beta \gamma$ edge/propagator on the interior wheel. 
Similarly, denote the second term by
$\Theta^{\rm grav}_\epsilon[L] (\xi_0,\xi_1)$,
which uses the vector field edge/propagator on the interior wheel.

THE SIGN IS PROBABLY FIXED IN THE DOCUMENT "CRITICAL DIM".

We have
\bestar
\Theta^{\beta\gamma}_\epsilon[L] (\xi_0,\xi_1) & = & \int_{(z_0,z_1)
  \in \CC^2} \left(\xi_0(z_0) \cdot
  \mathbb{P}_{\epsilon}^{\beta\gamma,L}(z_0,z_1)\right) \left(\xi_1 \cdot
  \mathbb{K}^{\beta\gamma}_\epsilon(z_0,z_1) \right) .
\eestar

We assume that $\xi_0,\xi_1$ are homogenous. Since the obstruction is
a degree one element it is nonzero provided ${\rm deg} (\xi_0) + {\rm
  deg}(\xi_1) = 1$. For concreteness, suppose
\ben
\xi_0 = f(z,\zbar) \partial_z \;\; , \;\; \xi_1 =
g(z,\zbar) \d \zbar \partial_z .
\een 
Then
\ben
\Theta^{\beta\gamma}_\epsilon[L] (\xi_0,\xi_1) = \int_{(z_0,z_1) \in \CC \times \CC} \left(f(z_0) \partial_{z_0}
  P_\epsilon^L(z_0,z_1)\right) \left(g(z_1) \partial_{z_1}
  K_\epsilon(z_0,z_1)\right) .
\een
After making the change of coordinates this is
\ben
\int_{y_0,y_1} \int_{\epsilon}^L f g \epsilon^{-2}t^{-3} y_0^{3}\exp\left(-\left(\frac{1}{t} +
    \frac{1}{\epsilon}\right) y_0\right)  \d t \d {\rm vol}_{{\bf y}}.
\een
Taylor expanding $fg$ in $y_0$ and performing Wick in the $y_0$-variable this reduces to 
\ben
\int_{y_1} \partial_{y_1}^3 (fg) \d y_1 \d \Bar{y}_1 \int_{\epsilon}^L \epsilon^{-2}
t^{-3} \frac{\epsilon t}{\epsilon + t} \left[ \left(\frac{\epsilon
      t}{\epsilon + t}\right)^3 + {\rm higher\;order\;terms} \right] .
\een
Taking the limit as $\epsilon \to 0$ we see that
\ben
\Theta^{\beta \gamma}[L] = \frac{1}{12}
  \int_{y_1} \partial_{y_1}^3(fg) \d y_1 \d \Bar{y}_1 .
\een

\section{Computing $\Theta^{\rm grav}$}

Note that $\nu \wedge [\xi,\xi] = \nu L_\xi \xi$ where $L_\xi$
denotes Lie derivative. 
\be\label{strob}
\Theta^{\rm grav}_\epsilon[L] (\xi_0,\xi_1) = \int_{(z_0,z_1)
  \in \CC^2} \left(L_{\xi_0} 
  \mathbb{P}_{\epsilon}^{\beta\gamma,L}(z_0,z_1)\right) \left(L_{\xi_1} \cdot
  \mathbb{K}^{\beta\gamma}_\epsilon(z_0,z_1) \right) .
\ee

Again, we assume that $\xi_0,\xi_1$ are homogenous and we must have ${\rm deg} (\xi_0) + {\rm
  deg}(\xi_1) = 1$. For concreteness, suppose
\ben
\xi_0 = f(z,\zbar) \partial_z \;\; , \;\; \xi_1 =
g(z,\zbar) \d \zbar \partial_z .
\een 
Expanding out, the integral (\ref{strob}) splits into four parts

\begin{eqnarray}
\Theta^{\rm grav}_\epsilon[L] (\xi_0,\xi_1) & = & \int_{(z_0,z_1) \in \CC \times \CC} f(z_0) [\partial_{z_0}
  P_\epsilon^L(z_0,z_1)]g(z_1) [\partial_{z_1}
  K_\epsilon(z_0,z_1)]\\   
  & - & \int_{(z_0,z_1) \in \CC \times \CC} [\partial_{z_0}f(z_0)]
  P_\epsilon^L(z_0,z_1) g(z_1) [\partial_{z_1}
  K_\epsilon(z_0,z_1)] \\ &- & \int_{(z_0,z_1) \in \CC \times \CC} f(z_0) [\partial_{z_0}
  P_\epsilon^L(z_0,z_1)] [\partial_{z_1}  g(z_1)] 
  K_\epsilon(z_0,z_1) \\   
  & + &\int_{(z_0,z_1) \in \CC \times \CC} [\partial_{z_0} f(z_0)] 
  P_\epsilon^L(z_0,z_1) [\partial_{z_1}  g(z_1) ]
  K_\epsilon(z_0,z_1).
\end{eqnarray} 

We make the change of coordinates
$y_0 = z_0 - z_1$, $y_1 = z_1$. Before evaluating the obstruction we compute the three basic integrals. 
\begin{itemize} 
\item[(I)] First consider 
\ben
\int_{(z_0,z_1) \in \CC \times \CC} \left(f(z_0) \partial_{z_0}
  P_\epsilon^L(z_0,z_1)\right) \left(g(z_1) \partial_{z_1}
  K_\epsilon(z_0,z_1)\right) .
\een
After making the change of coordinates this is
\ben
\int_{y_0,y_1} \int_{\epsilon}^L f g \epsilon^{-2}t^{-3} y_0^{3}\exp\left(-\left(\frac{1}{t} +
    \frac{1}{\epsilon}\right) y_0\right)  \d t \d {\rm vol}_{{\bf y}}.
\een
Taylor expanding $fg$ in $y_0$ and performing Wick in the $y_0$-variable this reduces to 
\ben
\int_{y_1} \partial_{y_1}^3 (fg) \d y_1 \d \Bar{y}_1 \int_{\epsilon}^L \epsilon^{-2}
t^{-3} \frac{\epsilon t}{\epsilon + t} \left[ \left(\frac{\epsilon
      t}{\epsilon + t}\right)^3 + {\rm higher\;order\;terms} \right] .
\een
%In the limit as $\epsilon \to 0$ only the first term contributes and
%is easily seen to be finite and gives
%\ben
%\frac{1}{12} \int_{y_1} \partial_{y_1}^3 (fg) \d y_1 \d \Bar{y}_1 .
%\een
\item[(II)] Next we evaluate
\ben
\int_{(z_0,z_1) \in \CC \times \CC} \left(f(z_0) \partial_{z_0} \partial_{z_1}
  P_\epsilon^L(z_0,z_1)\right) \left(g(z_1)
  K_\epsilon(z_0,z_1)\right) = \int_{z_0,z_1} fg
(\partial_{z_0} \partial_{z_1} P K) .
\een 
After changing coordinates and Wick expanding as above this has the
form
\ben
\int_{y_1} \partial_{y_1}^3 (fg) \d y_1 \d \Bar{y}_1 \int_{\epsilon}^L \epsilon^{-1}
t^{-4} \frac{\epsilon t}{\epsilon + t} \left[ \left(\frac{\epsilon
      t}{\epsilon + t}\right)^3 + {\rm higher\;order\;terms} \right] .
\een
%Only the first term contributes in the limit and gives
%\ben
%\frac{1}{24}\int_{y_1} \partial_{y_1}^3 (fg) \d y_1 \d \Bar{y}_1 .
%\een

\item[(III)] Finally we evaluate 
\ben
\int_{(z_0,z_1) \in \CC \times \CC} \left(f(z_0) 
  P_\epsilon^L(z_0,z_1)\right) \left(g(z_1)
  \partial_{z_0} \partial_{z_1}  K_\epsilon(z_0,z_1)\right) = \int_{z_0,z_1} fg
(P \partial_{z_0} \partial_{z_1}  K) .
\een 
Changing coordinates and Wick expanding this reduces to
\ben
\int_{y_1} \partial_{y_1}^3 (fg) \d y_1 \d \Bar{y}_1 \int_{\epsilon}^L \epsilon^{-3}
t^{-2} \frac{\epsilon t}{\epsilon + t} \left[ \left(\frac{\epsilon
      t}{\epsilon + t}\right)^3 + {\rm higher\;order\;terms} \right]
\een
\end{itemize}

Since only the lowest order terms in the above integrals contribute in
the limit as $\epsilon \to 0$ we will forget about the higher order
terms in the following calculation. Consider the first integral in the
second line above. We integrate by parts to get
\ben
\int (\partial_{z_0} f P) (g \partial_{z_1} K) = - \left(\int fg \partial_0
  P \partial_{z_1} K + \int fg P \partial_{z_0} \partial_{z_1}
  K\right) .
\een
Integration by parts applied to the second line gives
\ben
\int (f \partial_{z_0} P)(\partial_{z_1} g K) = -\left(\int
  fg \partial_{z_0} \partial_{z_1} P K + \int fg \partial_{z_0}
  P \partial_{z_1} K \right) .
\een
Finally, we can integrate by parts twice for the term in the last line
to get
\bestar
\int (\partial_{z_0} f P)(\partial_{z_1} g K) & = & - \left(\int
  f \partial_{z_0} P (\partial_{z_1} g K) + \int (fP)(\partial_{z_1}
  g \partial_{z_0} K) \right) \\ & = & 2 \int fg \partial_{z_0}
P \partial_{z_1} K + \int fg \partial_{z_0}\partial_{z_1} P + \int fg
P \partial_{z_0} \partial_{z_1} K .
\eestar

Thus, we see the obstruction has the form
\bestar
\Theta^{\rm grav}_\epsilon[L] (\xi_0,\xi_1) & = & 5 {\rm I} + 2 ( {\rm
  II} + {\rm III})  + \left[{\rm higher\;\epsilon\;terms}\right] \\ & = &
\int_{y_1} \partial_{y_1}^3 (fg) \d y_1 \d \Bar{y}_1 \int_{t =
  \epsilon}^L \frac{\epsilon}{(\epsilon + t)^4} \left(\epsilon t + 2
  (2\epsilon t + \epsilon^2 + t^2) \right) \d t  + \left[{\rm higher\;\epsilon\;terms}\right]\\  & = & \int_{y_1} \partial_{y_1}^3 (fg) \d y_1 \d \Bar{y}_1 \int_{t =
  \epsilon}^L \frac{\epsilon}{(\epsilon + t)^4} \left(\epsilon t + 2(\epsilon + t)^2 \right) \d t + \left[{\rm higher\;\epsilon\;terms}\right] .
\eestar
In the limit as $\epsilon \to 0$ we get
\ben
\Theta^{\rm grav}[L] (\xi_0,\xi_1) = \int_{y_1} \partial_{y_1}^3 (fg) \d y_1 \d \Bar{y}_1
\left(\frac{1}{12} + 1 \right) = \frac{13}{12}
\int_{y_1} \partial_{y_1}^3 (fg) \d y_1 \d \Bar{y}_1 .
\een


\end{document}

