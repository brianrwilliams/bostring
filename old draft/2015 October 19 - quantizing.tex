\documentclass[10pt]{amsart}

\usepackage{macros}

\def\brian{\textcolor{blue}}
\def\owen{\textcolor{red}}

\title{Quantizing the holomorphic bosonic string}
\begin{document}
\maketitle

\section{Obstruction deformation complex}
\brian{compute obstruction groups. Symmetry arguments to argue relevant
  terms.}

The role of $\fg_X$ is bookkeeping, so we will denote it by $\fg$ for
notational simplicity. As usual, we will use the coordinate $z$ for
the source $\CC$, and let $\partial_z$ denote the holomorpic vector field $\partial / \partial
z$. 

\subsection{Jets}

The role of $\fg_X$ here is bookkeeping, so we denote it by $\fg$ for
now. We use the coordinate $z$ on $\CC$, and we let $\partial_z$
denote $\frac{\partial}{\partial z}$. 

The Lie algebra $\fv = \CC \llbracket z \rrbracket \partial_z$ has a
natural basis $L_n := i z^{n+1} \partial_z$, with $n \geq -1$. With
respect to this basis we have
\[
[L_m,L_n] = (m-n) L_{m+n} .
\]
Thus, the $L_0$ element induces a natural filtration on $\fv$: (1) $z^k$
has weight $k$, (2) $\partial_z$ has weight $-1$. 


\section{Calculating the obstructions}
\subsection{The ``string'' condition}
\brian{just Kevin's result restated. No vector fields involved.}
$ch_2(T_X)$
\subsection{The ``Calabi-Yau'' condition}
$ch_1(T_X)$. \brian{This looks like it involves one input from the
  target and one input from vector fields. Need to say why this lands
  in the right piece of the deformation complex.}

\subsection{The conformal anomaly} 
\brian{vector fields inputs only, two diagrams must cancel eachother.
}
\section{Main result}
\brian{existence of quantization}


\end{document}