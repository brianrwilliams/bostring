\documentclass[10pt]{amsart}

\usepackage{macros}

\def\brian{\textcolor{blue}}
\def\owen{\textcolor{red}}

\title{BV-quantization}
\begin{document}
\maketitle

\brian{I'm going to start using $\sE, \sL$ for fields and $\cE, \cL$
  for Lie algebras. Thus $\sE = \cE[1]$, etc.. }

\section{Renormalization}
\subsection{Generalities on renormalization}
\brian{existence of counterterms, effective theories, gauge fixing,
  $\beta \gamma$ examples etc.}
\subsection{Effective propagator}
There is a natural choice for the gauge fixing operator for our
theory. Namely, 
\ben
Q_{GF} := \Bar{\partial}^* : \sE \to \sE
\een
defined using the flat metrix on $\CC$ coming from the holomorphic
volume element $\d z$. 
\begin{lemma} Let $Q$ be the total differential for fields
  $\sE$. Then
\ben
D = [\dbar^*, Q] : \sE \to \sE
\een 
is the usual Laplacian acting on the appropriate Dolbeault complex. 
\end{lemma}

The heat kernel is the element $\mathbb{K}_t \in {\rm Sym}^2(\sE)$ defined
for $t >0$ determined by the equation
\ben
\<\mathbb{K}_t(z,w), \varphi(w)\> = (e^{-t D} \varphi) (z)
\een
for all $\varphi \in \sE$. 

For us, the fields split as $\sE = \sE_{\beta \gamma} \oplus
\sE_{\rm grav}$. 

\begin{lemma} The heat kernel is of the form
\ben
\mathbb{K}_t = \mathbb{K}_t^{\beta \gamma} \oplus \mathbb{K}_t^{\rm
  grav} \in {\rm Sym}^2(\sE_{\beta \gamma}) \oplus {\rm
  Sym}^2(\sE_{\rm grav})
\een
where 
\brian{write down kernels}
\end{lemma}

\subsection{Pre-theory}
\brian{naive quantization works, i.e. no counterterms. Reduction to
  wheels, reduction to trivalent wheels.}


\section{The quantum master equation}
\subsection{Obstruction theory}
\brian{introduce deformation complex for bosonic string.}
By definition, holomorphic string is described by the local curved
$L_\infty$-algebra 
\ben
\mathbb{D}_S (\cL \ltimes \cE) =  .
\een 
Thus, the full obstruction-deformation complex is
\ben
{\rm Def}_X := C^*_{\rm loc}(\mathbb{D}_S (\cL \ltimes \cE)) .
\een 
\subsection{Symmetries}
\brian{symmetries to consider for the deformation complex. Cotangent
  quantization. ${\rm Aff}(\CC)$-symmetry. }



\end{document}
