\section{OPE and the string vertex algebra}

Vertex algebras are mathematical objects that axiomatize the behavior of local observables 
(i.e., point-like observables) of a chiral conformal field theory (CFT),
such as the $bc\beta\gamma$ system or the holomorphic bosonic string.
The vertex operator of a vertex algebras encodes the operator product expansions (OPE) for local observables,
which is of central interest in understanding a chiral CFT.
(We will not review vertex algebras here
as there are many nice expositions \owen{cite some}.)

In this section we will explain how to extract the vertex algebra of the holomorphic bosonic string,
using machinery developed in~\cite{CG1,LiVA,CDO}.
The answer we recover is precisely the chiral sector of the usual bosonic string.
\owen{Is that true?}

\subsection{A reminder on the chiral algebra of the string}\label{subsec: string vert}

We provide a brief background on the vertex algebra for the chiral sector of the bosonic string. 
For a detailed reference we refer the reader to the series of papers \cite{LZ1,LZ2}. 
It is easiest to introduce this as a {\em differential graded vertex algebra}. 
This is simply a vertex algebra internal to the category of chain complexes. 
The underlying graded vertex algebra has state space of the form
\ben
\cV_{\beta \gamma}^{\tensor 13} \tensor \cV_{bc}
\een
where $\cV_{\beta\gamma}$ and $\cV_{bc}$ are the $\beta\gamma$ and $bc$ vertex algebras, respectively. 
The $\beta$ and $\gamma$ generators are in grading degree zero, the $c$ generator is in grading degree $-1$, and the $b$ is in grading degree $+1$. 
In the physics literature this is referred to as the {\em BRST} grading.

Forgetting the cohomological (or BRST) grading, this vertex algebra is a conformal vertex algebra of central charge zero (by construction). 
In particular, this means that the vertex algebra has a stress energy tensor. 
Explicitly, it is of the form
\ben
T_{\rm string} (z) = \left(\sum_{i = 1}^{13} \beta_i (z) \partial_z \gamma_i (z) + \partial_z \beta_i(z) \gamma_i (z) \right) + \left(b(z) \partial_z c(z) + 2 \partial_z b(z) c(z) \right) . 
\een
Note that $T_{\rm string}$ is of cohomological degree zero. 
The first parenthesis is interpreted as the stress energy tensor of the vertex algebra $\cV_{\beta \gamma}^{\tensor 13}$ and the second term is the stress energy tensor of $\cV_{bc}$. 

We have not yet described the differential on the graded vertex algebra. 
The BRST differential is defined to be the vertex algebra derivation obtained by taking the following residue
\ben
Q^{BRST} = \oint c(z) T_{\rm string}(z) .
\een 
By construction this operator satisfies $(Q^{BRST})^2 = 0$. 

\begin{dfn} The {\em string vertex algebra} is the dg vertex algebra 
\ben 
\cV_{\rm string} = \left(\cV_{\beta \gamma}^{\tensor 13} \tensor \cV_{bc}, \; Q^{BRST}\right)  .
\een
\end{dfn}

There is another grading on $\cV_{\rm string}$ coming from the eigenvalues of the vertex algebra derivation $c_0$ called the {\em conformal weight}. 
In particular, this determines a filtration and we can consider the associated graded ${\rm Gr} \; \cV_{\rm string}$. 
The conformal weight grading preserves the cohomological grading so that this object still has the structure of a dg vertex algebra. 

Note that the cohomology of a dg vertex algebra is an ordinary (graded) vertex algebra. 
The cohomology of the string vertex algebra is called the {\em BRST cohomology} of the bosonic string. 
In the remainder of this section we will show how we recover the string vertex algebra from the quantization of the holomorphic bosonic string.

\subsection{Some context}

In the BV formalism one constructs a cochain complex of observables,
for both the classical and the quantized theory, if it exists.
The cochain complexes are local on the source manifold of a theory:
on each open set $U$ in that manifold~$\Sigma$,
one can pick out the observables with support in~$U$ by asking for the observables that vanish on fields with support outside~$U$.
It is the central result of~\cite{CG1,CG2} that the observables also satisfy a local-to-global property,
akin to the sheaf gluing axiom,
and hence form a {\em factorization algebra} on~$\Sigma$.

We will not need that general notion here.
Instead, we will use vertex algebras.
Theorem~\owen{???} of~\cite{CG1} explains how a factorization algebra~$F$ on $\Sigma = \CC$
yields a vertex algebra~$\Vert(F)$, under natural hypotheses on~$F$. 
It assures us that the observables of a chiral CFT determine a vertex algebra.

In particular, Section~\owen{???} of~\cite{CG1} examines the free $\beta\gamma$ system in great detail.
Its main result is that the well-known $\beta\gamma$ vertex algebra is recovered by the two-step process of BV quantization, which yields a factorization algebra, and then the extraction of a vertex algebra.

The exact same arguments apply to the free $bc\beta\gamma$ system,
where the $\beta\gamma$ sector is valued in a vector space $V$, as we introduced in Section \ref{sec: freebgbc}
Let $\Obs^\q_{free}$ denote the observables of this theory on $\Sigma = \CC$.
As a quantization of a free field theory, it is a factorization algebra valued in the category of $\CC [\hbar]$-modules.
In particular, the associated vertex algebra $\Vert(\Obs^\q_{free})$ is also valued in $\CC[\hbar]$-modules.

\begin{prop}
Let $n = \dim_{\CC}(V)$. Then, there is an isomorphism of vertex algebras
\ben
\Vert(\Obs^{\q}_{free})_{\hbar = 2 \pi i} \cong \cV_{bc} \tensor \cV_{\beta\gamma}^{\tensor n} 
\een 
where on the left-hand side we have set $\hbar = 2\pi i$.
\end{prop}

\subsection{The case of the string}

The holomorphic bosonic string is a chiral CFT and so the machinery of~\cite{CG1} applies to it.
One can extract a vertex algebra directly by this method.

But there is a slicker approach, using Li's work~\cite{Li},
which studies chiral deformations of {\em free} chiral BV theories such as the $bc\beta\gamma$ system.
Recall, a deformation of a classical field theory is given by a local functional. 
We have seen that this is essentially the data of a Lagrangian density, which is a density valued multilinear functional that depends on (arbitrarily high order) jets of the fields. 
In other words, for a field $\varphi$, a Lagrangian density is of the form
\ben
\cL(\varphi) = \sum (D_{k_1} \varphi) \cdots (D_{k_m} \varphi) \cdot {\rm vol}_\Sigma
\een 
for $C^\infty(\Sigma)$-valued differential operators $D_{k_i}$.
By a {\em chiral} Lagrangian density we mean a Lagrangian for which the differential operators $D_{k_i}$ are all holomorphic. 
For instance, on $\Sigma = \CC$, we require $D_{k_i}$ to be a sum of operators of the form $f(z) \partial_z^n$ where $f(z)$ is a holomorphic function. 
On $\Sigma = \CC$ we will also require the chiral Lagrangian to be translation invariant. 
This means that all differential operators $D_{k_i}$ are of the form $\partial_z^n$. 
Thus, a {\em translation invariant chiral deformation} is a local functional of the form
\ben
I(\varphi) = \sum \int (\partial^{k_1}_z \varphi) \cdots (\partial^{k_m} \varphi) \d^2 z .
\een

One of Li's main results is that for a free chiral BV theory with action $S_{\rm free}$ and associated vertex algebra $\cV_{\rm free}$, one has the following:
\owen{Check that this is an accurate summary of his work}
\begin{itemize}
\item For any chiral interaction~$I$, the action $S_{\rm free} + I$ needs no counterterms, 
and yields a renormalized interaction~$\{I [L]\}$.
\item If the renormalized action $\{I[L]\}$ satisfies the quantum master equation,
then it determines a vertex algebra derivation $D_I$ of~$\cV_{\rm free}$ of the form
\ben
D_I = \oint I^q \d z
\een
where $I^q = \lim_{L \to 0} I[L]$.
\item \owen{The vertex algebra $\cV_I$ for such an action $\{I[L]\}$ is isomorphic to the vertex algebra~$\cV_{\rm free}$ via the automorphism~$\exp(D_I)$}. \brian{Not really, right? Deformations are Maurer-Cartan elements in degree 0 (or 0 depending on convention) and automorphisms are of degree -1 (or 0 depending on convention) Here's what I think is true:}.
\brian{EDIT: I checked with Si, and he says that the below assertion is accurate.}
\item The dg vertex algebra $\cV_I$ for such an action $\{I[L]\}$ has the same underlying graded vertex algebra $\cV_{\rm free}$ but it is equipped with the differential $\oint I^q \d z$. 
\end{itemize}

The holomorphic bosonic string with target $V=\CC^{13}$ provides a concrete example of this situation.
The free theory is the $bc\beta\gamma$ system, 
and we have seen that the renormalized action satisfies the QME.
Hence we obtain the following.

\begin{prop} 
Let $\Obs^q_{\rm string}$ be the factorization algebra on $\Sigma = \CC$ of the holomorphic bosonic string with target~$V = \CC^{13}$. 
There is an isomorphism of vertex algebras $\cV_{\rm string} \cong \Vert(\Obs^q_{\rm string})_{\hbar = 2 \pi i}$.
Moreover, this vertex algebra is isomorphic to the chiral sector of the bosonic string as in Section~\ref{subsec: string vert}.
\end{prop}

The factorization algebra $\Obs^\q_{\rm string}$ is a quantization of the factorization algebra $\Obs^{\cl}_{\rm string}$ of classical observables of the free $bc\beta\gamma$ system.
We have noted that the classical observables of any theory has the structure of a $P_0$ factorization algebra, and the $\hbar \to 0$ limit of $\Obs^\q_{\rm string}$ is isomorphic to $\Obs^{\cl}_{\rm string}$ as $P_0$ factorization algebras. 
In this case, the classical factorization algebra $\Obs^{\cl}_{\rm string}$ still has enough structure to determine a vertex algebra $\Vert(\Obs^\cl_{\rm string})$.
Moreover, the $P_0$ bracket on the classical observables determines the structure of a {\em Poisson vertex algebra} on $\Vert(\Obs^{\cl}_{\rm string})$. 

\begin{cor} In the classical limit, there is an isomorphism of Poisson vertex algebras~$\Vert(\Obs^{\cl}_{\rm string}) \cong {\rm Gr} \; \cV_{\rm string}$.
\end{cor}

\owen{describe classical observables \& why we get a Poisson vertex algebra}
\brian{Should I say more?}

\brian{Write down vertex algebra from quantization above. Possibly state the relationship to semi-infinite cohomology}

\subsection{The $E_2$ algebra}

We continue to consider the theory on the Riemann surface $\Sigma = \CC$. 
In this section we show how to produce, from the point of view of factorization algebras, the structure of a Gerstenhaber algebra on the BRST cohomology of the bosonic string. 
A Gerstenhaber algebra is equivalent to an algebra over the homology of the frame little 2-disk operad. 
It is a well-known result of Lurie \cite{Lurie} that a {\em locally constant} factorization algebra on $\RR^n$ is equivalent to an algebra over the little $n$-disks operad, or an $E_n$-algebra. 
We will show that the cohomology of the factorization algebra $\Obs^{\q}_{\rm string}$ on $\CC \cong \RR^2$ has the structure of a Gerstenhaber algebra, which implies that $\Obs^\q_{\rm string}$ is equivalent to an $E_2$-algebra. 

Another occurrence of $E_n$-algebras is as the observables of topological field theories in (real) dimension $n$. 
At this level, this implies that the theory of the holomorphic bosonic string is equivalent to a topological field theory. There is a more direct link to topological field theories as defined in \cite{wtop} that we recall here. 
There is an action of the differential operators $\frac{\d}{\d z}$ and $\frac{\d}{\d \zbar}$ on the Dolbeault complex $\Omega^{0,*}(\CC)$. 
This extends to an action of the differential operators to the holomorphic bosonic string, and hence to the observables as well. 
By Noether's theorem the symmetry of the theory determined by these operators define observables: these are simply the $zz$ and $\zbar \zbar$ components of the stress energy tensor $T_{zz}$, $T_{\zbar \zbar}$. 
We will see that in the case of the string the stress energy tensors are cohomologically trivial. 
This is the general definition of a topological theory given in \cite{top}.

\brian{Am I oversimplifying above?}

\subsubsection{}

We consider the factorization algebra $\Obs^\q_{\rm string}$. 
For each open $U \subset \CC$ we can define the operators
\ben
\frac{\d}{\d z} , \frac{\d}{\d \zbar} : \Obs^\q_{\rm string} (U) \to \Obs^\q_{\rm string}(U) .
\een 
In fact, these operators define {\em derivations} of the factorization algebra, in the sense that they are compatible with the factorization product. 
Note that these operators preserve the cohomological degree. 

Consider the operator acting on Dolbeault forms
\ben
\Bar{\eta} = \frac{\partial}{\partial (\d \zbar)} .
\een 
This extends to a derivation of degree $-1$ on the factorization algebra $\Obs^\q_{\rm string}$. 
Moreover, we have the following relation
\ben
[\dbar + \hbar \Delta + \{I^\q, -\} , \Bar{\eta}] =  \frac{\d}{\d \zbar}
\een 
as endomorphisms of the factorization algebra. 
This means that the operator $\frac{\d}{\d \zbar}$ acts cohomologically trivial on $\Obs^\q_{\rm string}$. 

Recall that the $b$-fields of the bosonic string are concentrated in degree $+1$ and $+2$. 
A field in degree $+1$ is of the form $f(z,\zbar) \d z^{\tensor 2} \in \Omega^{1,0}(\Sigma, T_\Sigma^{1,0*})$. 
Define the following linear observable on $\CC$ that vanishes on all fields besides the degree $+1$ component of the $b$-fields:
\ben
b_{-1} : f(z,\zbar) \d z^{\tensor 2} \mapsto f(0) . 
\een 
This observable is a closed element of degree $-1$ in $\Obs^\q_{\rm string}$. 
Given any other observable $O \in \Obs^{\q}_{\rm string}$ we can define the observable $b_{-1} \cdot O$ using the symmetric product.
Denote the derivation $\eta : O \mapsto b_{-1} \cdot O$

\begin{lem} The derivation $\eta$ is degree $-1$ and satisfies 
\ben
[\dbar + \hbar \Delta + \{I^\q, -\}, \eta] = \frac{\d}{\d z}
\een
\end{lem}
\begin{proof} \brian{do this}
\end{proof}
This shows that the derivation $\frac{\d}{\d z}$ acts trivial up to homotopy on the factorization algebra. 
