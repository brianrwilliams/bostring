\section{OPE and the string vertex algebra}

Vertex algebras are mathematical objects that axiomatize the behavior of local observables 
(i.e., point-like observables) of a chiral conformal field theory (CFT),
such as the $bc\beta\gamma$ system or the holomorphic bosonic string.
The vertex operator of a vertex algebras encodes the operator product expansions (OPE) for local observables,
which is of central interest in understanding a chiral CFT.
(We will not review vertex algebras here
as there are many nice expositions \owen{cite some}.)

In this section we will explain how to extract the vertex algebra of the holomorphic bosonic string,
using machinery developed in~\cite{CG1,LiVA,CDO}.
The answer we recover is precisely the chiral sector of the usual bosonic string.
\owen{Is that true?}

\subsection{A reminder on the chiral algebra of the string}\label{subsec: string vert}

We provide a brief background on the vertex algebra for the chiral sector of the bosonic string. 
For a detailed reference we refer the reader to the series of papers \cite{LZ1,LZ2}. 
It is easiest to introduce this as a {\em differential graded vertex algebra}. 
This is simply a vertex algebra internal to the category of chain complexes. 
The underlying graded vertex algebra has state space of the form
\ben
\cV_{\beta \gamma}^{\tensor 13} \tensor \cV_{bc}
\een
where $\cV_{\beta\gamma}$ and $\cV_{bc}$ are the $\beta\gamma$ and $bc$ vertex algebras, respectively. 
The $\beta$ and $\gamma$ generators are in grading degree zero, the $c$ generator is in grading degree $-1$, and the $b$ is in grading degree $+1$. 
In the physics literature this is referred to as the {\em BRST} grading.

Forgetting the cohomological (or BRST) grading, this vertex algebra is a conformal vertex algebra of central charge zero (by construction). 
In particular, this means that the vertex algebra has a stress energy tensor. 
Explicitly, it is of the form
\ben
T_{\rm string} (z) = \left(\sum_{i = 1}^{13} \beta_i (z) \partial_z \gamma_i (z) + \partial_z \beta_i(z) \gamma_i (z) \right) + \left(b(z) \partial_z c(z) + 2 \partial_z b(z) c(z) \right) . 
\een
Note that $T_{\rm string}$ is of cohomological degree zero. 
The first parenthesis is interpreted as the stress energy tensor of the vertex algebra $\cV_{\beta \gamma}^{\tensor 13}$ and the second term is the stress energy tensor of $\cV_{bc}$. 

We have not yet described the differential on the graded vertex algebra. 
The BRST differential is defined to be the vertex algebra derivation obtained by taking the following residue
\be\label{brst}
Q^{BRST} = \oint c(z) T_{\rm string}(z) .
\ee
By construction this operator satisfies $(Q^{BRST})^2 = 0$. 

\begin{dfn} The {\em string vertex algebra} is the dg vertex algebra 
\ben 
\cV_{\rm string} = \left(\cV_{\beta \gamma}^{\tensor 13} \tensor \cV_{bc}, \; Q^{BRST}\right)  .
\een
\end{dfn}

There is another grading on $\cV_{\rm string}$ coming from the eigenvalues of the vertex algebra derivation $c_0$ called the {\em conformal dimension}. 
In particular, this determines a filtration and we can consider the associated graded ${\rm Gr} \; \cV_{\rm string}$. 
The conformal weight grading preserves the cohomological grading so that this object still has the structure of a dg vertex algebra. 

Note that the cohomology of a dg vertex algebra is an ordinary (graded) vertex algebra. 
The cohomology of the string vertex algebra is called the {\em BRST cohomology} of the bosonic string. 
In the remainder of this section we will show how we recover the string vertex algebra from the quantization of the holomorphic bosonic string.

\subsection{Some context}

In the BV formalism one constructs a cochain complex of observables,
for both the classical and the quantized theory, if it exists.
The cochain complexes are local on the source manifold of a theory:
on each open set $U$ in that manifold~$\Sigma$,
one can pick out the observables with support in~$U$ by asking for the observables that vanish on fields with support outside~$U$.
It is the central result of~\cite{CG1,CG2} that the observables also satisfy a local-to-global property,
akin to the sheaf gluing axiom,
and hence form a {\em factorization algebra} on~$\Sigma$.

We will not need that general notion here.
Instead, we will use vertex algebras.
Theorem~5.2.3.1 of~\cite{CG1} explains how a factorization algebra~$F$ on $\Sigma = \CC$
yields a vertex algebra~$\Vert(F)$, under natural hypotheses on~$F$. 
It assures us that the observables of a chiral CFT determine a vertex algebra.

In particular, Section~5.3 of~\cite{CG1} examines the free $\beta\gamma$ system in great detail.
Its main result is that the well-known $\beta\gamma$ vertex algebra is recovered by the two-step process of BV quantization, which yields a factorization algebra, and then the extraction of a vertex algebra.

The exact same arguments apply to the free $bc\beta\gamma$ system,
where the $\beta\gamma$ sector is valued in a vector space $V$, as we introduced in Section \ref{sec: freebgbc}
Let $\Obs^\q_{free}$ denote the observables of this theory on $\Sigma = \CC$.
As a quantization of a free field theory, it is a factorization algebra valued in the category of $\CC [\hbar]$-modules.
In particular, the associated vertex algebra $\Vert(\Obs^\q_{free})$ is also valued in $\CC[\hbar]$-modules.

\begin{prop}\label{prop: bcbg vertex}
Let $n = \dim_{\CC}(V)$. Then, there is an isomorphism of vertex algebras
\ben
\Vert(\Obs^{\q}_{free})_{\hbar = 2 \pi i} \cong \cV_{bc} \tensor \cV_{\beta\gamma}^{\tensor n} 
\een 
where on the left-hand side we have set $\hbar = 2\pi i$.
\end{prop}

\subsection{The case of the string}

The holomorphic bosonic string is a chiral CFT and so the machinery of~\cite{CG1} applies to it.
One can extract a vertex algebra directly by this method.

But there is a slicker approach, using Li's work~\cite{Li},
which studies chiral deformations of {\em free} chiral BV theories such as the $bc\beta\gamma$ system.
Recall that a deformation of a classical field theory is given by a local functional. 
We have seen that this is essentially the data of a Lagrangian density, which is a density valued multilinear functional that depends on (arbitrarily high order) jets of the fields. 
In other words, for a field $\varphi$, a Lagrangian density is of the form
\ben
\cL(\varphi) = \sum (D_{k_1} \varphi) \cdots (D_{k_m} \varphi) \cdot {\rm vol}_\Sigma
\een 
for $C^\infty(\Sigma)$-valued differential operators $D_{k_i}$.
By a {\em chiral} Lagrangian density we mean a Lagrangian for which the differential operators $D_{k_i}$ are all holomorphic. 
For instance, on $\Sigma = \CC$, we require $D_{k_i}$ to be a sum of operators of the form $f(z) \partial_z^n$ where $f(z)$ is a holomorphic function. 
On $\Sigma = \CC$ we will also require the chiral Lagrangian to be translation invariant. 
This means that all differential operators $D_{k_i}$ are of the form $\partial_z^n$. 
Thus, a {\em translation-invariant chiral deformation} is a local functional of the form
\ben
I(\varphi) = \sum \int (\partial^{k_1}_z \varphi) \cdots (\partial^{k_m} \varphi) \d^2 z .
\een

One of Li's main results is that for a free chiral BV theory with action $S_{\rm free}$ and associated vertex algebra $\cV_{\rm free}$, one has the following:
\begin{itemize}
\item For any chiral interaction~$I$, the action $S_{\rm free} + I$ needs no counterterms, 
and yields a renormalized interaction~$\{S [L]\}$.
\item If the renormalized action $\{S [L]\}$ satisfies the quantum master equation,
then it determines a vertex algebra derivation $D_I$ of~$\cV_{\rm free}$ of the form
\ben
D_I = \oint I^q \d z
\een
of cohomological degree one, where $I^q = \lim_{L \to 0} I[L]$.
\item The dg vertex algebra $\cV_I$ for such an action $\{I[L]\}$ has the same underlying graded vertex algebra $\cV_{\rm free}$ but it is equipped with the differential $\oint I^q \d z$. 
\end{itemize}

\begin{rmk} The fact that $I$ satisfies the quantum master equation implies that one has a map, for each open set $U \subset \CC$, from the free factorization algebra evaluated on $U$ to the factorization algebra of the deformed theory evaluated on $U$:
\ben
e^{I /\hbar} : \Obs^q_{free}(U) \to \Obs^q_I (U) .
\een
This map sends an observable $O \in \Obs^q_{free}(U)$ to $O \cdot e^{I/\hbar}$. 
In fact, this map is an isomorphism with inverse given by $O \mapsto O \cdot e^{-I/\hbar}$. 
So, open by open, the factorization algebra assigns the same vector space for the deformed theory.
This isomorphism is {\em not} compatible with the factorization product, so we do get a different factorization algebra in the presence of a deformation.
\end{rmk}

The holomorphic bosonic string with target $V=\CC^{13}$ provides a concrete example of this situation.
The free theory is the $bc\beta\gamma$ system, 
and we have seen that the renormalized action satisfies the QME.
Hence we obtain the following.

\begin{prop} \label{prop: fact is vert}
Let $\Obs^\q_{\rm string}$ be the factorization algebra on $\Sigma = \CC$ of the holomorphic bosonic string with target~$V = \CC^{13}$. And let $\Vert(\Obs^q_{\rm string})$ be the dg vertex algebra (defined over $\CC[\hbar]$) obtained via Li's construction. 
There is an isomorphism of vertex algebras $\cV_{\rm string} \cong \Vert(\Obs^q_{\rm string})_{\hbar = 2 \pi i}$.
Moreover, this vertex algebra is isomorphic to the chiral sector of the bosonic string as in Section~\ref{subsec: string vert}.
\end{prop}

The factorization algebra $\Obs^\q_{\rm string}$ is a quantization of the factorization algebra $\Obs^{\cl}_{\rm string}$ of classical observables of the free $bc\beta\gamma$ system.
We have noted that the classical observables of any theory has the structure of a $P_0$ factorization algebra, and the $\hbar \to 0$ limit of $\Obs^\q_{\rm string}$ is isomorphic to $\Obs^{\cl}_{\rm string}$ as $P_0$ factorization algebras.
By definition, the classical observables are simply functions on the solutions to the classical equations of motion.
The $P_0$ structure is induced from the symplectic pairing of degree $(-1)$ on the fields. 
The classical factorization algebra still has enough structure to determine a vertex algebra $\Vert(\Obs^\cl_{\rm string})$.
Moreover, the $P_0$ bracket on the classical observables determines the structure of a {\em Poisson vertex algebra} on $\Vert(\Obs^{\cl}_{\rm string})$. 

\begin{cor} In the classical limit, there is an isomorphism of Poisson vertex algebras~$\Vert(\Obs^{\cl}_{\rm string}) \cong {\rm Gr} \; \cV_{\rm string}$.
\end{cor}

\begin{proof}[Proof of Proposition \ref{prop: fact is vert}] By Proposition \ref{prop: bcbg vertex} we know that the vertex algebra of the associated free theory is identified with the $bc\beta\gamma$ vertex algebra. 
The thing we need to check is that the differential induced from the quantization of the holomorphic string agrees with the differential of the string vertex algebra. 
In fact, we observe that the induced differential $\oint I \d z$ from the classical interaction of the holomorphic bosonic string agrees with the BRST charge in Equation (\ref{brst}). 
To see that this persists at the quantum level we need to check that there are no quantum corrections. 
Indeed, this follows from the fact that the quantum master equation holds identically (as opposed to holding up to an exact term in the deformation complex) provided $\dim_\CC V = 13$. 
\end{proof}

\subsection{Relation to semi-infinite cohomology}

\brian{O to take a crack at this}

\subsection{The $E_2$ algebra and descent}

We continue to consider the theory on the Riemann surface $\Sigma = \CC$. 
In this section we show how to produce, from the point of view of factorization algebras, the structure of a Gerstenhaber algebra on the BRST cohomology of the bosonic string. 
A Gerstenhaber algebra is equivalent to an algebra over the homology of the framed little 2-disk operad. 
It is a well-known result of Lurie \cite{Lurie} that a {\em locally constant} factorization algebra on $\RR^n$ is equivalent to an algebra over the little $n$-disks operad, or an $E_n$-algebra. 
We will show that the cohomology of the factorization algebra $\Obs^{\q}_{\rm string}$ on $\CC \cong \RR^2$ has the structure of a Gerstenhaber algebra, which implies that $\Obs^\q_{\rm string}$ is equivalent to an $E_2$-algebra. 

Another occurrence of $E_n$-algebras is as the observables of topological field theories in (real) dimension $n$. 
At this level, this implies that the theory of the holomorphic bosonic string is equivalent to a topological field theory. 
In fact, we can see directly that the factorization algebra outputted by our construction is topological. 

\begin{prop} The factorization algebra $\Obs^\q_{\rm string}$ is locally constant.
\end{prop}
\begin{proof} 
We need to show that for any inclusion of open disks $D \hookrightarrow D'$ that natural map
\ben
\Obs^\q_{\rm string}(U) \to \Obs^\q_{\rm string}(U')
\een
is a quasi-isomorphism. 
We first show that this is true when we replace the quantum observables by the classical observables. 
We have already mentioned that the classical observables are functions on solutions to the classical equations of motion. 
It is convenient to phrase this in terms of Lie algebra cohomology. 
The classical interaction induces the structure of a sheaf of dg Lie algebras on the shift of the space of fields shifted up by one. 
This sheaf of dg Lie algebras is of the form
\ben
\Omega^{0,*}(\Sigma ; \cT_\Sigma) \ltimes \left(\Omega^{0,*}(\Sigma; V)[-1] \oplus \Omega^{1,*}(\Sigma;V^*)[-1] \oplus \Omega^{1,*}(\Sigma ; \cT_\Sigma^*)[-2] \right) . 
\een
We view this as a square zero extension of the dg Lie algebra $\Omega^{0,*}(\Sigma ; \cT_\Sigma)$ by the dg module inside the parentheses where the module structure is determined by the Lie derivative. 
For simplicity we denote $\cL = \Omega^{0,*}(\Sigma ; \cT_\Sigma)$ and $\cM$ the module inside the parentheses.
In this language, the space of classical observables supported on an open set $U \subset \Sigma$ is the following Chevalley-Eilenberg cochain complex
\ben
\Obs^{\cl}_{\rm string}(U) = \clie^*\left(\cL(U) ; \; \Sym(\cM(U)^*[-1])\right)
\een
where $\cM(U)^*$ denotes the continuous dual of $\cM(U)$. 
We consider the case that $U = D$ a disk centered at zero. 
By the $\dbar$-Poincar\'{e} lemma there is a quasi-isomorphism of dg Lie algebras $\cT_{hol}(D) \hookrightarrow \cL(D)$ where $\cT_{\rm hol}(D)$ is the vector space of holomorphic vector fields on $D$. 
Thus, the classical observables on $D$ are quasi-isomorphic to $\clie^*\left(\cT_{hol}(U) ; \; \Sym(\cM(U)^*[-1])\right)$.
Now, consider the composition of Lie algebras
\ben
{\rm W}_1^{\rm poly} \hookrightarrow \cT_{hol} (D) \to {\rm W}_1
\een
where ${\rm W}_1^{\rm poly}$ are the holomorphic vector fields with polynomial coefficients, and ${\rm W}_1$ is the Lie algebra of formal vector fields. 
Moreover, via the inclusion $D \hookrightarrow D'$ we have an induced commutative diagram
\ben
\xymatrix{
\clie^*\left({\rm W}_1^{\rm poly} ; {\rm res} \; \Sym(\cM(D)^*[-1])\right) \ar[d] & \ar[l] \clie^*\left(\cT_{\rm hol} (D) ; \Sym(\cM(D)^*[-1])\right) \ar[d] & \ar[l]  \clie^*\left({\rm W}_1 ; \Sym(j^\infty_0 \cM(D)^*[-1])\right) \ar[d] \\
\clie^*\left({\rm W}_1^{\rm poly} ; {\rm res} \; \Sym(\cM(D')^*[-1])\right) & \ar[l] \clie^*\left(\cT_{\rm hol} (D') ; \Sym(\cM(D')^*[-1])\right) & \ar[l] \clie^*\left({\rm W}_1 ; \Sym(j^\infty_0 \cM(D')^*[-1])\right) .
}
\een
Here, ${\rm res}$ denotes the restriction of the module along the inclusion ${\rm W}_{1}^{\rm poly} \hookrightarrow \cT_{\rm hol}(D)$ and $j_0^\infty$ denotes the infinite jet of the sheaf $\cM$ at zero.
For instance, if $\Omega^1_{\rm hol}(D)$ is the $\cT_{\rm hol}(D)$-module of holomorphic one-forms then $j_0^\infty\Omega^1_{\rm hol}$ is the ${\rm W}_1$-module of formal one-forms (one-forms with coefficients in formal power series). 

By Lemma \ref{lem: gf} (and an analogous result for polynomial vector fields) we know that $\clie^*({\rm W}_1 ; M)$ and $\clie^*({\rm W}^{\rm poly}_1; M)$ are quasi-isomorphic to the conformal dimension zero subcomplex, that is, the constants. 
The conformal zero subcomplex does not depend on the size of the disk, so we conclude that vertical arrows on the outside of the diagram above are quasi-isomorphisms. 
It follows that the middle vertical arrow is as well, thus showing that $\Obs^{\cl}_{\rm string}(D) \to \Obs^{\cl}_{\rm string}(D')$ is a quasi-isomorphism, as desired. 

To finish the proof we consider the spectral sequence induced from the filtration of the module~$\Sym \;\cM(D)$ by symmetric polynomial degree. 
The $E_1$ page of this spectral sequence is the classical observables above and it converges to the cohomology of the quantum observables. 
The map of factorization algebras induced by the inclusion $D \hookrightarrow D'$ preserves this filtration, hence by the convergence of this spectral sequence we conclude that $\Obs^{\q}_{\rm string}(D) \to \Obs^\q_{\rm string}(D')$ is also a quasi-isomorphism. 
\end{proof}

\subsubsection{Gerstenhaber bracket via descent}

There is another link to topological field theories as defined in \cite{wtop} that we recall here. 
Consider the action of the differential operators $\frac{\d}{\d z}$ and $\frac{\d}{\d \zbar}$ on the Dolbeault complex $\Omega^{0,*}(\CC)$. 
This extends to an action of the differential operators to the holomorphic bosonic string, and hence to the observables as well. 
By Noether's theorem the symmetry of the theory determined by these operators define observables: these are simply the $zz$ and $\zbar \zbar$ components of the stress energy tensor $T_{zz}$, $T_{\zbar \zbar}$. 
We will see that in the case of the string the stress energy tensors are cohomologically trivial. 
This is the general definition of a topological theory given in \cite{top}.

For each open $U \subset \CC$ we can have the operators
\ben
\frac{\d}{\d z} , \frac{\d}{\d \zbar} : \Obs^\q_{\rm string} (U) \to \Obs^\q_{\rm string}(U) .
\een 
In fact, these operators define {\em derivations} of the factorization algebra, in the sense that they are compatible with the factorization product. 
Note that these operators preserve the cohomological degree. 

Consider the operator acting on Dolbeault forms
\ben
\Bar{\eta} = \frac{\partial}{\partial (\d \zbar)} .
\een 
This extends to a derivation of degree $-1$ on the factorization algebra $\Obs^\q_{\rm string}$. 
Moreover, we have the following relation
\be\label{d/dzbar}
[\dbar + \hbar \Delta + \{I^\q, -\} , \Bar{\eta}] =  \frac{\d}{\d \zbar}
\ee
as endomorphisms of the factorization algebra. 
Indeed, one immediately obvserves $[\dbar, \Bar{\eta}] =  \frac{\d}{\d \zbar}$. 
Moreover, since $I^\q$ is a chiral deformation one has $\Bar{\eta} \cdot I^\q = 0$. 
Finally, since the pairing defining the $(-1)$ shifted symplectic structure is holomorphic we see that $\Bar{\eta}$ also commutes with the BV Laplacian $[\Bar{\eta}, \Delta] = 0$. 
This means that the operator $\frac{\d}{\d \zbar}$ acts cohomologically trivial on $\Obs^\q_{\rm string}$. 

Recall that the $b$-fields of the bosonic string are concentrated in degree $+1$ and $+2$. 
A field in degree $+1$ is of the form $f(z,\zbar) \d z^{\tensor 2} \in \Omega^{1,0}(\Sigma, T_\Sigma^{1,0*})$. 
Define the following linear observable on $\CC$,
\ben
b_{-1} : f(z,\zbar) \d z^{\tensor 2} \mapsto f(0),
\een 
which vanishes on all fields besides the degree $+1$ component of the $b$-fields.
This observable is a closed element of degree $-1$ in $\Obs^\q_{\rm string}$. 
Given any other observable $O \in \Obs^{\q}_{\rm string}$ we can define the observable $b_{-1} \cdot O$ using the symmetric product.
Denote the derivation $\eta : O \mapsto b_{-1} \cdot O$

\begin{lem} 
The derivation $\eta$ is degree $-1$ and satisfies 
\be\label{d/dz}
[\dbar + \hbar \Delta + \{I^\q, -\}, \eta] = \frac{\d}{\d z}
\ee
\end{lem}
\begin{proof} Immediately we see that both $\dbar$ and $\Delta$ commute with $\eta$. 
Thus, it suffices to show that $[\{I^\q, -\}, \eta] = \frac{\d}{\d z}$. 
Consider, for example the observable $O_{\gamma,0} : \gamma \mapsto \gamma (0)$ which we can write as the following residue $O_{\gamma,0} = \oint \frac{\gamma(z)}{z}$. 
Then, we see that the only part of the interaction that contributes is $\int \<\beta, c \cdot \gamma\>$ and 
\ben
[\{\int \<\beta, c \cdot \gamma\> , -\}, \eta] O_{\gamma, 0} = \{\int \<\beta, \partial_z \gamma\>, O_{\gamma,0}\} = \oint \frac{\partial_z \gamma}{z} \d z = \frac{\d}{\d z} O_{\gamma,0} .
\een
The calculation is similar for a general linear observable. 
To get the result for an arbitrary observable we use the fact that the bracket is a derivation. 
\brian{I tried to come up with a more conceptual proof by interpreting $\{I,-\}$ as the Chevalley-Eilenberg differential for vector fields, but failed.}
\end{proof}

This shows that the derivation $\frac{\d}{\d z}$ acts trivial up to homotopy on the factorization algebra.
Given any observable $O$, the relations (\ref{d/dzbar}) and (\ref{d/dz}) allow us to define the following differential form valued observable $\Tilde{O}$ as follows. 
The zero form part $\Tilde{O}^{0}$ is just the observable $O$. 
The one form part $\Tilde{O}^1$ is equal to the linear combination
\ben
\d z \; (\eta \cdot O) + \d \zbar \; (\Bar{\eta}\cdot O) . 
\een
Similarly, the two form part is $\Tilde{O}^2 = \d z \d \zbar \eta \Bar{\eta} \cdot O$. 
An easy calculation shows that if $O$ is closed for the quantum differential $\d^\q = \dbar + \hbar \Delta + \{I^\q, -\}$ then $\Tilde{O} = \Tilde{O}^{0} + \Tilde{O}^1 + \Tilde{O}^2$ satisfies $(\d_{dR} + \d^\q) \Tilde{O} = 0$. 
This implies that given any closed observable $O$ and submanifold $C \subset \Sigma$ we can define the closed observable
\ben
\int_C \Tilde{O} \in \Obs^\q(C) .
\een
If $O$ has cohomological degree $k$ and $C$ is of dimension $l$ then $\int_C \Tilde{O}$ has degree $k - l$. 

\begin{rmk}
A priori the factorization algebra is only well-defined on open sets $U \subset \Sigma$. 
One can define the value on a closed submanifold $C \subset \Sigma$ by taking the direct limit of what the factorization algebra assigns to open submanifolds containing $C$. 
\end{rmk} 

We are now able to define the bracket. 
Suppose $O,O'$ are closed observables on a disk $D,D'$.
For simplicity, we assume that $D \subsetneq D'$. 
Then, if $C = \partial D'$ we can define $\int_C \Tilde{O}'$ as above. 
Finally, for $D''$ another disk containing $D$ and $C$ consider the factorization product $\star : \Obs^\q(D) \tensor \Obs^\q (C) \to \Obs^\q(D'')$ and define
\ben
\{O,O'\}_{\rm Ger} := O \star \int_C \Tilde{O}' \in \Obs^\q (D'') .
\een 
We note that if ${\rm deg}(O) = k$ and ${\rm deg}(O') = k'$ then ${\rm deg}(\{O,O'\}_{\rm Ger}) = k+k' -1$. 

The last piece of structure we need to define is that of a commutative product. 
In this context, this is simply given by the factorization product of two disjoint disks including into a larger disk:
\ben
\cdot : \Obs^\q(D) \tensor \Obs^\q(D') \to \Obs^\q(D'') .
\een 

\begin{prop} The bracket $\{-,-\}_{\rm Ger}$ together with the product $\cdot$ determine the structure of a Gerstenaber algebra on the cohomology of the observables on an open disk $H^* \Obs^\q_{\rm string}(D)$. 
\end{prop}

\brian{Should we prove this or just argue that the bracket and product above agree with what Lian-Zuckerman get.}

