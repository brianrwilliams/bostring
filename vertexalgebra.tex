\section{OPE and the string vertex algebra}

Vertex algebras are mathematical objects that axiomatize the behavior of local observables 
(i.e., point-like observables) of a chiral conformal field theory (CFT),
such as the $bc\beta\gamma$ system or the holomorphic bosonic string.
The vertex operator of a vertex algebras encodes the operator product expansions (OPE) for local observables,
which is of central interest in understanding a chiral CFT.
(We will not review vertex algebras here
as there are many nice expositions \owen{cite some}.)

In this section we will explain how to extract the vertex algebra of the holomorphic bosonic string,
using machinery developed in~\cite{CG1,LiVA,CDO}.
The answer we recover is precisely the chiral sector of the usual bosonic string.
\owen{Is that true?}

\subsection{A reminder on the chiral algebra of the string}

We provide a brief background on the vertex algebra for the chiral sector of the bosonic string. 
For a detailed reference we refer the reader to the series of papers \cite{LZ1,LZ2}. 
It is easiest to introduce this as a {\em differential graded vertex algebra}. 
This is simply a vertex algebra internal to the category of chain complexes. 
The underlying graded vertex algebra has state space of the form
\ben
\cV_{\beta \gamma}^{\tensor 13} \tensor \cV_{bc}
\een
where $\cV_{\beta\gamma}$ and $\cV_{bc}$ are the $\beta\gamma$ and $bc$ vertex algebras, respectively. 
The $\beta$ and $\gamma$ generators are in grading degree zero, the $c$ generator is in grading degree $-1$, and the $b$ is in grading degree $+1$. 
In the physics literature this is referred to as the {\em BRST} grading.

Forgetting the cohomological (or BRST) grading, this vertex algebra is a conformal vertex algebra of central charge zero (by construction). 
In particular, this means that the vertex algebra has a stress energy tensor. 
Explicitly, it is of the form
\ben
T_{\rm string} (z) = \left(\sum_{i = 1}^{13} \beta_i (z) \partial_z \gamma_i (z) + \partial_z \beta_i(z) \gamma_i (z) \right) + \left(b(z) \partial_z c(z) + 2 \partial_z b(z) c(z) \right) . 
\een
Note that $T_{\rm string}$ is of cohomological degree zero. 
The first parenthesis is interpreted as the stress energy tensor of the vertex algebra $\cV_{\beta \gamma}^{\tensor 13}$ and the second term is the stress energy tensor of $\cV_{bc}$. 

We have not yet described the differential on the graded vertex algebra. 
The BRST differential is defined to be the vertex algebra derivation obtained by taking the following residue
\ben
Q^{BRST} = \oint c(z) T_{\rm string}(z) .
\een 
By construction this operator satisfies $(Q^{BRST})^2 = 0$. 

\begin{dfn} The {\em string vertex algebra} is the dg vertex algebra 
\ben 
\cV_{\rm string} = \left(\cV_{\beta \gamma}^{\tensor 13} \tensor \cV_{bc}, Q^{BRST}\right)  .
\een
\end{dfn}


\subsection{Some context}

In the BV formalism one constructs a cochain complex of observables,
for both the classical and the quantized theory, if it exists.
The cochain complexes are local on the source manifold of a theory:
on each open set $U$ in that manifold~$\Sigma$,
one can pick out the observables with support in~$U$ by asking for the observables that vanish on fields with support outside~$U$.
It is the central result of~\cite{CG1,CG2} that the observables also satisfy a local-to-global property,
akin to the sheaf gluing axiom,
and hence form a {\em factorization algebra} on~$\Sigma$.

We will not need that general notion here.
Instead, we will use vertex algebras.
Theorem~\owen{???} of~\cite{CG1} explains how a factorization algebra~$F$ on $\Sigma = \CC$
yields a vertex algebra~$\Vert(F)$, under natural hypotheses on~$F$. 
It assures us that the observables of a chiral CFT determine a vertex algebra.

In particular, Section~\owen{???} of~\cite{CG1} examines the free $\beta\gamma$ system in great detail.
Its main result is that the well-known $\beta\gamma$ vertex algebra is recovered by the two-step process of BV quantization, which yields a factorization algebra, and then the extraction of a vertex algebra.

The exact same arguments apply to the free $bc\beta\gamma$ system,
and we obtain the following.

\begin{prop}
\owen{State the $bc\beta\gamma$ VOA. Mention the classical (=associated graded) observables are a Poisson vertex algebra.}
\end{prop}

\subsection{The case of the string}

The holomorphic bosonic string is a chiral CFT and so the machinery of~\cite{CG1} applies to it.
One can extract a vertex algebra directly by this method.

But there is a slicker approach, using Li's work~\cite{Li},
which studies chiral deformations of free chiral BV theories such as the $bc\beta\gamma$ system.
Recall, a deformation of a classical field theory is given by a local functional. 
We have seen that this is essentially the data of a Lagrangian density, which is a density valued multilinear functional that depends on (arbitrarily high order) jets of the fields. 
In other words, for a field $\varphi$, a Lagrangian density is of the form
\ben
\cL(\varphi) = \sum (D_{k_1} \varphi) \cdots (D_{k_m} \varphi) \cdot {\rm vol}_\Sigma
\een 
for $C^\infty(\Sigma)$-valued differential operators $D_{k_i}$.
By a {\em chiral} Lagrangian density we mean a Lagrangian for which the differential operators $D_{k_i}$ are all holomorphic. 
For instance, on $\Sigma = \CC$, we require $D_{k_i}$ to be a sum of operators of the form $f(z) \partial_z^n$ where $f(z)$ is a holomorphic function. 
On $\Sigma = \CC$ we will also require the chiral Lagrangian to be translation invariant. 
This means that all differential operators $D_{k_i}$ are of the form $\partial_z^n$. 
Thus, a {\em translation invariant chiral deformation} is a local functional of the form
\ben
I(\varphi) = \sum \int (\partial^{k_1}_z \varphi) \cdots (\partial^{k_m} \varphi) \d^2 z .
\een

One of his main results is that for a free chiral BV theory with action $S_{\rm free}$ and associated vertex algebra $\cV_{\rm free}$, one has the following:
\owen{Check that this is an accurate summary of his work}
\begin{itemize}
\item For any chiral interaction~$I$, the action $S_I = S_{\rm free} + I$ needs no counterterms, 
and yields a renormalized action~$\{S_I[L]\}$.
\item If the renormalized action $\{S_I[L]\}$ satisfies the quantum master equation,
then it determines a vertex algebra derivation $D_I$ of~$\cV_{\rm free}$.
\item The vertex algebra $\cV_I$ for such an action $\{S_I[L]\}$ is isomorphic to the vertex algebra~$\cV_{\rm free}$ via the automorphism~$\exp(D_I)$.
\end{itemize}
The holomorphic bosonic string with target $V=\CC^{13}$ provides a concrete example of this situation.
The free theory is the $bc\beta\gamma$ system, 
and we have seen that the renormalized action satisfies the QME.
Hence we obtain the following.

\begin{prop} 
Let $\Obs^q_{\rm string}$ be the factorization algebra on $\Sigma = \CC$ of the holomorphic bosonic string with target~$V = \CC^{13}$. 
There is an isomorphism of vertex algebras $\cV_{\rm string} \cong \Vert(\Obs^q_{\rm string})$.
Moreover, this vertex algebra is isomorphic to the chiral sector of the bosonic string \owen{cite some description}.
\end{prop}

\owen{Compare with Lian-Zuckerman. We need to see that our differential on the dg vertex algebra is the BRST operator they write down.}

\owen{describe classical observables \& why we get a Poisson vertex algebra}


\brian{Write down vertex algebra from quantization above. Possibly state the relationship to semi-infinite cohomology}

\owen{I believe these vertex algebras are cohomologically graded, unless we're lucky and the cohomology all sits in degree zero. In which case, we should point out this miracle. Perhaps better would be to extract the dg vertex algebra.}

\owen{Do you know a citation where the string vertex algebra is already written down? Of course it's almost explicit in any discussion of the ``modern''/BRST quantization of the bosonic string, where they write down $Q$, which ought to be the differential of the dg vertex algebra using our construction.}

\owen{Si's theorem gives us a nice approach: we conjugate the free $bc\beta\gamma$ vertex algebra by the appropriate VOA automorphism to recover the string. This may be close to what physicists do anyway.}

\brian{How deformations discussed in Section 3 gives explicit deformations of the the vertex algebra.}
