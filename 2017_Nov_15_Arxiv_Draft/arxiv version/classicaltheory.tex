\section{The classical holomorphic bosonic string} \label{sec:classical}

There is a basic format for a string theory, at least in the perturbative approach. 
One starts with a nonlinear $\sigma$-model, 
whose fields are smooth maps from a Riemann surface to a target manifold $X$;
in this setting we want the theory to make sense for an arbitrary Riemann surface as the source manifold.
In the usual bosonic string theory, 
this nonlinear $\sigma$-model picks out the harmonic maps from a Riemannian 2-manifold to a Riemannian manifold.
In our holomorphic setting,
the nonlinear $\sigma$-model picks out holomorphic maps from a Riemann surface to a complex manifold.
One then quotients the space of fields (and solutions to the equations of motion) with respect to reparametrization.
In the usual bosonic string,
one quotients by diffeomorphisms and Weyl scalings, which can thus change the metric on the source.
In our setting, we quotient by biholomorphisms, which act on the complex structure on the source.

In this section we begin by describing our theory in the BV formalism.
We do not expect the reader to find the action functional immediately clear,
so we devote some time to analyzing what it means and how it arises from concrete questions.
We then turn to interpreting this classical BV theory using dg Lie algebras and derived geometry
(i.e., we identify the moduli space it encodes).
Finally, we conclude by sketching how our theory appears as the chiral sector of a degeneration of the usual bosonic string when the target is a complex manifold with a Hermitian metric.
Our theory thus does provide insights into the usual bosonic string;
moreover, it clarifies why so many aspects of the bosonic string,
like the anomalies or $B$-fields, 
have holomorphic analogues.

\subsection{The theory we study} 

Let $V$ denote a complex vector space (the target),
and let $\langle-,-\rangle_V$ denote the evaluation pairing between $V$ and its linear dual~$V^\vee$.
Let $\Sigma$ denote a Riemann surface (the source).
Let $T_\Sigma^{1,0}$ denote the holomorphic tangent bundle on $\Sigma$, 
let $\langle-,-\rangle_T$ denote the evaluation pairing between $T_\Sigma^{1,0}$ and its vector bundle dual~$T_\Sigma^{1,0*}$. 
These are the key geometric inputs.

In a BV theory, the fields are $\ZZ$-graded;
we call this the {\em cohomological grading}.
We have four kinds of fields:
\[
\begin{array}{ccccc}
\text{field} & -1 & 0 & 1 & 2\\
\hline
\gamma & & \Omega^{0,0}(\Sigma) \otimes V & \Omega^{0,1}(\Sigma) \otimes V & \\
\beta & & \Omega^{1,0}(\Sigma) \otimes V^\vee & \Omega^{1,1}(\Sigma) \otimes V^\vee & \\
c & \Omega^{0,0}(\Sigma, T^{1,0}_\Sigma) & \Omega^{0,1}(\Sigma, T^{1,0}_\Sigma) & \\
b & & & \Omega^{1,0}(\Sigma, T^{1,0 *}_\Sigma) & \Omega^{1,1}(\Sigma, T^{1,0 *}_\Sigma)
\end{array}
\]
More accurately, we have eight different kinds of fields, 
but we view each row as constituting a single type 
since each given row consists of the Dolbeault forms of a holomorphic vector bundle.
For instance, the field $\gamma$ is a $(0,*)$-form with values in the trivial bundle with fiber~$V$,
and the field $b$ is a $(0,*)$-form with values in the bundle $T^{1,0 *} \otimes~T^{1,0 *}$.

To orient oneself it is helpful to start by examining the fields of cohomological degree zero,
since these typically have a manifest physical meaning.
For instance, the degree zero $\gamma$ field is a smooth $V$-valued function
and hence the natural field for the nonlinear $\sigma$-model into~$V$.
The degree zero $c$ field is a smooth $(0,1)$-form with values in vector field ``in the holomorphic direction,''
and hence encodes an infinitesimal change of complex structure of~$\Sigma$.
The degree $-1$ part of $c$ contains the gauge fields of the theories, vector fields. 
The equations of motion dictate that these vector fields be holomorphic, so we are seeing the infinitesimal version of the symmetry by biholomorphisms we mentioned above.
These constitute the obvious fields to introduce for a holomorphic version of the bosonic string.
The fields $\beta$ and $b$ are less obvious but appear as ``partners'' (or antifields)
whose role is clearest once we have the action functional and hence equations of motion.

The action functional is
\begin{equation}\label{bosaction}
S(\gamma,\beta,c,b) = 
\int_\Sigma \langle \beta, \dbar \gamma \rangle_V 
+ \int_\Sigma \langle b, \dbar c \rangle_T 
+ \int_\Sigma \langle \beta, [c,\gamma] \rangle_V 
+ \int_\Sigma \langle b, [c,c] \rangle_T.
\end{equation}
(We discuss below how to think about fields with nonzero cohomological degrees as inputs.)
The equations of motion are thus
\begin{alignat*}{2}
0 &= \dbar \gamma + [c,\gamma] & \quad\quad  0 &= \dbar \beta + [c,\beta] \\
0  &= \dbar c + \tfrac{1}{2} [c,c] & \quad\quad  0 &= \dbar b + [c,b] 
\end{alignat*}
Note that these equations are familiar in complex geometry.
For instance, the equation purely for $c$ encodes a deformation of complex structure on $\Sigma$; concretely, it modifies the $\dbar$ operator to $\dbar + c$.
The other equations then amount to solving for holomorphic sections (of the relevant bundle) withe respect to this deformed complex structure.
For instance, the equation in $\gamma$ picks out holomorphic maps from $\Sigma$,
with the $c$-deformed complex structure, to~$V$.

Note that $b$ essentially appears as a Lagrange multiplier, so it doesn't have any intrinsic meaning physical meaning by itself. 
The field $b$ can be understood as an ``antifield" to the ghost field $c$, in other words an {\em antighost}. 

\begin{rmk}
\label{rmk:bcbg}
Just looking at this action functional, one might notice that if one drops the last two terms,
which are cubic in the fields, then one obtains a free theory
\begin{equation}
S_{free}(\gamma,\beta,c,b) = 
\int_\Sigma \langle \beta, \dbar \gamma \rangle_V 
+ \int_\Sigma \langle b, \dbar c \rangle_T,
\end{equation}
which is known as the {\em free $bc\beta\gamma$ system}.
Thus, one may view the holomorphic bosonic string as a deformation of this free theory
by ``turning on'' those interaction terms.
We will repeatedly try a construction first with this free theory before tackling the string itself,
as it often captures important information with minimal work.
For instance, we will examine the vertex algebra for the free theory before seeing how the interaction affects the operator products.
Similarly, one can identify the anomaly already at the level of the free theory.

This viewpoint of arriving at the bosonic string as a deformation of a free CFT is central to the analysis of the string in the physics literature \cite{GSW1} and Chapter 2 of \cite{polchinski}. 
See also the work in \cite{Scherk}. 
\end{rmk}

\begin{rmk}
\label{rmk:curved}
It is easy to modify this action functional to allow a curved target,
i.e., one can replace the complex vector space $V$ with an arbitrary complex manifold~$X$. 
The fields $b,c$ remain the same.
The degree 0 field $\gamma$ still encodes smooth maps into $X$, but now the degree 1 field is a section of $\Omega^{0,1}(\Sigma, \gamma^*T^{1,0}_X)$.
Similarly, $\beta$ is now a section of $\Omega^{1,*}(\Sigma, \gamma^*T^{1,0*}_X)$.
The action is then
\begin{equation}\label{curved action}
S(\gamma,\beta,c,b) = 
\int_\Sigma \langle \beta, \dbar \gamma \rangle_{T_X}
+ \int_\Sigma \langle b, \dbar c \rangle_{T_\Sigma} 
+ \int_\Sigma \langle \beta, [c,\gamma] \rangle_{T_X}
+ \int_\Sigma \langle b, [c,c] \rangle_{T_\Sigma}.
\end{equation}
In Section \ref{sec:curved} we will indicate how the results with linear target generalize to this situation.
\end{rmk}

\subsection{From the perspective of derived geometry}

We would like to explain what this theory is about in more conceptual terms,
rather than simply by formulas and equations.
Thankfully this theory is amenable to such a description.
We will be informal in this section and not specify a particular geometric context (e.g., derived analytic stacks),
except when we specialize to the deformation-theoretic situation (i.e., perturbative setting) that is our main arena.

\def\Maps{\operatorname{Maps}}

Let $\cM$ denote the moduli space of Riemann surfaces,
so that a surface $\Sigma$ determines a point in~$\cM$.
Let $\Maps_{\dbar}(\Sigma,V)$ denote the space of holomorphic maps from $\Sigma$ to $V$,
and hence a bundle $\Maps_{\dbar}(-,V)$ over $\cM$ by varying~$\Sigma$.
For our equations of motion, the $\gamma$ and $c$ fields of a solution determine a point in this bundle~$\Maps_{\dbar}(-,V)$. 
The commutative algebra $\cO(\Maps_{\dbar}(\Sigma,V))$ of functions on the space encodes the  observables of the classical theory.

\def\RS{{\cR\cS}}

This construction makes sense on noncompact Riemann surfaces as well.
Let $\RS$ denote the category whose objects are Riemann surfaces and whose morphisms are holomorphic embeddings.
There is a natural site structure: a cover is a collection of maps $\{S_i \to \Sigma\}_i$ such that the union of the images is all of~$\Sigma$.
Then $\Maps_{\dbar}(-,V)$ defines a sheaf of spaces over~$\RS$.
The observables for the classical theory is, in essence, the {\em co}\/sheaf of commutative algebras~$\cO(\Maps_{\dbar}(-,V))$,
and hence provides a factorization algebra.

In fact, it is better to work with the derived version of these spaces.
One important feature of derived geometry is that the appropriate version of a tangent space at a point is, in fact, a cochain complex.
In our setting, a point $(c,\gamma)$ in $\Maps_{\dbar}(-,V)$ determines a a complex structure $\dbar + c$ on $\Sigma$---we denote this Riemann surface by $\Sigma_c$---and $\gamma$ a $V$-valued holomorphic function on~$\Sigma_c$.
The tangent complex of $\Maps_{\dbar}(-,V)$ at $(c,\gamma)$ is precisely 
\[
\Omega^{0,*}(\Sigma_c,T^{1,0})[1] \oplus \Omega^{0,*}(\Sigma_c,V).
\]
The first summand is the usual answer from the theory of the moduli of surfaces 
(recall, for example, that the ordinary tangent space is the sheaf cohomology $H^1(\Sigma,T_\Sigma)$ of the holomorphic tangent sheaf),
and the second is usual elliptic complex encoding holomorphic maps.

\begin{rmk}
It is useful to bear in mind that the degree zero cohomology of the tangent complex will recover the ``naive'' tangent space. 
In our case, we have 
\[
H^1(\Sigma_c,T_{\Sigma_c}) \oplus H^0(\Sigma_c,V),
\]
which encodes deformations of complex structure and holomorphic maps.
Negative degree cohomology of the tangent complex detects infinitesimal automorphisms (and automorphisms of automorphisms, etc) of the space.
For instance, here we see $H^0(\Sigma_c,T_{\Sigma_c})$ appear in degree -1, 
since a holomorphic vector field is an infinitesimal automorphism of a complex curve.
These negative directions are called ``ghosts'' (or ghosts for ghosts, and so on) in physics.
The positive degree cohomology detects infinitesimal relations (and relations of relations, and so on).
\end{rmk}

Note that the underlying graded spaces of this tangent complex are the $c$ and $\gamma$ fields from the BV theory described above.
We emphasize that the tangent complex is only specified up to quasi-isomorphism,
but it is compelling that a natural representative is the BV theory produced by the usual physical arguments.
This behavior, however, is typical of the relationship between derived geometry and BV theories:
when physicists write down a classical BV theory, 
the underlying free theory is essentially always the tangent complex of a nice derived stack.

The reader has probably noticed that, yet again, we have postponed discussing the $\beta$ and $b$ fields.
From a derived perspective, the full BV theory describes the shifted cotangent bundle $\TT^*[-1]\Maps_{\dbar}(-,V)$.
At the level of a tangent complex, the shifted cotangent direction contributes
\[
\Omega^{1,*}(\Sigma_c,T^{1,0*})[-1] \oplus \Omega^{1,*}(\Sigma_c,V^\vee),
\]
whose underlying graded spaces are the $\beta$ and $b$ fields.
These ``antifields'' are added so that the overall space of fields has a 1-shifted symplectic structure  when $\Sigma$ is closed, and a shifted Poisson structure when $\Sigma$ is open.

\subsection{Relationship to the Polyakov action functional}

This holomorphic bosonic string has a natural relationship with the usual bosonic string.
We sketch it briefly, only considering a linear target.

We begin with a bosonic string theory where the source is a 2-dimensional smooth oriented Riemannian manifold $\Sigma$ and the target is a Hermitian vector space~$(V,h)$. 
The ``naive'' action functional is
\ben
S^{\text{naive}}_{Poly}(\varphi, g) = \int_\Sigma h(\varphi, \Delta_{g} \varphi)\, \dvol_g
\een
where the field $g$ is a Riemannian metric on $\Sigma$ and the field $\varphi$ is a smooth map from $\Sigma$ to~$V$.
The notation $\Delta_g$ denotes the Laplace-Beltrami operator on~$\Sigma$. 

Note that $S^{naive}_{Poly}$ is invariant under the diffeomorphism group ${\rm Diff}(\Sigma)$ and under rescalings of the metric
(i.e., the theory is classically conformal).
Typically we express rescaling as $g \mapsto e^{f} g$ with $f \in~C^\infty(\Sigma)$.
As we are interested in a string theory, we want to gauge these symmetries.
In geometric language, we want to think about the quotient stack 
obtained by taking solutions to the equations of motion and quotienting by these symmetry groups.

Our focus is perturbative, so that we want to study the behavior of this action near a fixed solution to the equations of motion.
In other words, we want to work with the Taylor expansion of the true action near some solution.
Hence, we work around a fixed metric $g_0$ on $\Sigma$, 
and we substitute for the field $g$,
the term $g_0+\alpha$ where $\alpha \in \Gamma(\Sigma,\Sym^2(T_\Sigma))$.
That is, we will consider deformations of~$g_0$.
As $\varphi$ is linear, we just consider expanding around the zero map.
Thus our new fields are $\varphi \in C^\infty(\Sigma,V)$ and $\alpha \in \Gamma(\Sigma,\Sym^2(T_\Sigma))$.

There are also ghost fields associated to the symmetries we gauge.
First, there are infinitesimal diffeomorphisms,  which are described by vector fields on~$\Sigma$.
We denote this ghost field by $X \in \Gamma(\Sigma,T_\Sigma)$.
It acts on the fields by the transformation 
\[
(\varphi,\alpha) \mapsto (\varphi + X \cdot \varphi, \alpha + L_X \alpha), 
\]
where $L_X$ denotes the Lie derivative on tensors.
Second, there are infinitesimal rescalings of the metric, 
such as $\alpha \mapsto \alpha + f \alpha$, 
with ghost field $f \in~C^\infty(\Sigma)$.
The rescaling does not affect $\varphi$.
The two symmetries are compatible: 
given $f$ and $X$, then $L_{X} (f \alpha) = X(f) \alpha + f L_X \alpha$ for any $\alpha \in \Sym^2(T_\Sigma)$.

To summarize, we have the following graded vector space of fields:
\[
\begin{array}{ccccc}
\text{field/antifield} & -1 &0 & 1 & 2\\
\hline
\varphi, \varphi^\vee & & \Omega^{0}(\Sigma) \tensor V & \Omega^2(\Sigma) \tensor V & \\
\alpha, \alpha^\vee & &  \Omega^0(\Sigma,\Sym^2(T_\Sigma)) & \Omega^2(\Sigma ; \Sym^{2}(T^*_\Sigma)) & \\
X, X^\vee & {\rm Vect}(\Sigma) & & & \Omega^2(\Sigma ; T^*_\Sigma)\\
f, f^\vee & C^\infty(\Sigma) &&& \Omega^2(\Sigma) .
\end{array}
\]
The BV action functional is of the form:
\begin{align}\label{polyakov bv}
S^{BV}_{Poly} (\varphi, \alpha, X, f) = 
& \int_\Sigma h(\varphi, \Delta_{g_0} \varphi)\, \dvol_{g_0} + \sum_{n \geq 1} \frac{1}{n!} \int_{\Sigma} h(\varphi, D_n(\alpha) \varphi) \dvol_{g_0} \\
& +  \int_{\Sigma} h(\varphi, X \cdot \varphi) \dvol_{g_0} \\
& +  S'(X, f, \alpha) 
\end{align}
The right hand side of the first line amounts to expanding out the Laplace-Beltrami operator $\Delta_{g_0 + \alpha}$ as a function of $\alpha$.
Hence, the $D_n$ are differential operators of the form 
$D_n : \left(\Sym^2(T_\Sigma)\right)^{\tensor n} \to {\rm Diff}^{\leq 2} (\Sigma)$ 
where ${\rm Diff}^{\leq 2}(\Sigma)$ are order $\leq 2$ differential operators on $\Sigma$.
Thus, for each section $\alpha$ of $\Sym^2(T_\Sigma)$, we get a second-order differential operator $D_n(\alpha)$ acting on functions on $\Sigma$. (This term is the $n$th term in the Taylor expansion, so its homogeneous of order $n$: $D_n(t \alpha) = t^n D_n(\alpha)$ for a scalar $t$.)
The second line encodes how vector fields act on the maps of the $\sigma$-model.
The third line $S'(X, f, \alpha)$ is independent of $\varphi$
and only depends on the fields $f$, $X$, $\alpha$ and their antifields (denoted with checks $(-)^\vee$).
It is of the form
\ben
S'(f,X,\alpha) = \int_\Sigma \<\alpha^\vee, L_X (g_0 + \alpha) + f (g_0 + \alpha)\> + \int_\Sigma \<X^\vee, [X,X]\> + \int_\Sigma \<f^\vee, X \cdot f\> .
\een
The first term encodes how vector fields and Weyl transformations act on the perturbed metric $g_0 + \alpha$ and the remaining terms are required to ensure the gauge symmetry is consistent (satisfies the classical master equation). 

An explicit formula for $D_n(\alpha,\ldots,\alpha)$ is a rather involved exercise (and not needed here).
For instance, if we are working locally on $\Sigma = \RR^2$ with the $g_0$ the flat metric, 
then the operator $D_1(\alpha)$ is sum of a first-order and a second-order differential operator
\ben
D_1(\alpha) = \frac{1}{2} \frac{\partial}{\partial x^i} ({\rm tr}(\alpha)) \frac{\partial}{\partial x^i} + \frac{1}{2} {\rm tr} (\alpha) \frac{\partial}{\partial x^i} \frac{\partial}{\partial x^i}, 
\een
or in a more coordinate-free notation, 
\[
D_1(\alpha) = \frac{1}{2} \star \d \left({\rm tr}(\alpha) \star \d\right).
\] 
Here, we use the natural trace map ${\rm tr} : \Sym^2 T\Sigma \to C^\infty(\Sigma)$ of symmetric $2\times2$ matrices. 

There is an important parameter in this action functional: the Hermitian inner product $h$.
We can consider scaling it $h \to t h$, with $t \in (0,\infty)$.
The ``infinite volume limit'' as $t \to \infty$ admits a nice description,
provided one rewrites the action functional in a first-order formalism
(i.e., adjoins fields so that only first-order differential operators appear in the action,
which is a sort of action functional analogue of working with phase space).

\begin{lem} 
In this infinite volume limit
the bosonic string becomes equivalent to a BV theory whose action functional has the form
\ben
S(\beta, \gamma, b,c) + \Bar{S}(\Bar{\beta}, \Bar{\gamma}, \Bar{b}, \Bar{c}),
\een
where $S(\beta, \gamma, b,c)$ is the action functional for the holomorphic bosonic string in Equation (\ref{bosaction}) and $\Bar{S}$ is its anti-holomorphic conjugate. 
\end{lem}

\begin{rmk} 
The action functional $\Bar{S}$ is similar to $S$ where the fields $\gamma,\beta,b,c$ are replaced by sections in the the relevant conjugate bundles. 
For example, $\beta \in \Omega^{1,*}(\Sigma)$ becomes $\Bar{\beta} \in \Omega^{*,1} (\Sigma)$. 
Moreover, the operator $\dbar$ is replaced by the holomorphic Dolbeault operator $\partial$. 
Another way of saying this is that $\Bar{S}$ is the holomorphic string on $\Bar{\Sigma}$, which is the conjugate complex structure to $\Sigma$. 
\end{rmk}

\begin{proof}[Outline of proof] 
There are two things that may cause alarm in the statement of the claim. 
First, the space of fields of the Polyakov string (in the BV language) and those of the holomorphic bosonic string do not match up. 
Second, the infinite volume limit $t \to \infty$ is naively ill-defined using the action functional (\ref{polyakov bv}). 
It turns out that these two issues are solved by the same maneuver. 

We begin with the first term in the first line of (\ref{polyakov bv}). 
Notice that it is simply the action functional for the $\sigma$-model of maps from $(\Sigma, g_0)$ to $(V, h)$. 
It is shown in Appendix 21 of \cite{GGW} how to make sense of the infinite volume limit of this usual $\sigma$-model. 
The idea is to rewrite this theory in the {\em first order formalism}.
This amounts to introducing a new field $B \in \Omega^1(\Sigma) \tensor V^\vee$ and action functional 
\ben
\int_\Sigma \<B, \d \varphi\>_V - \frac{1}{2} \int_\Sigma h^\vee(B, \star B)
\een
where $\<-,-\>_V$ represents the evaluation pairing between $V$ and its dual, 
$\star$ is the Hodge star operator for the metric $g_0$, 
and $h^\vee$ denotes the dual metric on $V$. 
This action functional is equivalent to the original $\sigma$-model;
one can compare the equations of motion. 
Moreover, the limit $(th)^\vee = (1/t)h^\vee$, 
and so in the infinite volume limit $t \to \infty$, 
the dual $(th)^\vee$ goes to $0$, which kills the second term in the first order action. 
The remaining theory splits as the direct sum of the free $\beta\gamma$ system with target $V$ and its anti-holomorphic conjugate. 
At the level of fields, the original field $\varphi$ corresponds to $\gamma + \Bar{\gamma}$ in the first order description,
and $B$ corresponds to~$\beta+\Bar{\beta}$. 

Of the remaining terms on the first line of the action, only the $D_1$ term survives in this infinite volume limit. 
Furthermore, the metric $g_0$ determines an injective map
\ben
\Omega^{0,1}(\Sigma ; T^{1,0}_\Sigma) \oplus \Omega^{1,0}(\Sigma ; T^{0,1}_\Sigma) \hookrightarrow \Sym^2(T_\Sigma) .
\een
Restricting the term $\int_\Sigma h(\varphi, D_1(\alpha) \varphi)$ to the image,
we obtain a term whose infinite volume limit~is
\ben
\int_\Sigma \<\beta , [c, \gamma]\>_V + \int_\Sigma \<\Bar{\beta} , [\Bar{c}, \bar{\gamma}]\>_V.
\een
This first term is precisely the third term in the holomorphic string action functional (\ref{bosaction}), which describes how deformations of complex structure couple to the fields of the $\sigma$-model. 

In the infinite volume limit, the term $S'(f, X, \alpha)$ recovers the terms
\[
\int_\Sigma \langle b, \dbar c \rangle_T + \int_\Sigma \langle b, [c,c] \rangle_T
\]
in the action of the holomorphic string, plus their conjugates. 
The arguments are similar to those we have just sketched. 
\end{proof}

\begin{rmk} Another approach to arrive at the holomorphic theory we consider comes from considering supersymmetry. 
Without gravity, the pure holomorphic $\sigma$-model can be viewed as the {\em holomorphic twist} of the $N=(2,0)$ supersymmetric $\sigma$ model (in this case the target is required to be K\"{a}hler). 
Moreover, the $\beta\gamma bc$ system is the holomorphic twist of the $N=(2,2)$ model. 
Conjecturally, we expect the holomorphic theory of gravity we consider to be the holomorphic twist of two-dimensional $N=2$ supergravity.
\end{rmk}

\begin{rmk} In this infinite volume limit, one can put the dependence of the target metric back into the theory by choosing a certain background to work in. 
In the BV formalism this amounts to choosing a certain deformation parameter, which in this instance corresponds to infinitesimal deformations of the target metric.
Note that to deform the metric on the target we leave the world of ``holomorphic field theory" as the deformation involves both $z$ and $\zbar$ dependent terms. 
It would be interesting to study how to formulate the theory with finite target metric in the BV formalism.
\end{rmk}
