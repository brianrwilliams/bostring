\section{Looking ahead: curved targets}
\label{sec:curved}

In this section we briefly advertise our future work, which is to provide a complete analysis of the bosonic string with a complex manifold as the target. 
Our approach is a modification of our treatment of the curved $\beta\gamma$ system given in \cite{GGW}.
The main idea there was to consider the $\beta\gamma$ system with target a formal disk $\hD^n$. 
Then, in the style of Gelfand and Kazhdan's treatment of formal geometry, we show how working equivariantly for formal automorphisms allows us to globalize this theory to a complex manifold. 
In general, we find an obstruction to doing this, which is measured by the second component of the Chern character of the tangent bundle of the complex manifold. 
The appearance of the characteristic class is expected from the theory of chiral differential operators.
In fact, we show that the factorization algebra of observables descends to the sheaf of chiral differential operators on the target manifold. 

We will give a similar argument for the bosonic string. 
The key difference to the $\beta\gamma$ system is that even in the case of a flat target, the bosonic string is an interacting theory.
Nevertheless, the theory of BV quantization that is equivariant for formal automorphisms can still be applied and we arrive at the following result. 

\begin{thm}[\cite{GWcurved}] 
Consider the holomorphic bosonic string with target a complex manifold $X$. 
There exists a one-loop exact quantization if and only if
\begin{itemize}
\item[(1)] $\dim_\CC X = 13$,

\item[(2)] $\ch_2(T_X) = 0$, and

\item[(3)] $c_1(T_X) = 0$.
\end{itemize}
Moreover, if the conditions above hold, the space of all quantizations is a torsor for the abelian group
\ben
H^1(X , \Omega^2_{cl}) \oplus H^0(X, \Omega^1) .
\een
\end{thm}

There are two further directions we hope to address in our future work:
\begin{itemize}
\item[(1)] We have seen that the local observables for the case of a flat target return the semi-infinite BRST cohomology of the $\beta\gamma$ vertex algebra. 
We expect that the local observables in the case of a curved target produce a sheafy refinement of semi-infinite cohomology. 
This should produce a sensitive invariant of the target manifold and gives a variant of quantum sheaf cohomology. 
\item[(2)] The partition function of the curved $\beta\gamma$ system on elliptic curves is known to produce the Witten genus of the target manifold \cite{wg2}. 
For flat space, the partition function of the string is given by the Mumford form \cite{BM}. 
We propose that the partition function for the curved string produces an invariant of the target manifold analogous to the Witten genus. 
\end{itemize}

\owen{We probably should mention, e.g., Alvarez-Singer and others}
