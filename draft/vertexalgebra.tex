\section{OPE and the string vertex algebra}

Vertex algebras are mathematical objects that axiomatize the behavior of local observables 
(i.e., point-like observables) of a chiral conformal field theory (CFT),
such as the $bc\beta\gamma$ system or the holomorphic bosonic string.
The vertex operator of a vertex algebras encodes the operator product expansions (OPE) for local observables,
which is of central interest in understanding a chiral CFT.
(We will not review vertex algebras here
as there are many nice expositions \cite{FHL, BZF}.)

In this section we will explain how to extract the vertex algebra of the holomorphic bosonic string,
using machinery developed in~\cite{CG1,LiVA,GGW}.
The answer we recover is precisely the chiral sector of the usual bosonic string.

\subsection{A reminder on the chiral algebra of the string}\label{subsec: string vert}

We provide a brief background on the vertex algebra for the chiral sector of the bosonic string. 
For a detailed reference we refer the reader to the series of papers \cite{LZ1,LZ2}. 
It is easiest to introduce this as a {\em differential graded vertex algebra}. 
This is simply a vertex algebra internal to the category of chain complexes. 
The underlying graded vertex algebra has state space of the form
\ben
\cV_{\beta \gamma}^{\tensor 13} \tensor \cV_{bc}
\een
where $\cV_{\beta\gamma}$ and $\cV_{bc}$ are the $\beta\gamma$ and $bc$ vertex algebras, respectively. 
The $\beta$ and $\gamma$ generators are in grading degree zero, the $c$ generator is in grading degree $-1$, and the $b$ is in grading degree $+1$. 
In the physics literature this is referred to as the {\em BRST} grading.

Forgetting the cohomological (or BRST) grading, this vertex algebra is a conformal vertex algebra of central charge zero (by construction). 
In particular, this means that the vertex algebra has a stress-energy tensor. 
Explicitly, it is of the form
\ben
T_{\rm string} (z) = \left(\sum_{i = 1}^{13} \beta_i (z) \partial_z \gamma_i (z) + \partial_z \beta_i(z) \gamma_i (z) \right) + \left(b(z) \partial_z c(z) + 2 \partial_z b(z) c(z) \right) . 
\een
Note that $T_{\rm string}$ is of cohomological degree zero. 
The first parenthesis is interpreted as the stress-energy tensor of the vertex algebra $\cV_{\beta \gamma}^{\tensor 13}$ and the second term is the stress-energy tensor of $\cV_{bc}$. 

We have not yet described the differential on the graded vertex algebra. 
The BRST differential is defined to be the vertex algebra derivation obtained by taking the following residue
\be\label{brst}
Q^{BRST} = \oint c(z) T_{\rm string}(z) .
\ee
By construction this operator satisfies $(Q^{BRST})^2 = 0$. 

\begin{dfn} The {\em string vertex algebra} is the dg vertex algebra 
\ben 
\cV_{\rm string} = \left(\cV_{\beta \gamma}^{\tensor 13} \tensor \cV_{bc}, \; Q^{BRST}\right)  .
\een
\end{dfn}

There is another grading on $\cV_{\rm string}$ coming from the eigenvalues of the vertex algebra derivation $c_0$ called the {\em conformal dimension}. 
In particular, this determines a filtration and we can consider the associated graded ${\rm Gr} \; \cV_{\rm string}$. 
The conformal weight grading preserves the cohomological grading so that this object still has the structure of a dg vertex algebra. 

Note that the cohomology of a dg vertex algebra is an ordinary (graded) vertex algebra. 
The cohomology of the string vertex algebra is called the {\em BRST cohomology} of the bosonic string. 
In the remainder of this section we will show how we recover the string vertex algebra from the quantization of the holomorphic bosonic string.

\subsection{Some context}

In the BV formalism one constructs a cochain complex of observables,
for both the classical and the quantized theory, if it exists.
The cochain complexes are local on the source manifold of a theory:
on each open set $U$ in that manifold~$\Sigma$,
one can pick out the observables with support in~$U$ by asking for the observables that vanish on fields with support outside~$U$.
It is the central result of~\cite{CG1,CG2} that the observables also satisfy a local-to-global property,
akin to the sheaf gluing axiom,
and hence form a {\em factorization algebra} on~$\Sigma$.

We will not need that general notion here.
Instead, we will use vertex algebras.
Theorem~5.2.3.1 of~\cite{CG1} explains how a factorization algebra~$F$ on $\Sigma = \CC$
yields a vertex algebra~$\Vert(F)$, under natural hypotheses on~$F$. 
It assures us that the observables of a chiral CFT determine a vertex algebra.

In particular, Section~5.3 of~\cite{CG1} examines the free $\beta\gamma$ system in great detail.
Its main result is that the well-known $\beta\gamma$ vertex algebra is recovered by the two-step process of BV quantization, which yields a factorization algebra, and then the extraction of a vertex algebra.

The exact same arguments apply to the free $bc\beta\gamma$ system,
where the $\beta\gamma$ sector is valued in a vector space $V$, as we introduced in Section~\ref{sec:classical}.
Let $\Obs^\q_{free}$ denote the observables of this theory on $\Sigma = \CC$.
As a quantization of a free field theory, it is a factorization algebra valued in the category of $\CC [\hbar]$-modules.
In particular, the associated vertex algebra $\Vert(\Obs^\q_{free})$ is also valued in $\CC[\hbar]$-modules.

\begin{prop}\label{prop: bcbg vertex}
Let $n = \dim_{\CC}(V)$. Then, there is an isomorphism of vertex algebras
\ben
\Vert(\Obs^{\q}_{free})_{\hbar = 2 \pi i} \cong \cV_{bc} \tensor \cV_{\beta\gamma}^{\tensor n} 
\een 
where on the left-hand side we have set $\hbar = 2\pi i$.
\end{prop}

\subsection{The case of the string}

The holomorphic bosonic string is a chiral CFT and so the machinery of~\cite{CG1} applies to it.
One can extract a vertex algebra directly by this method.

But there is a slicker approach, using Li's work~\cite{LiVA},
which studies chiral deformations of {\em free} chiral BV theories such as the $bc\beta\gamma$ system.
Recall that a deformation of a classical field theory is given by a local functional. 
We have seen that this is essentially the data of a Lagrangian density, which is a density valued multilinear functional that depends on (arbitrarily high order) jets of the fields. 
In other words, for a field $\varphi$, a Lagrangian density is of the form
\ben
\cL(\varphi) = \sum (D_{k_1} \varphi) \cdots (D_{k_m} \varphi) \cdot {\rm vol}_\Sigma
\een 
for $C^\infty(\Sigma)$-valued differential operators $D_{k_i}$.
By a {\em chiral} Lagrangian density we mean a Lagrangian for which the differential operators $D_{k_i}$ are all holomorphic. 
For instance, on $\Sigma = \CC$, we require $D_{k_i}$ to be a sum of operators of the form $f(z) \partial_z^n$ where $f(z)$ is a holomorphic function. 
On $\Sigma = \CC$ we will also require the chiral Lagrangian to be translation invariant. 
This means that all differential operators $D_{k_i}$ are of the form $\partial_z^n$. 
Thus, a {\em translation-invariant chiral deformation} is a local functional of the form
\ben
I(\varphi) = \sum \int (\partial^{k_1}_z \varphi) \cdots (\partial^{k_m} \varphi) \d^2 z .
\een

One of Li's main results is that for a free chiral BV theory with action $S_{\rm free}$ and associated vertex algebra $\cV_{\rm free}$, one has the following:
\begin{itemize}
\item For any chiral interaction~$I$, the action $S_{\rm free} + I$ needs no counterterms, 
and yields a renormalized interaction~$\{S [L]\}$.
\item If the renormalized action $\{S [L]\}$ satisfies the quantum master equation,
then it determines a vertex algebra derivation $D_I$ of~$\cV_{\rm free}$ of the form
\ben
D_I = \oint I^q \d z
\een
of cohomological degree one, where $I^q = \lim_{L \to 0} I[L]$.
\item The dg vertex algebra $\cV_I$ for such an action $\{I[L]\}$ has the same underlying graded vertex algebra $\cV_{\rm free}$ but it is equipped with the differential $\oint I^q \d z$. 
\end{itemize}

\begin{rmk} The fact that $I$ satisfies the quantum master equation implies that one has a map, for each open set $U \subset \CC$, from the free factorization algebra evaluated on $U$ to the factorization algebra of the deformed theory evaluated on $U$:
\ben
e^{I /\hbar} : \Obs^q_{free}(U) \to \Obs^q_I (U) .
\een
This map sends an observable $O \in \Obs^q_{free}(U)$ to $O \cdot e^{I/\hbar}$. 
In fact, this map is an isomorphism with inverse given by $O \mapsto O \cdot e^{-I/\hbar}$. 
So, open by open, the factorization algebra assigns the same vector space for the deformed theory.
This isomorphism is {\em not} compatible with the factorization product, so we do get a different factorization algebra in the presence of a deformation.
\end{rmk}

The holomorphic bosonic string with target $V=\CC^{13}$ provides a concrete example of this situation.
The free theory is the $bc\beta\gamma$ system, 
and we have seen that the renormalized action satisfies the QME.
Hence we obtain the following.

\begin{prop} \label{prop: fact is vert}
Let $\Obs^\q_{\rm string}$ be the factorization algebra on $\Sigma = \CC$ of the holomorphic bosonic string with target~$V = \CC^{13}$. 
Let $\Vert(\Obs^q_{\rm string})$ be the dg vertex algebra (defined over $\CC[\hbar]$) obtained via Li's construction. 
There is an isomorphism of vertex algebras $\cV_{\rm string} \cong \Vert(\Obs^q_{\rm string})_{\hbar = 2 \pi i}$.
Moreover, this vertex algebra is isomorphic to the chiral sector of the bosonic string as in Section~\ref{subsec: string vert}.
\end{prop}

The factorization algebra $\Obs^\q_{\rm string}$ is a quantization of the factorization algebra $\Obs^{\cl}_{\rm string}$ of classical observables of the free $bc\beta\gamma$ system.
We have noted that the classical observables of any theory has the structure of a $P_0$ factorization algebra, and the $\hbar \to 0$ limit of $\Obs^\q_{\rm string}$ is isomorphic to $\Obs^{\cl}_{\rm string}$ as $P_0$ factorization algebras.
By definition, the classical observables are simply functions on the solutions to the classical equations of motion.
The $P_0$ structure is induced from the symplectic pairing of degree $(-1)$ on the fields. 
The classical factorization algebra still has enough structure to determine a vertex algebra $\Vert(\Obs^\cl_{\rm string})$.
Moreover, the $P_0$ bracket on the classical observables determines the structure of a {\em Poisson vertex algebra} on $\Vert(\Obs^{\cl}_{\rm string})$. 

\begin{cor} In the classical limit, there is an isomorphism of Poisson vertex algebras~$\Vert(\Obs^{\cl}_{\rm string}) \cong {\rm Gr} \; \cV_{\rm string}$.
\end{cor}

\begin{proof}[Proof of Proposition \ref{prop: fact is vert}] By Proposition \ref{prop: bcbg vertex} we know that the vertex algebra of the associated free theory is identified with the $bc\beta\gamma$ vertex algebra. 
The thing we need to check is that the differential induced from the quantization of the holomorphic string agrees with the differential of the string vertex algebra. 
In fact, we observe that the induced differential $\oint I \d z$ from the classical interaction of the holomorphic bosonic string agrees with the BRST charge in Equation (\ref{brst}). 
To see that this persists at the quantum level we need to check that there are no quantum corrections. 
Indeed, this follows from the fact that the quantum master equation holds identically (as opposed to holding up to an exact term in the deformation complex) provided $\dim_\CC V = 13$. 
\end{proof}

%\brian{remark about semi infinite cohomology}

%\subsection{Relation to semi-infinite cohomology}

%\brian{O to take a crack at this}

\subsection{The $E_2$ algebra and descent}

In this section we highlight a remarkable feature of the vertex algebra associated to the bosonic string. 
At first glance, the theory we have constructed is far from being topological.
Indeed, the classical theory depends delicately on the complex structure of the two-dimensional source. 
Nevertheless, the local observables of the bosonic string behave like the observables of a {\em topological} field theory (TFT). 
In particular, it was discovered that the observables of a 2-dimensional TFT have the structure of a {\em Gerstenhaber algebra} \cite{Getzler,LZ1}.
In this section we provide two equivalent methods for extracting this algebra.
The first is intuitive from the point of view of factorization algebras, but has the disadvantage of not giving a concrete description of the algebra. 
The second approach gives an explicit formula for bracket and is based on the formalism of ``descent" for local operators. 

\subsubsection{First method: the $E_2$ algebra}

We continue to consider the theory on the Riemann surface $\Sigma = \CC$. 
In this section we show how to produce, from the point of view of factorization algebras, the structure of a Gerstenhaber algebra on the BRST cohomology of the bosonic string. 

Recall that a Gerstenhaber algebra is equivalent to an algebra over the homology of the little 2-disk operad.
Hence, our approach is to see why the factorization algebra naturally exhibits the structure of a algebra of little 2-disks.
Here we use an important result of Lurie \cite{Lurie}: 
a {\em locally constant} factorization algebra on $\RR^n$ is equivalent to an algebra over the little $n$-disks operad, i.e., an $E_n$-algebra. 

\begin{prop} 
The factorization algebra $\Obs^\q_{\rm string}$ is locally constant, 
and hence it determines an $E_2$ algebra.
\end{prop}

In particular, the cohomology $H^*(\Obs^\q_{\rm string})$ is an algebra over the cohomology of the $E_2$ operad and hence a Gerstenhaber algebra.

\begin{rmk}
When a topological field theory arises from an action functional (e.g., Chern-Simons theories),
the factorization algebra is locally constant.
Hence such a TFT in $n$ real dimensions produces an $E_n$-algebra, by Lurie's result. 
(This claim holds true, at least, for all the examples with which we are familiar.)
In this sense, holomorphic bosonic string theory is a 2-dimensional topological field theory. 
Moreover, by work of Scheimbauer \cite{Scheim},
every $E_n$ algebra determines a fully-extended framed n-dimensional TFT in the functorial sense, albeit with values in an unusual target $(\infty,n)$-category.
In this sense, at least, the holomorphic bosonic string determines a functorial 2-dimensional TFT.
\end{rmk}

\begin{proof} 
We need to show that for any inclusion of open disks $D \hookrightarrow D'$ that natural map
\ben
\Obs^\q_{\rm string}(U) \to \Obs^\q_{\rm string}(U')
\een
is a quasi-isomorphism. 

We first show that the classical observables are locally constant. 
We have already mentioned that the classical observables are the commutative algebra of functions on the space of solutions to the classical equations of motion. 
This space of solutions forms a sheaf on $\Sigma$, 
since satisfying a PDE is a local condition.
We find it convenient to encode the equations of motion as the Maurer-Cartan equation of a sheaf  of dg Lie algebras:
\ben
\Omega^{0,*}(\Sigma ; \cT_\Sigma) \ltimes \left(\Omega^{0,*}(\Sigma; V)[-1] \oplus \Omega^{1,*}(\Sigma;V^*)[-1] \oplus \Omega^{1,*}(\Sigma ; \cT_\Sigma^*)[-2] \right) . 
\een
(Note that the underlying graded space is simply the fields shifted up by one degree,
which is a generic phenomenon in the BV formalism.)
The dg Lie algebra $\Omega^{0,*}(\Sigma ; \cT_\Sigma)$ is simply a sheaf-theoretic resolution of holomorphic vector fields, with the usual Lie bracket.
Our large dg Lie algebra is a square-zero extension of $\Omega^{0,*}(\Sigma ; \cT_\Sigma)$, 
by the dg module inside the parentheses.
The vector fields act by the Lie derivative on 
\[
\Omega^{0,*}(\Sigma; V)[-1] \oplus \Omega^{1,*}(\Sigma;V^*)[-1] \oplus \Omega^{1,*}(\Sigma ; \cT_\Sigma^*)[-2],
\]
which are simply a copy of the $\beta\gamma$ system with target vector space~$V$.
 
For simplicity, we write $\cL = \Omega^{0,*}(\Sigma ; \cT_\Sigma)$ and write $\cM$ for the module inside the parentheses.
In this language, the space of classical observables supported on an open set $U \subset \Sigma$ is the Chevalley-Eilenberg cochain complex
\ben
\Obs^{\cl}_{\rm string}(U) = \clie^*\left(\cL(U) \ltimes \cM(U)\right) = \clie^*\left(\cL(U) ; \; \Sym(\cM(U)^*[-1])\right),
\een
where $\cM(U)^*$ denotes the continuous linear dual of~$\cM(U)$. 

Consider now $U = D$, a disk centered at zero. 
By the $\dbar$-Poincar\'{e} lemma 
there is a quasi-isomorphism of dg Lie algebras $\cT_{hol}(D) \hookrightarrow \cL(D)$ where $\cT_{\rm hol}(D)$ is the vector space of holomorphic vector fields on $D$. 
Thus, we have a quasi-isomorphism
\[
\clie^*\left(\cT_{hol}(D) ; \; \Sym(\cM(D)^*[-1])\right) \simeq \Obs^{\cl}_{\rm string}(D).
\]
This quasi-isomorphism clearly holds for any disks, so it suffices to check that the left-hand side is a quasi-isomorphism for an inclusion of disks.

Consider the composition of Lie algebras
\ben
{\rm W}_1^{\rm poly} \hookrightarrow \cT_{hol} (D) \to {\rm W}_1
\een
where ${\rm W}_1^{\rm poly}$ are the holomorphic vector fields with {\em polynomial} coefficients, and ${\rm W}_1$ is the Lie algebra with {\em power series} coefficients (i.e., formal vector fields).
We will compare Lie algebra cohomology using these different Lie algebras.

Let $\cA(D)$ denote $\Sym(\cM(D)^*[-1])$.
It determines a module over ${\rm W}_1^{\rm poly}$ by restriction,
which we will abusively denote $\cA(D)$ as well.
Likewise, if $j_0^\infty \cM$ denotes the infinite jet of the sheaf $\cM$ at the origin of the disk $D$,
then it determines a natural module over ${\rm W}_1$.
Then $\Sym(\cM(D)^*[-1])$ determines a ${\rm W}_1$-module that we will also abusively denote by~$\cA(D)$.
 
The inclusion $D \hookrightarrow D'$ then yields a commutative diagram
\ben
\xymatrix{
\clie^*\left({\rm W}_1^{\rm poly} ; \cA(D)\right) \ar[d] & \ar[l] \clie^*\left(\cT_{\rm hol} (D) ; \cA(D))\right) \ar[d] & \ar[l]  \clie^*\left({\rm W}_1 ; \cA(D)\right) \ar[d] \\
\clie^*\left({\rm W}_1^{\rm poly} ; \cA(D')\right) & \ar[l] \clie^*\left(\cT_{\rm hol} (D') ; \cA(D')\right) & \ar[l] \clie^*\left({\rm W}_1 ; \cA(D')\right) .
}
\een
By Lemma \ref{lem: gf} (and an analogous result for polynomial vector fields),
the complexes $\clie^*({\rm W}_1 ; \cM)$ and $\clie^*({\rm W}^{\rm poly}_1; \cM)$ are quasi-isomorphic to the subcomplex of conformal dimension zero element, in this case, to the constants. 
As the conformal zero subcomplex does not depend on the size of the disk, we conclude that vertical arrows on the outside of the commutative diagram are quasi-isomorphisms. 
It follows that the middle vertical arrow is as well, 
thus showing that $\Obs^{\cl}_{\rm string}(D) \to \Obs^{\cl}_{\rm string}(D')$ is a quasi-isomorphism, as desired. 

To finish the proof, we need to prove the quasi-isomorphism for {\em quantum} observables.
Consider the spectral sequence induced from the filtration of the module~$\Sym \;\cM(D)$ by symmetric polynomial degree. 
The $E_1$ page of this spectral sequence is the classical observables above, 
and it converges to the cohomology of the quantum observables. 
As the map of factorization algebras induced by the inclusion $D \hookrightarrow D'$ preserves this filtration, 
we obtain a map of spectral sequences,
which is quasi-isomorphism on the first page.
Hence, $\Obs^{\q}_{\rm string}(D) \to \Obs^\q_{\rm string}(D')$ is also a quasi-isomorphism. 
\end{proof}

\subsubsection{The stress-energy tensor}

In \cite{WittenTop}, where the notion of a TFT was introduced,
Witten characterized a topological field theory
as a theory whose stress-energy tensor is (homotopy) trivial. 
We now verify that property of the holomorphic bosonic string.
That is, we want to show that the translation symmetries of the holomorphic bosonic string act trivially on the cohomology of the observables.

As a first step, consider the action of the differential operators $\frac{\d}{\d z}$ and $\frac{\d}{\d \zbar}$ on the Dolbeault complex $\Omega^{0,*}(\CC)$. 
This action extends to an action on the fields of the holomorphic bosonic string, and hence to their classical observables as well. 
By Noether's theorem any symmetry of a theory determines classical observables: 
for these symmetries, these are simply the $zz$ and $\zbar \zbar$ components of the stress-energy tensor $T_{zz}$, $T_{\zbar \zbar}$. 
In the case of the bosonic string, we will now show that the stress-energy tensor is cohomologically trivial on the quantum observables.
(Similar but simpler arguments apply to the classical case.)
%This property is, in fact, the general definition of a topological theory given in \cite{WittenTop}.

For each open $U \subset \CC$, the differential operators lift to cochain maps on the quantum level
\ben
\frac{\d}{\d z} , \frac{\d}{\d \zbar} : \Obs^\q_{\rm string} (U) \to \Obs^\q_{\rm string}(U) 
\een 
because the BV Laplacian is translation-invariant.
These cochain maps intertwine with the structure maps of the factorization algebra 
in the sense that they define {\em derivations} of the factorization algebra. 
(See Definition 7.3.2 of \cite{CG1} for a discussion of this notion.)
Note that these operators preserve the cohomological degree. 

Consider now the operator 
\ben
\Bar{\eta} = \frac{\partial}{\partial (\d \zbar)} 
\een 
acting on Dolbeault forms. 
This operator $\Bar{\eta}$ extends to a derivation of degree $-1$ on the factorization algebra $\Obs^\q_{\rm string}$. 
It satisfies the relation
\be\label{d/dzbar}
[\dbar + \hbar \Delta + \{I^\q, -\} , \Bar{\eta}] =  \frac{\d}{\d \zbar}
\ee
as endomorphisms of the factorization algebra, as we now explain.
One observes first that $[\dbar, \Bar{\eta}] =  \frac{\d}{\d \zbar}$. 
Moreover, since $I^\q$ is a chiral deformation, we also have $\Bar{\eta} \cdot I^\q = 0$. 
Finally, since the pairing defining the $-1$-shifted symplectic structure is holomorphic, 
we see that $\Bar{\eta}$ also commutes with the BV Laplacian $[\Bar{\eta}, \Delta] = 0$. 
Hence we have shown the following, by relation~(\ref{d/dzbar}).

\begin{lem}
The operator $\frac{\d}{\d \zbar}$ acts homotopically trivial on $\Obs^\q_{\rm string}$. 
\end{lem}

This fact ensures that the stress-energy tensor vanishes in the $\zbar\zbar$ direction.
We now turn to~$\d/\d z$.

We can view the vector field $\frac{\d}{\d z}$ as a constant $c$-field. 
Consider the linear local functional of cohomological degree $-2$:
\ben
O_{\frac{\d}{\d z}}(\beta,\gamma,b,c) = \int \<b, \frac{\d}{\d z}\>,
\een
It only depends on the $b$-field.
Note that for this integral to be nonzero, 
we must have $b \in \Omega^{1,1}(\Sigma , T_\Sigma^{1,0*})$. 
Using the BV bracket, we obtain a derivation of the factorization algebra 
\ben
\eta = \{O_{\frac{\d}{\d z}}, -\}
\een
of cohomological degree~$-1$. 

%Recall that the $b$-fields of the bosonic string are concentrated in degree $1$ and $2$. 
%In degree $1$ a $b$-field has the form 
%\[
%f(z,\zbar) \d z^{\tensor 2} \in \Omega^{1,0}(\Sigma, T_\Sigma^{1,0*}).
%\] 
%For any open $U \subset \CC$ containing the origin, consider the linear observable
%\ben
%b_{-1}(f(z,\zbar) \d z^{\tensor 2})= f(0),
%\een 
%which vanishes on all fields besides the degree $1$ component of the $b$-fields.
%This observable $b_{-1}$ is a closed element of degree $-1$ in $\Obs^\q_{\rm string}(U)$. 
%Given any other cochain-level observable $O \in \Obs^{\q}_{\rm string}(U)$,
%we define the observable $b_{-1} \cdot O$, since the underlying cochain complex has the structure of a symmetric algebra.

\begin{lem} 
The derivation $\eta$ satisfies 
\be\label{d/dz}
[\dbar + \hbar \Delta + \{I^\q, -\}, \eta] = \frac{\d}{\d z}.
\ee
\end{lem}
\begin{proof}
The derivation $\eta$ commutes with both $\dbar$ and $\Delta$. 
Thus, the left-hand side is 
\[
[\{I, -\}, \eta] = \{ \{I, O_{\frac{\d}{\d z}}\}, -\}.
\]
The only part of the interaction that contributes is $\int \<\beta, c \cdot \gamma\> + \int \<b, [c,c]\>$, and one computes that
\ben
\{I, O_{\frac{\d}{\d z}}\} = \int \<\beta, \partial_z \gamma\> + \int\<b, [\partial_z, c]\> .
\een
Bracketing with this local functional is precisely $\frac{\d}{\d z}$, as desired.
\end{proof}

Together these two lemmas ensure that translations act trivially on the cohomological observables.

\subsubsection{A local system of observables}

We'd like to sketch an important consequence of the work above.
As we will see, it gives an approach to the method of descent.

First, since the factorization algebra is locally constant, 
we can extract from it a local system (in a dg sense).
Note that for each point $x \in \CC$,
the {\em local observables around $x$} is a well-defined notion:
for any disk $D$ containing $x$, the observables $\Obs^\q_{\rm string} (D)$ are quasi-isomorphic by construction.
We let 
\[
\sO{\rm bs}_x = \lim_{x \in D} \Obs^\q_{\rm string} (D).
\]
We let $\sO{\rm bs} \to \CC$ denote the associated dg vector bundle (of infinite rank). 
Alternatively, we can consider the dg vector bundle whose fiber over every point is simply the global observables $\Obs^\q_{\rm string} (\CC)$.
Via the structure maps of $\Obs^\q_{\rm string}$, 
these two dg vector bundles are quasi-isomorphic.

This quasi-isomorphism also explains in what sense we have a dg local system:
for any two points, there is a zigzag of quasi-isomorphisms between their fibers,
passing through the global observables.
In other words, we know how to do ``transport up to quasi-isomorphism.''
This description is, however, rather indirect.

Our analysis of translations provides another approach to the local system.
Note that the local observables at every point are explicitly isomorphic under macroscopic translations.
That is, if $\tau_z: \CC \to \CC$ is the translation sending $w$ to $w+z$,
then there is a natural isomorphism between $\Obs^\q_{\rm string}$ and the pushforward $(\tau_z)_* \Obs^\q_{\rm string}$,
induced by pullback of fields along the translation.
This isomorphism induces an isomorphism of fibers
\[
\sO{\rm bs}_z \cong \sO{\rm bs}_0,
\]
by taking limits of isomorphisms between disks around those points.
In this way, we can identify smooth sections of $\sO{\rm bs}$ with elements of
\[
C^\infty(\CC) \otimes \sO{\rm bs}_0.
\]
But we also know how $\d/\d z$ and $\d/\d\zbar$ act.
In short, we have the following.

%We seemed to believe the trivializations also played a role but can't quite pin down a clear statement about what role they play

\begin{cor} 
The vector bundle $\sO{\rm bs}$ has a natural flat connection on~$\CC$.
\end{cor}

%This local system gives us yet another method for producing observables of the theory.
%Given a compactly-supported de Rham form with values in $\sO{\rm bs}$
%\[
%\widetilde{O} = \widetilde{O}_0 + \widetilde{O}_z \d z + \widetilde{O}_{\zbar} \d\zbar + \widetilde{O}_{z\zbar} \d z\, \d\zbar \in \Omega^*_c(U,\sO{\rm bs}),
%\]
%it can be viewed as determining an element of $\Obs^\q_{\rm string}(U)$ by integration over $U$.
%At the classical level this identification is particularly clear:
%we simply evaluate $\widetilde{O}_{z\zbar} \d z\, \d\zbar$ on a field, producing a top form, and then integrating.

We now explain the method of ``descent" for local observables using this framework. 
Expositions of this construction as related to two-dimensional gravity can be found in \cite{WittenDescent,Dijk}.
The basic idea is the following.
If $O$ is a local observable that is closed for the quantum differential, 
we can use the trivializations of translation to promote $O$ to a {\em non-local} observable supported on any closed submanifold of the Riemann surface. 

\begin{construction} 
Given any local observable $O \in \sO{\rm bs}_0$, the relations (\ref{d/dzbar}) and (\ref{d/dz}) allow us to define a differential form-valued observable $\Tilde{O}$ as follows. 
The 0-form part $\Tilde{O}^{0}$ is just the observable $O$,
viewed as a constant section over $\CC$. 
The 1-form part $\Tilde{O}^1$ is 
\ben
\d z \; (\eta O) + \d \zbar \; (\Bar{\eta} O),
\een
where, for instance, $\eta O$ denotes the image of $O$ under the map $\eta$.
(This form is constant over $\CC$ as well.)
The 2-form part is 
\[
\Tilde{O}^2 = \d z \,\d \zbar\, \eta \Bar{\eta} O.
\] 
A straightforward calculation shows that if $O$ is closed for the quantum differential $\d^\q = \dbar + \hbar \Delta + \{I^\q, -\}$, 
then 
\[
\Tilde{O} = \Tilde{O}^{0} + \Tilde{O}^1 + \Tilde{O}^2 
\]
satisfies $(\d_{dR} + \d^\q) \Tilde{O} = 0$. 
This relationship implies that for any $\d^\q$-closed observable $O$ and for any closed submanifold $C \subset \Sigma$, 
we obtain a cocycle in the observables via
\ben
\int_C \Tilde{O} \in \Obs^\q(C) .
\een
If $O$ has cohomological degree $k$ and $C$ is of dimension $l$, 
then $\int_C \Tilde{O}$ has degree $k - l$. 
\end{construction}

%\begin{rmk}
%{\it A priori} the factorization algebra is only well-defined on open sets $U \subset \Sigma$. 
%One can define the value on a closed submanifold $C \subset \Sigma$ by taking the (homotopy)  limit over open submanifolds containing~$C$. 
%\end{rmk} 

This construction can be understood in a different way:
we have described a ``horizontal/flat section" map 
\ben
\Obs^\q_{\rm string}(\CC) \simeq \sO{\rm bs}^\q_0 \to \Omega^*(\CC , \sO{\rm bs})
\een
that sends a closed observable $O$ to the $(\d_{dR} + \d^\q)$-closed differential form-valued observable~$\Tilde{\sO}$. 

We now put this construction to use in understanding the Gerstenhaber structure concretely.

%\begin{proof} 
%Both $\dbar$ and $\Delta$ commute with $\eta$, by direct computation. 
%Thus, it suffices to show that $[\{I^\q, -\}, \eta] = \frac{\d}{\d z}$. 
%Consider the observable 
%\[
%O_{\gamma,0}(\gamma) = \gamma (0),
%\] 
%which we can write as the residue $O_{\gamma,0} = \oint \frac{\gamma(z)}{z}$. 
%The only part of the interaction that contributes is $\int \<\beta, c \cdot \gamma\>$ and 
%\ben
%[\{\int \<\beta, c \cdot \gamma\> , -\}, \eta] O_{\gamma, 0} = \{\int \<\beta, \partial_z \gamma\>, O_{\gamma,0}\} = \oint \frac{\partial_z \gamma}{z} \d z = \frac{\d}{\d z} O_{\gamma,0} .
%\een
%The calculation is similar for a general linear observable. 
%To get the result for an arbitrary observable, we use the fact that the bracket is a derivation. 
%%\brian{I tried to come up with a more conceptual proof by interpreting $\{I,-\}$ as the Chevalley-Eilenberg differential for vector fields, but failed.}
%\end{proof}

\subsubsection{Second method: descent}

A Gerstenhaber algebra is a graded commutative algebra with a Lie bracket of cohomological degree $-1$ that is a graded biderivation for the product. 
In this section we show how to explicitly write down the product and bracket on the local observables (i.e., the observables on any disk) and compare our answer to the work of Lian-Zuckerman \cite{LZ1}.

The product of the Gerstenhaber algebra is simply given by the factorization product of two disjoint disks including into a larger disk:
\ben
\cdot : \Obs^\q(D) \tensor \Obs^\q(D') \to \Obs^\q(D'') .
\een 
At the level of cohomology, this product is commutative by the Eckmann-Hilton argument,
as we can just move one disk around the other.

To construct the shifted Poisson bracket, we use descent.
Suppose $O,O'$ are closed observables on disks $D,D'$, respectively.
For simplicity, we assume that the closure of $D$ is strictly contained in $D'$. 
If $C = \partial D'$, we can define $\int_C \Tilde{O}'$ as above. 
Finally, for $D''$ another disk containing $D$ and $C$, 
consider the factorization product $\star : \Obs^\q(D) \tensor \Obs^\q (C) \to \Obs^\q(D'')$ and define
\ben
\{O,O'\}_{\rm Ger} := O \star \int_C \Tilde{O}' \in \Obs^\q (D'') .
\een 
We note that if ${\rm deg}(O) = k$ and ${\rm deg}(O') = k'$, then ${\rm deg}(\{O,O'\}_{\rm Ger}) = k+k' -1$. 

Let $V$ denote the cohomology $H^* \cV_{\rm string}$ of the dg vertex algebra $\cV_{\rm string}$.

\begin{prop} 
The bracket $\{-,-\}_{\rm Ger}$ together with the product $\cdot$ determine the structure of a Gerstenhaber algebra on $V$. 
This Gerstenhaber structure is isomorphic to the one found by Lian-Zuckerman~\cite{LZ1}.
\end{prop}

\begin{proof}
The vertex algebra construction of \cite{CG1} extracts $\cV_{\rm string}$ as the direct sum of the weight spaces of $\Obs^\q_{\rm string}(D)$, 
where $D$ is a disk centered at the origin and we take weight space for the rotation action of $S^1$ on $\CC$.
The bracket and product restrict to this subspace of $\Obs^\q(D)$,
manifestly playing nicely with this eigenspace decomposition. 
Hence they descend to the cohomology $V$ of~$\cV_{\rm string}$.

Let $V_{LZ}$ be the Gerstenhaber algebra considered by Lian-Zuckerman.
As vector spaces, both $V$ and $V_{LZ}$ are isomorphic to the state space of the $\beta\gamma$ vertex algebra.

According to the construction of a vertex algebra from a holomorphic factorization algebra in Chapter 6 of \cite{CG1}, the factorization product of two disks is what defines the operator product map $Y(-,z) : V \tensor V \to V ((z))$ of the vertex algebra.
It is this operator product that Lian-Zuckerman use to define the commutative product.
Thus, as commutative algebras, the algebras coincide. 

The brackets coincide by noting that the derivation $\eta$ trivializing $\d / \d z$ agrees with Lian-Zuckerman's trivialization.
\end{proof}
%\brian{Should we prove this or just argue that the bracket and product above agree with what Lian-Zuckerman get.}

