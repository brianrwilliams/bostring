\section{From classical to quantum: anomalies in the BV formalism}

\brian{A rapid overview of classical BV and effective quantizations. Stress how obstructions appear, where they live, and how to compute them.}


\owen{I think we should articulate here the structural features of our BV package that make the arguments below more conceptual. For instance:
\begin{itemize}
\item Linear BV quantization is determinantal, which explains why we'll produce determinant line bundles when we do free $\beta\gamma$ system.
\item ``Gauging'' a theory corresponds to a stacky quotient of the original fields. Hence, obstruction to quantizing a gauged theory corresponds to descending the quantization to the quotient. \brian{Possibly also add the mechanism in which antifields and antighosts are introduced.} 
\item If a classical theory makes sense on a class (=site) of manifolds, then to quantize the whole class, it suffices to check on a generating cover (typically given by disks with geometric structure) but compatibly with all automorphisms. This often explains the appearance of characteristic classes as anomalies.
\item Every BV theory produces a factorization algebra. The local structure encodes the OPE algebra (and hence recovers a vertex algebra in chiral CFT situation). On compact manifolds, solutions to EoM typically form finite-dimensional space, and the global observables encode a volume form on this space. (An example is conformal blocks for the free $bc\beta\gamma$ system.)
\end{itemize}
Please add others as you think of them!}

\owen{We might also add that we view the BV formalism as the analogue in field theory of derived geometry in geometry. That is, in ordinary algebraic geometry, one first builds geometry and then adds (sheaf) cohomology on top: in ordinary physics, one first builds field theories and then adds (BRST) cohomology on top. But derived geometry (respectively, BV formalism) builds the cohomological aspect into the foundations.}

\subsection{\owen{description of algorithm}}
\label{sec:bvalgorithm}

For us, quantization will mean that we use perturbative constructions in the setting of the BV formalism.
Concretely, this means that we enforces the gauge symmetries using the homological algebra of the BV formalism 
and that we use Feynman diagrams and renormalization to obtain an expression for the desired, putative path integral. 
\owen{Be more careful about saying path integral. It's an approximation.}
There are toy models for this approach where one can see very clearly how it gives asymptotic expansions for finite-dimensional integrals \owen{add references}.
In particular, these toy models show that this approach need not recover the true integral
but does know important information about it;
a similar relationship should hold between this quantization method and the putative path integral, 
but in this case there is no {\em a priori} definition of the true integral in most cases.

This notion of quantization applies to any field theory arising from an action functional,
and the algorithm one applies to obtain a quantization is the following:
\begin{enumerate}
\item Write down the integrals labeled by Feynman diagrams arising from action functional.
\item Identify the divergences that appear in these integrals and add ``counterterms'' to the original action that are designed to cancel divergences.
\item Repeat these steps until no more divergences appear in Feynman diagrams.
We call this the ``renormalized action.''
\item Check if the renormalized action satisfies the quantum master equation. 
If it does, you have a well-posed BV quantum theory, and we call the result a {\em quantized action}. If not, guess a way to adjust the renormalized action and begin the whole process again.
\end{enumerate}
It should be clear that along the way, one makes many choices;
hence if a quantization exists, it may not be unique.
It is also possible that a BV quantization may not exist.
