\section{Introduction}
%\section{From classical to quantum: anomalies in the BV formalism}

We have two intertwined goals here.
First, we want to describe a two-dimensional field theory
and its perturbative quantization,
which we'll see encodes the chiral sector of a bosonic string with linear target space.
We view this theory as a holomorphic version of bosonic string theory.
Second, we want to use this theory as the running example for key ideas and techniques in the formalism for quantum field theory developed by Costello and collaborators \cite{CosBook, CG1,CG2, others}.
None of the results here about string theory are new, 
as the bosonic string has been under intensive study for several decades,
but this formalism recovers them in a single, systematic process,
often giving a novel argument or perspective.
It is compelling to have a direct path from the action functional to such sophisticated constructions as the semi-infinite cohomology of a vertex algebra.

Our focus is thus on narrative rather than detailed argumentation.
That is, we work systematically in the natural order of the formalism and along the way we emphasize the motivations behind each step rather than the nitty-gritty computations. 
Precedence is given to communicating the essence of an argument, over spelling everything out.
We do give detailed citations where such arguments can found in the literature,
and we defer some not-yet-extant details to a forthcoming work on this theory with curved target space~\cite{GWcurved}.

Given the vastness of the string theory literature,
it should not be surprise that there is already work along these lines,
notably by Losev, Marshakov, and Zeitlin \cite{LMZ}.
One could view this paper as attempting to communicate many of their insights to those with an intuition growing out of homotopical algebra and the functorial approach to geometry.
Again, we note that the formalism of Costello provides a mathematical articulation and verification of many ideas long known to physicists, such as the Wilsonian view of renormalization and the Batalin-Vilkovisky (BV) approach to gauge and gravity theories.\footnote{We also note that given the literature's size,
and our relative and unfortunate ignorance of much of it,
we have chosen to mention a reference when we feel its description is particularly useful for us, 
even if it is not the original or standard reference for a given result.}

As the bulk of the text is devoted to detailed exposition centered on our example,
we want to expound some key ideas of the Costello formalism so that the reader is alert to them when proceeding through the text.
That is, we wish to articulate here the structural features of this BV/renormalization package that make the arguments below conceptual.

\theoremstyle{definition}
\newtheorem{lesson}[thm]{Lesson}

For instance, in a gauge theory we know that connections provide the ``naive'' fields and that one must identify connections that are gauge-equivalent.
A mathematician would say the true fields are a {\em stacky} quotient of the naive fields.
Similarly, the critical locus of the action functional $S$ is the zero locus of its differential $\d S$ (ignoring some subtleties of the variational set-up),
which is the intersection of $\d S$ with the zero section of the cotangent space of the fields.
But in mathematics it is better to take {\em derived} intersections.

\begin{lesson}[Part 1, \cite{CG2}]
The classical BV formalism is a method for computing the derived critical locus of the action functional on the derived stack of fields.
Ghosts appear to describe the direction along which one quotients---the stacky direction---while the antifields appear to describe the direction along which one intersects---the derived direction.
\end{lesson}

We will describe our theory in the usual way, involving fields with varying ghost number, 
but we will also sketch its meaning in terms of global derived geometry,
which we find illuminating about the deep connections between string theory and algebraic geometry.

Path integral quantization amounts to trying to put a kind of measure or volume form on the derived stack of fields.
When the fields are linear in kind, 
there is a natural quantization that is translation-invariant along the fields,
which is the analogue of the Lebesgue measure on an ordinary vector space.

\begin{lesson}[\cite{GH}]
Linear BV quantization is functorial, and it behaves much like a determinant functor.
Hence, when one takes the fiberwise quantization of a family of linear theories,
one obtains determinant lines over the base.
\end{lesson}

This situation is relevant to us because the theory we study arises from a simple free theory,
the free $\beta\gamma$ system, which lives on any Riemann surface.
Hence the quantization of the free $\beta\gamma$ system makes sense over the moduli of Riemann surfaces and naturally produces a line bundle.

Our classical theory of interest arises by gauging the natural action of holomorphic vector fields on the free $\beta\gamma$ system.
One can say that we couple it to holomorphic gravity.
But then we recognize a natural consequence of our prior lessons.

\begin{lesson}[\S 5.11, \cite{CosBook}]
``Gauging'' a classical theory corresponds to a stacky quotient of the original fields. 
To quantize the gauged theory corresponds to descending the quantization to the quotient.
Hence, an anomaly that prevents quantization should be understood as an obstruction to descent.
\end{lesson}

The Costello formalism makes this relationship manifest, 
as the anomaly that appears in trying to produce a BV quantization---which is a Feynman diagram construction---is a cocycle in a dg Lie algebra determined by the classical field theory.
Thus, the anomaly determines an element of a natural Lie algebra cohomology group,
whose descent-theoretic meaning is typically easy to recognize.
Here we will discover the famed Weyl, or conformal, anomaly, which requires the target space to be real 26-dimensional. 

Anomalies are often characteristic classes, and this perspective offers a structural explanation.
Most classical field theories---at least most of broad interest---make sense on a class of manifolds,
and so the anomaly ought to be determined by the local geometry of this class.
In more mathematical language we have the following.

\begin{lesson}[\cite{GGW}]
If a classical theory determines a sheaf on some site of manifolds (such as the site of Riemann surfaces and local biholomorphisms), 
then to quantize the theory over the whole site, 
it suffices to check on a generating cover (typically given by disks with geometric structure) but compatibly with all automorphisms.
\end{lesson}

In particular, the BV anomaly is a cocycle for the Lie algebra of automorphisms of the {\em formal} disk equipped with such geometric structures.
In other words, it lives in some kind of Gelfand-Fuks cohomology, which gives deep and informative connections with foliation theory and topology.

So far, everything we have mentioned is well-known in field theory, 
albeit often expressed in a different dialect of mathematics.
We now turn to the main new notion in the formalism of Costello:
factorization algebras, which provide an efficient and powerful way to organize the local-to-global structure of the observables of a field theory.

\begin{lesson}[\cite{CG1,CG2}]
Every BV theory produces a factorization algebra. 
The local structure encodes the OPE algebra, so that for a chiral CFT, one recovers a vertex algebra. 
On compact manifolds, the global structure often has finite-dimensional cohomology because solutions to the equations of motion are typically finite-dimensional.
For a chiral CFT, one recovers the conformal blocks.
\end{lesson}

A technical result of \cite{CG1} gives a precise articulation of this lesson,
and we will apply it to identify the vertex algebra arising from our holomorphic version of the bosonic string.
%
%
%\subsection{\owen{description of algorithm}}
%\label{sec:bvalgorithm}
%
%For us, quantization will mean that we use perturbative constructions in the setting of the BV formalism.
%Concretely, this means that we enforces the gauge symmetries using the homological algebra of the BV formalism 
%and that we use Feynman diagrams and renormalization to obtain an expression for the desired, putative path integral. 
%\owen{Be more careful about saying path integral. It's an approximation.}
%There are toy models for this approach where one can see very clearly how it gives asymptotic expansions for finite-dimensional integrals \owen{add references}.
%In particular, these toy models show that this approach need not recover the true integral
%but does know important information about it;
%a similar relationship should hold between this quantization method and the putative path integral, 
%but in this case there is no {\em a priori} definition of the true integral in most cases.
%
%This notion of quantization applies to any field theory arising from an action functional,
%and the algorithm one applies to obtain a quantization is the following:
%\begin{enumerate}
%\item Write down the integrals labeled by Feynman diagrams arising from action functional.
%\item Identify the divergences that appear in these integrals and add ``counterterms'' to the original action that are designed to cancel divergences.
%\item Repeat these steps until no more divergences appear in Feynman diagrams.
%We call this the ``renormalized action.''
%\item Check if the renormalized action satisfies the quantum master equation. 
%If it does, you have a well-posed BV quantum theory, and we call the result a {\em quantized action}. If not, guess a way to adjust the renormalized action and begin the whole process again.
%\end{enumerate}
%It should be clear that along the way, one makes many choices;
%hence if a quantization exists, it may not be unique.
%It is also possible that a BV quantization may not exist.
