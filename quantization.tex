\section{Quantizing the holomorphic bosonic string on a disk} 

\brian{Gauge fixing condition. The theory is finite, no counterterms. Review Gelfand-Fuksy stuff. Local local deformation complex calculation. Do the anomaly calculation to obtain $\dim_{\CC} = 13$. Argue why this produces a quantization on any source Riemann surface.}

\owen{I'm not sure what is the optimal order: diagrams then cohomology group, or other way? If we do the concrete Feynman diagrams and show that something vanishes exactly when you'd expect from the literature, then people might feel assured. However, it's not clear where this anomaly lives until you do the obstruction-deformation computation. Perhaps we just indicate how the reader can pick her preferred order.}

For us, quantization will mean that we use perturbative constructions in the setting of the BV formalism.
Concretely, this means that we enforces the gauge symmetries using the homological algebra of the BV formalism 
and that we use Feynman diagrams and renormalization to obtain an expression for the desired, putative path integral. 
\owen{Be more careful about saying path integral. It's an approximation.}
There are toy models for this approach where one can see very clearly how it gives asymptotic expansions for finite-dimensional integrals \owen{add references}.
In particular, these toy models show that this approach need not recover the true integral
but does know important information about it;
a similar relationship should hold between this quantization method and the putative path integral, 
but in this case there is no {\em a priori} definition of the true integral in most cases.

This notion of quantization applies to any field theory arising from an action functional,
and the algorithm one applies to obtain a quantization is the following:
\begin{enumerate}
\item Write down the integrals labeled by Feynman diagrams arising from action functional.
\item Identify the divergences that appear in these integrals and add ``counterterms'' to the original action that are designed to cancel divergences.
\item Repeat these steps until no more divergences appear in Feynman diagrams.
We call this the ``renormalized action.''
\item Check if the renormalized action satisfies the quantum master equation. 
If it does, you have to 
\end{enumerate}

