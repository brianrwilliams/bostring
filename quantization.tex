\section{Quantizing the holomorphic bosonic string on a disk} 

\brian{Gauge fixing condition. The theory is finite, no counterterms. Review Gelfand-Fuksy stuff. Local local deformation complex calculation. Do the anomaly calculation to obtain $\dim_{\CC} = 13$. Argue why this produces a quantization on any source Riemann surface.}

\owen{I'm not sure what is the optimal order: diagrams then cohomology group, or other way? If we do the concrete Feynman diagrams and show that something vanishes exactly when you'd expect from the literature, then people might feel assured. However, it's not clear where this anomaly lives until you do the obstruction-deformation computation. Perhaps we just indicate how the reader can pick her preferred order.}

We will apply the algorithm described in Section~\ref{sec:bvalgorithm}
in the case of $\Sigma = \CC$ (or any open subset of the plane).
For this theory we are lucky, however:
the integrals that appear from the Feynman diagrams do not have divergences,
so that renormalized action is easy to compute.
This aspect is the subject of the first part of this section.
Moreover, it is easy to check whether the quantum master equation is satisfied,
and the answer is simple.
This aspect is the subject of the second part.
The results can be summarized as follows.

\begin{prop}
The holomorphic bosonic string with source $\CC$ and target $\CC^d$ admits a BV quantization
if $d = 13$.
This quantized action only has terms of order $\hbar^0$ and $\hbar$ (i.e., it quantizes at one loop).
\end{prop}

\subsection{The Feynman diagrams}

Let us describe the combinatorics of the Feynman diagrams that appear here
before we describe the associated integrals.

\subsubsection{}

The procedure is to construct graphs out of a prescribed type of vertices and edges;
we must consider all graphs with such local structure.
The classical action functional determines the allowed kinds of vertices and edges.
The quadratic terms of the action tell us the edges;
each quadratic term yields an edge whose boundary is labeled by the two fields appearing in the term.
For us there are thus two types of edges: 
an edge that flows from $c$ to $b$, 
and an edge that flows from $\gamma$ to~$\beta$.
\owen{Add picture.}
The nonquadratic terms tell us the vertices:
each $n$-ary term yields a vertex with $n$ legs,
and the legs are labeled by the $n$ types of fields appearing in the term.
For us there are thus two types of trivalent vertices:
a vertex with two $c$ legs and a $b$ leg, 
and a vertex with a $c$ leg, a $\gamma$ leg, and a $\beta$ leg.
It helpful to picture these legs as directed,
so that $c$ and $\gamma$ legs flow into a vertex
and $b$ and $\beta$ legs flow out.
\owen{Add picture.}

The kinds of graphs one can build with such vertices and edges are limited.
We focus on connected graphs.
A tree (i.e., a connected graph with no loops) must have at most one outgoing leg,
which can be either a $b$ or a~$\beta$;
the other legs are incoming, so each must be labeled by a $c$ or a~$\gamma$.
\owen{Add picture.}
A 1-loop graph will consist of a wheel (i.e., a sequence of edges that form an overall loop)
with trees attached.
\owen{Add pictures.}
Every edge along a wheel will have the same type.
It is not possible to build a connected graph with more than one loop.
This combinatorics is the essential reason that we can quantize at one loop.

\subsubsection{}

These graphs describe linear maps associated to the field.
More precisely, a graph with $k$ legs describes a linear functional on the $k$-fold tensor product of the space of fields.
One builds this linear functional out of the data of the action functional.

For instance, a $k$-valent vertex corresponds to a $k$-ary term in the action,
which manifestly takes in $k$ copies of the fields and outputs a number.
Thus, the vertex labels an element of a (continuous) linear dual of the $k$-fold tensor product of fields.
An edge corresponds an element of the 2-fold tensor product of the space of fields.
More accurately, one would like to associate to the edge 
the Green's function for the linear differential operator 
appearing in the associated quadratic term of the action,
hence an element of the {\em distributional completion} of the 2-fold tensor product.
Thus, for us the $\beta\gamma$ leg should be labeled by $\dbar^{-1} \otimes {\rm id}_V$,
where $\dbar^{-1}$ denotes an inverse to the Dolbeault operator on functions.
The $bc$ leg should be labeled by $\dbar^{-1}_T$, 
the inverse of the Dolbeault operator on the bundle~$T^{1,0}$.

Given a graph, one should contract the tensors associated to the vertices and edges.
In most situations this contraction is ill-defined, 
since the vertices label distributional sections of some vector bundle, 
the edges are likewise distributional,
and one cannot (usually) pair distributions.
Thus, in practice one chooses a mollification of the Green's functions. 


