\section{Quantizing the holomorphic bosonic string on a disk} 

\brian{Gauge fixing condition. The theory is finite, no counterterms. Review Gelfand-Fuksy stuff. Local local deformation complex calculation. Do the anomaly calculation to obtain $\dim_{\CC} = 13$. Argue why this produces a quantization on any source Riemann surface.}

We will apply the algorithm described in Section~\ref{sec:bvalgorithm}
in the case of $\Sigma~=~\CC$.
For this theory we are lucky, however:
the integrals that appear from the Feynman diagrams do not have divergences,
so that renormalized action is easy to compute.
This aspect is the subject of the first part of this section.
(Later we will explain why these divergences do not appear on an arbitrary Riemann surface.
\owen{add cross ref})
Moreover, it is easy to check whether the quantum master equation is satisfied,
and the answer is simple.
This aspect is the subject of the second part.
The results can be summarized as follows.

\begin{prop}
The holomorphic bosonic string with source $\CC$ and target $\CC^d$ admits a BV quantization
if $d = 13$.
This quantized action only has terms of order $\hbar^0$ and $\hbar$ (i.e., it quantizes at one loop).
\end{prop}

\subsection{The Feynman diagrams}

Let us describe the combinatorics of the Feynman diagrams that appear here
before we describe the associated integrals.

\subsubsection{}

The procedure constructs graphs out of a prescribed type of vertices and edges;
we must consider all graphs with such local structure.
The classical action functional determines the allowed kinds of vertices and edges.
The quadratic terms of the action tell us the edges;
each quadratic term yields an edge whose boundary is labeled by the two fields appearing in the term.
For us there are thus two types of edges: 
an edge that flows from $c$ to $b$, 
and an edge that flows from $\gamma$ to~$\beta$.
\owen{Add picture.}
The nonquadratic terms tell us the vertices:
each $n$-ary term yields a vertex with $n$ legs,
and the legs are labeled by the $n$ types of fields appearing in the term.
For us there are thus two types of trivalent vertices:
a vertex with two $c$ legs and a $b$ leg, 
and a vertex with a $c$ leg, a $\gamma$ leg, and a $\beta$ leg.
It helpful to picture these legs as directed,
so that $c$ and $\gamma$ legs flow into a vertex
and $b$ and $\beta$ legs flow out.
\owen{Add picture.}

The kinds of graphs one can build with such vertices and edges are limited.
We focus on connected graphs, since an arbitrary graph is just a union of connected components.

A tree (i.e., a connected graph with no loops) must have at most one outgoing leg,
which must be either a $b$ or a~$\beta$;
the other legs are incoming, so each must be labeled by a $c$ or a~$\gamma$.
\owen{Add picture.}
Note that there are two types of trees.
If there is a $\gamma$ leg, then there is a $\beta$ leg,
and there is a chain of $\gamma\beta$ edges connecting them;
all other external legs are of $c$~type.
If there is a $b$ leg, then the only other legs are $c$~type.

A 1-loop graph will consist of a wheel (i.e., a sequence of edges that form an overall loop)
with trees attached.
The outer legs are all of $c$~type.
\owen{Add pictures.}
Every edge along a wheel will have the same type.
It is not possible to build a connected graph with more than one loop.
This combinatorics is the essential reason that we can quantize at one loop.

We write ${\bf Graph}_{\rm string}$ for the collection of connected graphs just described,
namely the directed trees and 1-loop graphs allowed by the string action functional.
Let ${\bf Graph}_{\rm string}^{(0)}$ denote the 0-loop graphs (i.e., trees) and let ${\bf Graph}_{\rm string}^{(1)}$ denote the 1-loop graphs (i.e., wheels with trees attached).

See Figure \ref{fig:wheel} for a simple example.

\begin{figure}
\begin{tikzpicture}[decoration={markings,mark=at position 0.6cm with {\arrow[black,line width=.4mm]{stealth}}}];
\draw[postaction=decorate, line width=.2mm] (-0.7, 1) -- (0,0);
\draw[postaction=decorate, line width=.2mm] (0.7,1) -- (0,0);
\draw[postaction=decorate, line width=.2mm] (0,0) -- (0,-1);
\filldraw[color=black]  (0,0) circle (.1);
\draw (1.2,-0.1) node {$I_k = t^n \partial_t$};
\draw (-0.75,1.2) node {$c$};
\draw (0.75,1.2) node {$\gamma$};
\draw (0,-1.3) node {$\beta$};
\end{tikzpicture}
\caption{The trivalent vertex for $\int \langle \beta, c \gamma\rangle$}
\label{fig:vertex}
\end{figure}

%\begin{figure}
%\begin{tikzpicture}[decoration={markings,
%   mark=at position 1.2cm with {\arrow[black,line width=.4mm]{stealth}}}];
%\draw[postaction=decorate, line width=.2mm] (-4,0) -- (-2,0);
%\draw (-4.2,0) node {$\gamma$};
%\draw (-1.8,0) node {$\beta$};
%\draw[->,decorate,decoration=snake] (2,0) -- (4,0);
%\draw (1.8,0) node {$c$};
%\draw (4.2,0) node {$b$};
%\draw (0, 0.4) node {$P$};
%\end{tikzpicture}
%\caption{The propagators as directed from $\gamma$ to $\beta$}
%\label{fig:prop}
%\end{figure}

\begin{figure}
\begin{center}
\feynmandiagram [horizontal=a to b]{
a [particle=\(\gamma\)] -- [charged boson] b [particle=\(\beta\)]
};
\end{center}
\end{figure}

\feynmandiagram[horizontal=a to b]{
a -- [charged boson] b,
i1 -- [fermion] a -- [fermion] i2,
f1 -- [fermion] b -- [fermion] f2,
};

\begin{figure}
\begin{center}
\begin{tikzpicture}[scale=0.8,decoration={markings,mark=at position 0.9cm with {\arrow[black,line width=.4mm]{stealth}}}];
\draw[postaction=decorate, line width=.2mm] (2,0) -- (0,0);
\draw (1, -0.4) node {$P$};
\draw[postaction=decorate, line width=.2mm] (0,0) -- (0,2);
\draw (-0.4, 1) node {$P$};
\draw[postaction=decorate, line width=.2mm] (0,2) -- (2,2);
\draw (1, 2.4) node {$P$};
\draw[postaction=decorate, line width=.2mm] (2,2) -- (2,0);
\draw (2.4, 1) node {$P$};
\draw[postaction=decorate, line width=.2mm] (-1.4,0) -- (0,0);
\draw (-1.6,0) node {$\gamma$};
\draw[postaction=decorate, line width=.2mm] (0,-1.4) -- (0,0);
\draw (0,-1.6) node {$\gamma$};
\draw[postaction=decorate, line width=.2mm] (-1.12,-1.12) -- (0,0);
\draw (-1.3,-1.3) node {$\gamma$};
\draw[postaction=decorate, line width=.2mm] (-1.12,3.12) -- (0,2);
\draw (-1.3,3.3) node {$\gamma$};
\draw[postaction=decorate, line width=.2mm] (3.4,2.5) -- (2,2);
\draw (3.6,2.6) node {$\gamma$};
\draw[postaction=decorate, line width=.2mm] (2.5,3.4) -- (2,2);
\draw (2.6,3.6) node {$\gamma$};
\draw[postaction=decorate, line width=.2mm] (3.4,-0.5) -- (2,0);
\draw (3.6,-.6) node {$\gamma$};
\draw[postaction=decorate, line width=.2mm] (2.5,-1.4) -- (2,0);
\draw (2.6,-1.6) node {$\gamma$};
\filldraw[color=black]  (0,0) circle (.1);
\filldraw[color=black]  (0,2) circle (.1);
\filldraw[color=black]  (2,0) circle (.1);
\filldraw[color=black]  (2,2) circle (.1);
\end{tikzpicture}
\caption{A wheel with four vertices}
\label{fig:wheel}
\end{center}
\end{figure}


\subsubsection{}

These graphs describe linear maps associated to the field.
More precisely, a graph with $k$ legs describes a linear functional on the $k$-fold tensor product of the space of fields.
One builds this linear functional out of the data of the action functional.

As an example, a $k$-valent vertex corresponds to a $k$-ary term in the action,
which manifestly takes in $k$ copies of the fields and outputs a number.
Thus, the vertex labels an element of a (continuous) linear dual of the $k$-fold tensor product of fields.
In fact, one restricts to {\em compactly-supported} fields,
since the action functional is rarely well-defined on all fields when the source manifold is non-compact.
(Note this domain of compactly-supported fields is all one needs for making variational arguments or for constructing a BV quantization.)

An edge corresponds an element $P$ of the 2-fold tensor product of the space of fields,
often called a {\em propagator}.
More precisely, the edge should correspond to
the Green's function for the linear differential operator 
appearing in the associated quadratic term of the action;
hence the propagator is an element of the {\em distributional completion} of the 2-fold tensor product.
For us the $\beta\gamma$ leg should be labeled by $\dbar^{-1} \otimes {\rm id}_V$,
where $\dbar^{-1}$ denotes an inverse to the Dolbeault operator on functions.
The $bc$ leg should be labeled by $\dbar^{-1}_T$, 
the inverse of the Dolbeault operator on the bundle~$T^{1,0}$.

Given a graph~$\Gamma$, one should contract the tensors associated to the vertices and edges.
We denote the linear functional for this graph by~$w_\Gamma(P,I)$,
where $w$ stands for ``weight,'' the term $P$ indicates we label edges by the propagator~$P$,
and the term $I$ indicates we label vertices by the ``interaction'' term of the action~$S$ 
(i.e., the terms that are cubic and higher).

This contraction is not always well-posed, unfortunately.
Each vertex labels a distributional section of some vector bundle on~$\Sigma$,
and each edge labels a distributional section of a vector bundle on~$\Sigma^2$.
Thus the desired contraction can be written {\em formally} as an integral over the product manifold~$\Sigma^{v}$,
where $v$ denotes the number of vertices.
In most situations this contraction is ill-defined, 
since one cannot (usually) pair distributions.
Concretely, one sees that the integral expression is divergent.

Thus, to avoid these divergences, one labels the edges by a smooth replacement of the Green's functions. 
(Imagine replacing a delta function $\delta_0$ by a bump function.)
Since one can pair smooth functions and distributions,
each graph yields a linear functional on fields using these mollified edges.
Thus we have {\em regularized} the divergent expression.

But now this linear functional depends on the choice of mollifications.
Hence the challenge is to show that 
if one picks a sequence of smooth replacements that approaches the Green's function,
there is a well-defined limit of the linear functionals.

\subsubsection{}

We will now sketch one method well-suited to complex geometry
that allows us to see that no divergences appear for the holomorphic bosonic string.
Our approach is an example of the renormalization method developed by Costello in ~\cite{CosBook},
which applies to many more situations.

Our primary setting in this section is $\Sigma=\CC$.
For this Riemann surface, 
a standard choice of Green's function for the $\dbar$ that acts on functions is
\[
P(z,w) = \frac{1}{2 \pi i} \frac{\d z + \d w}{z-w}.
\]
It is a distributional one-form on $\CC^2$ that satisfies $\dbar \otimes 1(P) = \delta_\Delta$, 
where $\delta_\Delta$ is the delta-current supported along the diagonal $\Delta: \CC \hookrightarrow \CC^2$ and providing the integral kernel for the identity.
In terms of our discussion above,
we view this one-form as a distributional section of the fields $\gamma$ and~$\beta$: 
for example, for fixed $w$, the one-form $\d z/(z - w)$ is a $\beta$ field in the $z$-variable 
as it is a $(1,0)$-form.
(This propagator is for the $\beta\gamma$ fields---and one must tensor with a kernel for the identity on $V$---but a similar formula provides a propagator for the $bc$ fields.)
\brian{maybe expand on this and give formulas for the individual propagators}
\owen{We could write $P^{\beta\gamma}$ and $P^{bc}$, but I worry the notation gets heavy.}

\subsubsection{}

We will now describe the integral associated to a simple diagram.
For simplicity, we assume $V = \CC$ so that the $\gamma$ and $\beta$ fields are simply functions and $1$-forms on $\CC$, respectively.
Consider a ``tadpole'' diagram $\Gamma_{\rm tad}$ whose outer legs are $c$~fields 
(i.e., vector fields on~$\CC$).
\owen{Add picture.}
There is only one vertex here, corresponding to the cubic function on fields
\[
w_{\Gamma_{\rm tad}}(P,I_{\rm string}) = \int_{z \in \CC} \beta \wedge c\gamma.
\]
If the field $c$ is of the form $f(z) \d \zbar \partial_z$,
with $f$ compactly supported, 
then our integral is
\[
\int_{z \in \CC} \beta \wedge f(z)(\partial_z\gamma) \d \zbar.
\]
(Note that a general cubic function could be described as an integral over $\CC^3$,
but our function is supported on the small diagonal $\CC \hookrightarrow \CC^3$.)
The linear functional for this tapole diagram should be given by inserting the propagator $P$ in place of the $\beta$ and $\gamma$ fields. 
Hence it ought to be given by the following integral over~$\CC$:
\[
\int_{z \in \CC} c(z)P(z,w)|_{z = w}  
= \int_{z \in \CC} f(z) \partial_z \left(  \frac{1}{2 \pi i} \frac{\d z + \d w}{z-w}\right)|_{z = w}\, \d \zbar.
\]
This putative integral is manifestly ill-defined,
since the distribution is singular along the diagonal.

\subsubsection{}

We smooth out the propagator $P$ using familiar tools from differential geometry.
Fix a Hermitian metric on $\Sigma$, 
which then associates provides an adjoint $\dbar^*$ to the Dolbeault operator~$\dbar$.
For the usual metric on $\CC$, we have
\[
\dbar^* = -2 \frac{\partial}{\partial (\d \zbar)} \frac{\partial}{\partial z}.
\]
In physics one calls a choice of $\dbar^*$ a {\em gauge-fix} as it \owen{not sure how to end this sentence for noncompact manifolds \dots maybe this comment should go elsewhere.}
The commutator $[\dbar,\dbar^*]$, which we will denote $D$, 
is equal to $\tfrac{1}{2} \Delta$, where $\Delta$ is the Laplace-Beltrami operator for this metric \owen{correct?}.
We can thus call upon Hodge theory and many nice results about finding partial inverses to the Laplacian.

\owen{I'm not sure how much to say here.}

We introduce a smoothed version of the propagator using the heat kernel~$e^{-tD}$,
which is a notation that denotes a solution to the heat equation $\partial_t^2 f(t,z) = Df(t,z)$.
For $\CC$ with the Euclidean metric, the standard heat kernel is
\[
e^{-tD}(z,w) =  \frac{1}{4\pi t} e^{-|z-w|^2/4t} (\d z - \d w) \wedge (\d\zbar - \d\wbar) . 
\]
For $0 < \ell < L < \infty$, we define
\[
P_\ell^L = \dbar^* \int_{\ell}^L e^{-tD}\d t.
\]
We compute
\[
\dbar P_\ell^L = D \int_{\ell}^L e^{-tD}\d t =  \int_{\ell}^L \frac{d}{dt} e^{-tD}\d t = e^{-LD} - e^{-\ell D}.
\]
In the limit as $\ell \to 0$ and $L \to \infty$, the operator $P_\ell^L$ goes to a propagator (or Green's function) $P$ for~$\dbar$.
To see this, consider an eigenfunction $f$ of $D$ where $Df=\lambda f$.
\owen{with our conventions, is $\lambda$ positive or negative?}
Then
\[
(\dbar P_\ell^L) f = (e^{-L\lambda} - e^{-\ell \lambda})f, 
\]
which goes to $f$ as $L \to \infty$ and $\ell~\to~0$.
\owen{I want to be careful about this since eigenfunction decomposition is subtle on noncompact manifolds \dots}
Thus, if one works with the correct space of functions, 
$P_\ell^L$ is almost an inverse to $\dbar$;
moreover, it is a smooth function on $\Sigma~\times~\Sigma$.
\owen{Should I say why?}

\subsubsection{}

We now return to the tadpole diagram and put $P_\ell^L$ on the edge instead of~$P$.
(We again assume $V = \CC$ for simplicity.)
The propagator is
\begin{align}\label{propagator}
P_\ell^L(z,w) &= \int_{\ell}^L \d t \, \frac{\partial}{\partial (\d \zbar)} \frac{\partial}{\partial z}\left( \frac{1}{4\pi t} e^{-|z-w|^2/4t} (\d z - \d w) \wedge (\d\zbar - \d\overline{w})\right)\\
&= \int_{\ell}^L \d t \frac{1}{4\pi t} \frac{\zbar - \overline{w}}{2t} e^{-|z-w|^2/4t} (\d z - \d w).
\end{align}
Note that it is smooth everywhere on~$\CC^2$.
The integral for the tadpole diagram is 
\begin{align*}
w_{\Gamma_{\rm tad}}(P_\ell^L,I_{\rm string})
&= \int_{z \in \CC} c(z)P_\ell^L(z,w)|_{z = w}  \\
&= \int_{z \in \CC} \int_{\ell}^L \d t f(z) \partial_z \left(\frac{1}{4\pi t} \frac{\zbar - \overline{w}}{2t} e^{-|z-w|^2/4t} (\d z - \d w) \right)|_{z = w}\, \d \zbar\\
&= \int_{z \in \CC} \int_{\ell}^L \d t f(z) \left(\frac{1}{4\pi t} \left(\frac{\zbar - \overline{w}}{2t}\right)^2 e^{-|z-w|^2/4t} (\d z - \d w) \right)|_{z = w}\, \d \zbar\\
&= 0,
\end{align*}
since the integrand vanishes along the diagonal.
Note that this integral is independent of $\ell$ and $L$ and hence the limit is zero.

\subsubsection{}

By explicitly analyzing the $\ell \to 0$ limit for the integral associated to every Feynman diagram,
we find the following result.

\begin{prop}
For any graph $\Gamma \in {\bf Graph}_{\rm string}$ allowed by the combinatorics of the string action functional and for any $L > 0$,
there is a well-defined limit $\lim_{\ell \to 0} w_{\Gamma}(P_{\ell}^L,I_{\rm string})$.
\end{prop}

We denote this limit by~$w_{\Gamma}(P_{0}^L,I_{\rm string})$.

The necessary manipulations and inequalities are very close to those used in~\cite{}.
We recommend looking at \owen{exact location} for model arguments.

\owen{I think we should sketch a proof: we explain why trees are no issue, and then mention key ideas in the wheel case. We can even include a nontrivial example if there's space.}

\begin{proof}[Outline of proof]
When $\Gamma$ is a tree, there is never an issue with divergences; 
we could even use the Green's function $\dbar^{-1}$ on each edge.
To see this, note that one can view a tree as having a distinguished root,
given by the leg that is either of $\beta$ or $b$~type.
One can then see the tree as describing a multilinear map from the leaves (i.e., legs that are not roots) to the root.
Indeed, one can view each cubic vertex as such an operator.
For instance, $\langle b, [c,c]\rangle$ corresponds to the Lie bracket of vector fields,
since we view $\langle b,-\rangle$ as an element of the $c$~fields.
For a tree, one can then input arbitrary elements into the leaves, 
apply the operations labeled by the vertices,
apply the operator labeled by the edge, and so on,
until one reaches the root.
The composite multilinear operator sends smooth sections to smooth sections,
even if the edges are labeled by distributional sections,
since the associated operator sends smooth sections to smooth sections.

When $\Gamma$ is a one-loop graph, it consists of a wheel with trees attached to the outer legs.
By the preceding argument, we know those trees do not introduce singularities;
hence any divergences are due solely to the wheel.
It thus suffices to consider pure wheels (i.e., those with no trees attached).

Let the wheel have $n$ vertices. 
The $k$th vertex has a coordinate $z_k$ on $\CC$;
the $k$th external leg has input $c_k = f_k(z_k,\zbar_k) \d\zbar_k\, \partial_{z_k}$, 
where $f_k$ is a compactly-supported smooth function.
Then the integral has the form
\[
\int_{(z_1,\ldots,z_n) \in \CC^n}\d^n \zbar \,(f_1  \partial_{z_1} P_\ell^L(z_1,z_n))(f_2 \partial_{z_2} P_\ell^L(z_2,z_1)) \cdots (f_n\partial_{z_n} P_\ell^L(z_n,z_{n-1})),
\] 
since the $k$th input will act on one of the propagators entering the $k$th vertex.

Let us write the full expression for the case $n=2$.
Then we have
\begin{align*}
\int_{z_1,z_2 \in \CC} \d\zbar_1\d\zbar_2 \int_{\ell}^L \d t_1 \int_{\ell}^L \d t_2\, 
& f_1(z_1)f_2(z_2) 
\partial_{z_1} \left(\frac{1}{4\pi t_1} \frac{\zbar_1 - \zbar_2}{2t_1} e^{-|z_1-z_2|^2/4t_1} (\d z_1 - \d z_2) \right)\\
& \times \partial_{z_2} \left(\frac{1}{4\pi t_2} \frac{\zbar_1 - \zbar_2}{2t_2} e^{-|z_1-z_2|^2/4t_2} (\d z_2 - \d z_1) \right),
\end{align*}
which is already a bit lengthy.
As our focus is on showing a limit exists, we will throw out unimportant factors and simplify the expression.
First, note that taking the partial derivative $\partial_{z_i}$ will simply multiply the integrand by $(\zbar_1 - \zbar_2)/2t_i$.
Moreover, we change coordinates to $u = z_1 - z_2$ and $v = z_2$. 
Then the integral is proportional to
\[
\int_{\ell}^L \d t_1 \int_{\ell}^L \d t_2\int_{\CC^2} \d^2 u \, \d^2 v \, f_1 f_2 \frac{\overline{u}^4}{t_1^4 t_2^4} e^{-|u|^2(\tfrac{1}{t_1} + \tfrac{1}{t_2})}.
\]
\owen{I fixed the powers of $t$ but that also adds powers of $\Bar{u}$.}
Consider first the integral just over~$\CC^2$,
so that we are computing the expected value of $F=f_1 f_2$ against a Gaussian measure 
whose variance is determined by $t_1$ and~$t_2$.
(Namely, the variance is~$\sqrt{t_1 t_2/(t_1+t_2)}$.)
We might as well focus on values of $t_i$ that are very small, as those would be the source of divergences when $\ell \to 0$.
But for small $t_i$, we only care about the behavior of $F$ near the origin.
Consider a partial Taylor expansion of $F$.
The polynomial part can be computed quickly since the expected values of monomials against a Gaussian measure (i.e., the moments) have a simply expression in terms of the variance.
\owen{and now I should do it!}
On the other hand, the expected value of the error term, which vanishes to some positive order at the origin,
 \owen{explain why it doesn't matter}
\end{proof}

\owen{Define the renormalized action at scale $L$ as the sum of the Feynman diagrams labeled by $P_0^L$.}

\begin{dfn}
The {\em renormalized action functional} at scale~$L$ for the holomorphic bosonic string is
\[
S[L] = \sum_{\Gamma \in {\bf Graph}_{\rm string}^{(0)}} w_{\Gamma}(P_{0}^L,I_{\rm string}) + \hbar\sum_{\Gamma \in {\bf Graph}_{\rm string}^{(1)}} w_{\Gamma}(P_{0}^L,I_{\rm string}).
\]
We denote the first summand---the tree-level expansion---by $S_0[L]$ 
and the second summand---the one-loop expansion---by~$S_1[L]$
\end{dfn}

\owen{Will also remark that this flow satisfies an exact RG equation, and Kevin calls it ``RG flow" (but not to be confused with Wilsonian RG flow)}

\owen{Cite Si's paper about chiral deformations.}

\subsection{The quantum master equation}

In the BV formalism the basic idea is to replace integration against a path integral measure $e^{-S(\phi)/\hbar} \cD \phi$ with a cochain complex.
In this cochain complex, we view a cocycle as defining an observable of the theory,
and its cohomology class is viewed as its expected value against the path integral measure.
For toy models of finite-dimensional integration, see \cite{};
these examples are always cryptomorphically equivalent to a de Rham complex,
which is a familiar homological approach to integration.

Hence the content of the path integral, in this approach, is encoded in the differential. 
A key idea is that the differential is supposed to behave like a divergence operator for a volume form:
recall that given a volume form $\mu$ on a manifold, 
its divergence operator maps vector fields to functions by the relationship
\[
{\rm div}_\mu({\mathcal X}) \mu = L_{\mathcal X} \mu.
\] 
This relationship, in conjunction with Stokes lemma, 
implies that if a function~$f$ is a divergence ${\rm div}_\mu({\mathcal X})$,
then $\int f \mu = 0$,
i.e., its expected value against the measure~$\mu$ is zero.
%(The toy models extend the divergence operator of a true volume form on a finite-dimensional manifold to an operator on polyvector fields.)
The BV formalism axiomatizes general properties of divergence operators;
a putative differential must satisfy these properties to provide a BV quantization.

When following the algorithm of Section~\ref{sec:bvalgorithm},
we want the renormalized action
\[
S = S^{\rm cl} + \hbar S_1 + \hbar^2 S_2 + \cdots
\]
to determine a putative differential $\d^q_S$ on the graded vector space of observables.
To explain this operator, we need to describe further algebraic properties on the observables
that the BV formalism uses.

First, in practice, the observables are the symmetric algebra generated by the continuous linear duals to the vector spaces of fields.
There is also a pairing on fields that is part of the data of the classical BV theory,
between each field and its ``anti-field.''
(This pairing is a version of the action of constant vector fields on functions in the toy models.)
In our case, there is the pairing between $b$ and $c$ and between $\beta$ and~$\gamma$, respectively.
It behaves like a ``shifted symplectic'' pairing as it has cohomological degree~$-1$,
and hence it determines a degree~1 Poisson bracket $\{-,-\}$ on the graded algebra of observables.
Finally, the pairing also determines a second-order differential operator $\Delta_{BV}$ on the algebra of observables by the condition that
\[
\Delta_{BV}(FG) = (\Delta_{BV}F)G + (-1)^F F(\Delta_{BV}G) + \{F,G\}.
\]
(This equation is a characteristic feature of divergence operators with respect to the product of polyvector fields.)

With these structures in hand, we can give the formula
\[
\d^q_S=\{S,-\} + \hbar\Delta_{BV}
\]
for the putative differential.
As $S$ has cohomological degree~0, the operator $\{S,-\}$ has degree~1.
We remark that modulo~$\hbar$, one recovers the differential $\{S^{\rm cl},-\}$ on the classical observables;
the zeroth cohomology of the classical observables is functions on the critical locus of the classical action~$S^{\rm cl}$.

By construction, this putative differential $\d_S^q$ satisfies the conditions of behaving like a divergence operator.
The only remaining condition to check is that it is square-zero.
This condition ends up being equivalent to $S$ satisfying the {\em quantum master equation}
\begin{equation}
\hbar \Delta_{BV} S + \frac{1}{2}\{S,S\} = 0.
\end{equation}
More accurately, $\d^q_S$ is a differential if and only if the right hand side is a constant.


\subsubsection{}

We now turn to examining this condition in our setting.
It helps to understand it is diagrammatic terms.

As the bracket is determined by a linear pairing,
it admits a simple diagrammatic description as an edge.
For instance, given an observable $F$ that is a homogeneous polynomial of arity~$m$
and an observable $G$ of arity~$n$, 
then $\{F,G\}$ has arity~$m+n-2$.
It can be expressed as a Feynman diagram 
\owen{add picture}
where the edge connecting $F$ and $G$ is labeled by a 2-fold tensor~$K$.

The BV Laplacian acts by attaching an edge labeled by~$K$ as a loop in all possible ways.
\owen{Add picture.}
This diagrammatic behavior corresponds to the fact that $\Delta_{BV}$ is a constant-coefficient second-order differential operator.

The tensor~$K$ determined by the pairing on fields is distributional.
As one might expect from our discussion of divergences above,
these diagrammatic descriptions of the BV bracket and Laplacian are thus typically ill-defined.
In other words, the quantum master equation is {\em a priori} ill-posed for the same reason that the initial Feynman diagrams are ill-defined.
We can apply, however, the same cure of mollification.

\subsubsection{}

Costello's framework \cite{CosBook} provides an approach to renormalization built to be compatible with the BV formalism.
A key feature is that for each ``length scale''~$L>0$, 
there is a BV bracket $\{-,-\}_L$ and BV Laplacian $\Delta_L$.
The scale~$L$ renormalized action $S[L]$ satisfies the scale~$L$ quantum master equation~(QME)
\[
\hbar \Delta_{L} S[L] + \frac{1}{2}\{S[L],S[L]\}_L = 0
\]
if and only if $S[L']$ satisfies the scale~$L'$ quantum master equation for every other scale~$L'$.
\owen{Cite Kevin's lemma in book.}
Hence, we say a renormalized action satisfies the quantum master equation if its solves the scale~$L$ equation for some~$L$.

Thus it remains for us to describe the scale~$L$ bracket and BV Laplacian in our setting,
so that we can examine whether the renormalized action satisfies the quantum master equation.

\begin{dfn}
The {\em scale~$L$ bracket} $\{-,-\}_L$ is given by pairing with the scale~$L$ heat kernel
\[
K_L(z,w) = \frac{1}{4\pi L} e^{-|z-w|/4L} (\d z - \d w) \wedge (\d\zbar - \d\overline{w}). 
\]
The {\em scale~$L$ BV Laplacian} $\Delta_L$ is given by the contraction~$\partial_{K_L}$.
\end{dfn}

These definitions mean that testing the quantum master equation leads to diagrams whose integrals are similar to those we encountered earlier.
We explain the diagrammatics and sketch the relevant integrals in the proof of the following result,
which characterizes when the string action admits a BV quantization.

We emphasize that up to now, we have not indicated explicitly which vector space $V$ is the target space for our string.
But the action functional explicitly depends on this choice,
so here we will write $S_V$ for the action with target~$V$.

\begin{prop}
The obstruction to satisfying the quantum master equation is the functional
\[
Ob_V[L] = \hbar \Delta_{L} S_V[L] + \frac{1}{2}\{S_V[L],S_V[L]\}_L.
\]
It has the form
\[
Ob_V[L] = \hbar (\dim_\CC(V) - 13) F[L],
\]
where $F[L]$ is a functional independent of~$V$.
\end{prop}

In short, the failure to satisfy the QME is a linear function of the dimension of the target space~$V$.
In particular, when $V \cong \CC^{13}$, 
the obstruction vanishes and the renormalized action {\em does} satisfy the QME, 
giving us an immediate corollary.
(Note that we do {\em not} need to know $F[L]$ to recognize that the obstruction vanishes!)

\begin{cor}
When the target vector space is 13-dimensional (i.e., has 26 real dimensions),
the holomorphic bosonic string admits a BV quantization.
\end{cor}

\begin{proof}
It is a general feature of Costello's formalism that the tree-level term $S_0[L]$ of the renormalized action satisfies the scale~$L$ equation
\[
\{S_0[L],S_0[L]\}_L = 0,
\]
known as the classical master equation.
Hence the first obstruction to satisfying the QME can only appear with positive powers of~$\hbar$.
We can also see quickly that no terms of~$\hbar^2$ appear:
the one-loop term $S_1[L]$ is only a function of the $c$~field, 
so 
\[
\{S_1[L],S_1[L]\}_L = 0 \quad\text{and}\quad \Delta_L S_1[L] = 0.
\]
Hence the obstruction to satisfying the QME is precisely
\[
\hbar\left( \{S_0[L],S_1[L]\} + \Delta_L S_0[L] \right).
\]
Thus we see that the obstruction is a multiple of~$\hbar$.
For simplicity, we will divide out that factor and let $Ob_V$ denote the term inside the parenthesis.

Consider the term $\{S_0[L],S_1[L]\}_L$. 
Diagrammatically, it corresponds to attaching a tree with a $b$ ``root'' to a wheel using an edge labeled by~$K_L$.
\owen{Need to reference/explain the results of Si and KC.}

Now consider the term $\Delta_L S_0[L]$. 
Diagrammatically, it corresponds to turning a tree into a wheel by using an edge---labeled by $K_L$---to attach the root to an incoming leaf.
There are thus two kinds of wheels that appear, since there are two kinds of trees.
There are the wheels where the $K$ edge is for $bc$ fields.
Note that these wheels are the same for every choice of target $V$
as they only depend on the $bc$ fields, i.e., are independent of the $\beta\gamma$ fields.
These will contribute a term $F[L]$ to the obstruction.
On the other hand, there are the wheels where the $K$ edge is for $\beta\gamma$ fields.
These depend on $V$ but in a very simple way: 
the distribution $K$ is just the heat kernel tensored with the identity on~$V$, 
and hence the contraction amounts to taking $\dim_\CC(V)$ copies of the $V = \CC$ value.
In other words, the $\beta\gamma$ wheels contribute a term $\dim_\CC(V) G[L]$ to the obstruction,
where $G[L]$ is the value for $V = \CC$.
The last part of the proof of the theorem is a direct calculation of the functionals $F[L]$ and $G[L]$. 
So as to not diverge from our track of thought we include this calculation in Section \ref{??} where we show that $F[L], G[L]$ are both independent of $L$ and satisfy $F = -13 G$, thus completing the proof.
\end{proof}

\begin{rmk} One can consider coupling the $\beta\gamma$ system to any tensor bundle on the Riemann surface. 
In the case the the target is one-dimensional, we can show that the part of the obstruction with internal edges labeled by the $\beta\gamma$ propagators is $6n^2 + 6n + 1$ times the same functional $G$ above where $n$ is the conformal weight of the tensor bundle (so $\gamma$ is a section of $T_\Sigma^{\tensor n}$. 
\brian{should I say more?}
\end{rmk}





