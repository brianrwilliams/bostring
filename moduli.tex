\section{Deformations of the theory and string backgrounds}
\label{sec: moduli}

\owen{Maybe the Gelfand-Fuk discussion can be anticipated in Section 1? I think versions of it are easy to motivate: ``}

\owen{We should observe that we see the deformations of the action, such as $B$-fields and dilatons. Observe we've rediscovered "string backgrounds."}

\brian{Might be good to hint at the curved sigma model here.}

Whenever one is studying a theory,
it is helpful to understand how it can be modified 
and how features of the theory change as one adjusts natural parameters of the theory,
such as coupling constants of the action functional.
In other words, one wants to understand the theory in the moduli space of classical theories.

In the BV formalism, because we are working homologically, this moduli space is derived,
and there is a tangent complex to our theory in the moduli of classical BV theories.
We call it the {\em deformation complex} of the theory.
A systematic discussion can be found in Chapter 5 of~\cite{CosBook}.

As a gloss, the underlying graded vector space of this deformation complex consists of the local functionals on the jets of fields, i.e., Lagrangian densities.
(Note that we allow local functionals of arbitrary cohomological degree.) 
There is also a shifted Lie bracket $\{-,-\}$, 
which arises from the pairing $\int_\Sigma \langle-,-\rangle$ on the fields.
It is, in essence, the shifted Poisson bracket corresponding to that shifted symplectic pairing on the fields.
The differential on the local functionals is then $\{S,?\}$, where $S$ is the classical action. 
All together, the deformation complex forms a shifted dg Lie algebra. 
Observe that if we find a degree zero element $I$ such that
\[
0=\{S +I,S +I\}=2\{S,I\}+\{I,I\},
\]
then $I$ is a shifted Maurer-Cartan element and 
hence determines a new classical BV theory whose action functional is $S + I$. 
In particular, degree 0 cocycles determine first-order deformations of the classical BV theory. Cocycles in degree -1 encode local symmetries of the classical theory; 
and obstructions to satisfying the quantum master equation end up being degree 1 cocycles.

In this section, we will explain why the deformation complex of the holomorphic string 
can be expressed in terms of Gelfand-Fuks cohomology. 
Along the way we will see how the usual backgrounds for the bosonic string (a target metric, dilaton term, and so on) appear as elements in this complex of local functionals and hence as deformations of the classical action. 

\subsection{}

We have already seen how to think of the holomorphic bosonic string theory 
as corresponding to the shifted cotangent bundle $\TT^*[-1]{\rm Maps}_{\dbar}(-, V)$, 
as a bundle over the moduli of Riemann surfaces. 
There is a natural action of the group $ \CC^\times$ on this space
by scaling the shifted cotangent fibers,
and we will use the notation $\CC^\times_{\rm cot}$ to indicate this appearance of the multiplicative group.

This group action can be seen on the level of the field theory as follows: 
we give the $\gamma$ and $c$ fields---the base of the cotangent bundle---weight $0$ and give the $\beta$ and $b$ fields---the cotangent fiber---weight~$1$. 
Note that this implies that the pairing $\langle-,-\rangle$ on fields thus has weight -1 
\owen{I'm kind of confused here: I'd hoped that the BV bracket has weight -1 so that $\{S,-\}$ has overall weight 0.}
\brian{I think this weight is preserved by duals. That is, both \<-,-\> and \{-,-\} have weight -1. So we're OK.}
In these terms, the classical action functional is weight 1. 
Thus, we focus on weight 1 deformations of the holomorphic bosonic string,
as we are interested in local functionals of the same kind.
That means we consider the subcomplex of weight 1 local functionals inside the deformation complex.

\owen{I think we should explain this choice: it's because we want the overall differential to have weight zero, and $\Delta$ should have weight $-1$.}
The weight of the parameter $\hbar$ is also one with respect to the scaling by $\CC^\times_{\rm cot}$.
Thus, for quantum corrections at one loop we consider local functionals that are of weight zero for this $\CC^\times_{\rm cot}$-action.
Put simply, these are local functionals that only depend on the base of the shifted cotangent bundle. 
On a Riemann surface $\Sigma$ the $\CC^\times_{\rm cot}$-invariant local functionals are of the form
\ben
\left(\Oloc\left(\Omega^{0,*}(\Sigma, V) \oplus \Omega^{0,*}(\Sigma, T^{1,0})[1] \Omega^{1,*}(\Sigma ; V) \oplus \Omega^{1,*}(\Sigma, T^{1,0*})[-1] \right), \{S, -\}\right) .
\een
We are concerned with the case that $\Sigma = \CC$ and we restrict ourselves to studying {\em translation invariant} local functionals. 

We have already mentioned that Lagrangian densities are functions on jets of fields integrated against a density. 
The simplest case is to consider functions on jets at a point, which is sufficient to understand a translation invariant Lagrangian densities. 
Hence, for translation invariant local functionals on $\CC$ it suffices to look at functionals of the jet at $0 \in \CC$ of the fields. 
For instance, if we look at the jet at zero of the complex of sheaves $\Omega^{0,*}(\CC ; V)$ we obtain the complex
\ben
(V [[z,\zbar]] [\d \zbar] , \dbar)
\een
where $\dbar$ is the formal Dolbeault differential. 
By the Poincar\'{e} lemma for the $\dbar$ operator this complex is quasi-isomorphic to $V[[z]]$ (concentrated in degree zero). 
Similarly, the jet at zero of the complex of sheaves $\Omega^{0,*}(\CC ; T^{1,0}_\CC)$ is quasi-isomorphic, as dg Lie algebras, to the Lie algebra of formal vector fields in one variable ${\rm W}_1$.

The main result of this section is the following:

\begin{prop}\label{prop: def complex} There are $\GL(V)$-equivariant quasi-isomorphisms of the translation invariant deformations
\begin{itemize}
\item[(0)] of weight $0$:
\ben
\left(\Oloc(\cE)^{(0)}, \{S,-\}\right)^{\CC} \simeq \CC[-1] \oplus \Omega^2_{cl}(V)[1] \oplus \Omega^1(V) \oplus \Omega^1_{cl}(V)[-1] 
\een
\item[(1)] and of weight $1$:
\ben 
\left(\Oloc(\cE)^{(1)} , \{S,-\}\right)^\CC \simeq T_V [1] .
\een
\end{itemize}
\end{prop}

\owen{At the moment, I find the discussion above a bit convoluted. Let's discuss how to disentangle.}

\owen{One option is to rearrange the order of this whole section. It's natural to talk first just about weight 1, since that is strictly classical. Postpone discussing $\hbar$ etc. Hence, right after introducing weight 1, we can just state the value of the deformation complex for that component. In the next subsection, we can say that weight 0 stuff shows up when quantizing at 1-loop and refer ahead in the document, explain the more subtle weight issues, and then state the deformation complex for that component.}

\subsection{}

The jets at $0 \in \CC$ of local Lie algebra $\Omega^{0,*}(\CC ; T^{1,0} \ltimes (V[-1] \oplus \d z V^*[-1] \oplus \d z T^{1,0 *})$ is quasi-isomorphic to the Lie algebra
\be\label{jet lie}
{\rm W}_1 \ltimes (V [[z]] [-1] \oplus \d z V^*[[z]] [-1] \oplus \hOmega^1_1 [-2])
\ee
where ${\rm W}_1 = \CC [[z]] \partial_z$ is the Lie algebra of formal vector fields in one variable and $\hOmega^1_1 = \CC [[z]] \d z$ is the space of formal one forms. 
The Lie bracket comes from bracket on ${\rm W}_1$ and the natural action of ${\rm W}_1$ on $\CC [[z]]$ and $\hOmega^1_1$. 

Denote by $\{L_n = z^{n+1} \partial_z\}$ the standard basis for the Lie algebra of formal vector fields ${\rm W}_1$. 
Let $\lambda_n \in {\rm W}_1^\vee$ be the dual vector to $L_n$ (we are using the continuous dual, as in the setting of Gelfand-Fuks cohomology). 
An arbitrary element of $V [[z]]$ is linear combination of vectors of the form $v \tensor z^k$. 
Write $\zeta_k$ for the dual element $(z^k)^\vee$. 
Thus an element of $(V [[z]])^\vee$ is a linear combination of the vectors of the form $v^\vee \tensor \zeta_k$. 

\subsubsection{The $\CC^\times_{\rm cot}$-weight zero piece}

The weight $\CC^\times_{\rm cot}$-weight zero sub Lie algebra of the Lie algebra (\ref{jet lie}) is simply $W_1 \ltimes V [[z]] [-1]$, where the semi-direct product comes from the natural action of formal vector fields on formal power series.
Thus, we have reduced the calculation of the $\CC^\times_{\rm cot}$-weight zero piece of the local deformation complex to calculating the Chevalley-Eilenberg complex of this Lie algebra:
\ben
\cred^* \left({\rm W}_1 \ltimes V [[z]] [-1] \right) .
\een
This splits into two terms $\cred^* ({\rm W}_1) \oplus \clie^*(W_1 ;  \Sym^{\geq 1} (V [[z]])^\vee)$. 

The first term in this summand is the reduced Gelfand-Fuks cohomology of formal vector fields with values in the trivial module.
It is well-known that the cohomology is one-dimensional and concentrated in degree $3$, $H^3_{\rm red} ({\rm W}_1) \cong \CC[-3]$. 
We will identify the anomaly for the holomorphic string with flat target as multiple of the generator of this space. 
The remaining piece of the weight zero deformation complex is the home of the anomalies for the holomorphic string placed in a non trivial background: for instance, when the target of the $\sigma$ model is curved. 
We will not see this in our theory, of course, but the following will hopefully be...
\owen{Not sure where this paragraph is going.}

The first step in computing this is to notice that there is a quasi-isomorphic subcomplex.
The vector field $L_0 = z^{n+1} \partial_z$ induces a grading on ${\rm W}_1$ and hence on the Chevalley-Eilenberg complex of ${\rm W}_1$ with coefficients in any module. 
We will call this grading the {\em conformal dimension}.

 \begin{lem} Let $M$ be any ${\rm W}_1$-module. Then, the inclusion of the conformal dimension zero subcomplex
\ben
\clie^*({\rm W}_1 ; M)^{(0)} \xto{\simeq} \clie^*({\rm W}_1 ; M)
\een
is a quasi-isomorphism. \brian{Is this true?}
\end{lem}

\begin{proof} For $p-1$ define the operator $\iota_{L_0} : \clie^{p}({\rm W}_1 ; M) \to \clie^{p-1}({\rm W}_1 ; M)$ defined by sending a cochain $\varphi$ to the cochain
\ben
(\iota_{L_0}\varphi)(X_1,\ldots,X_p) = \varphi(L_0, X_1,\ldots,X_p) .
\een 
Let $\d$ be the differential for the complex $\clie^*({\rm W}_1 ; M)$. It is easy to check that the difference $\d \iota_{L_0} - \iota_{L_0} \d$ is equal to the projection onto the dimension zero subspace. 
\end{proof}

The underlying graded vector space of this conformal dimension zero subcomplex splits as follows:
\ben
\clie^{\#}({\rm W}_1)^{(0)} \tensor \left(\Sym^{\geq 1}\left(V [[z]]\right)^\vee \right)^{(0)} \oplus \clie^{\#}({\rm W}_1)^{(1)} \tensor \left(\Sym^{\geq 1}\left(V [[z]]\right)^\vee\right)^{(-1)}
\een
Observe that the dimension zero part of the reduced symmetric algebra is simply $\Sym^{\geq 1}(V^\vee)$ which is identified $\GL(V)$-equivariantly with $\cO_{red}(V)$. 
That is, power series on $V$ with no constant term. 
Similarly, the dimension one part of $\Sym^{\geq 1}\left(V[[z]]\right)^\vee$ is of the form ${\rm Sym}(V^\vee) \tensor z^\vee V^\vee$, which is identified $\GL(V)$-equivariantly with $\Omega^1(V)$. 

The full dimension zero complex, including the differential is
\[
\xymatrix{
1 \otimes \sO_{red}(V) \ar[rd]^{\d_{dR}} & \lambda^0 \otimes \sO_{red}(V) \ar[r] \ar[rd]^{\d_{dR}} & \lambda^{-1} \wedge \lambda^1 \otimes \sO_{red}(V) & \lambda^{-1} \wedge \lambda^1 \wedge \lambda^0 \otimes \sO_{red}(V) \\
 & \lambda^{-1} \otimes \Omega^1(V) \ar[r] & \lambda^{-1} \wedge \lambda^0 \otimes\Omega^1(V) &
}
\]
The top horizontal map sends $\lambda^0 \mapsto 2 \cdot \lambda^{-1} \wedge \lambda^1$ and the bottom horizontal map sends $\lambda^{-1}$ to $\lambda^{-1} \wedge \lambda^0$ (both are the identity on $V$). 
The diagonal maps are given by the de Rham differential $\d_{dR} : \sO_{red}(V) \to \Omega^1(V)$. 
This complex is quasi-isomorphic to 
\[
\xymatrix{
1 \otimes \sO_{red}(V) \ar[rd]^{\d_{dR}} & & & \lambda^{-1} \wedge \lambda^1 \wedge \lambda^0 \otimes \sO_{red}(V) \\
 & \lambda^{-1} \otimes \Omega^1(V) & \lambda^{-1} \wedge \lambda^0 \otimes\Omega^1(V) &
}
\]
which, in turn, is identified with $\Omega^{2}_{cl}(V)[1] \oplus \Omega^1(V) \oplus \Omega^1_{cl}(V)$. This completes the calculation of the $\CC^\times_{\rm cot}$-weight zero component. 

\subsubsection{The $\CC^\times_{\rm cot}$-weight one piece}

The $\CC^\times_{\rm cot}$-weight one part of the Lie algebra (\ref{jet Lie}) is $\d z V^\vee [[z]] [-1] \oplus \d z \hOmega^1_{1} [-2]$.

A totally analogous calculation as in the weight zero case yields the following. 

\begin{prop} There is a $\GL(V)$-equivariant quasi-isomorphism
\ben
\clie^*\left(W_1 ; \Sym \left(V [[z]]\right)^\vee \tensor (\d z V^*[[z]] \oplus \hOmega^{1\tensor 2}_1[-1])^\vee \right) \simeq T_V [-1]
\een
where $T_V$ denotes the adjoint representation which as a vector space is $T_V = \Sym(V^*) \tensor V$. 
\end{prop}

This result implies the second part of Proposition \ref{prop: def complex}. 

\subsubsection{Interpretation as string backgrounds}

Since $V$ is flat we will see that all such deformations will be trivializable. 
Note that this trivializations will {\em not} be compatible with $\GL(V)$ (or, in particular general diffeomorphisms of the target). 
Thus, these deformations are relevant for the case of a curved target and we can give an interpretation of them in terms of the usual perspective of {\em string backgrounds}. 

We have already mentioned that we should think of the $\CC^\times_{\rm cot}$ weight $1$ local functionals as deformations of the classical theory as a cotangent theory.
The cohomological degree zero deformations of the weight one deformations is $H^1(V ; T_V)$. 
Given any such element $\mu \in H^1(V ; T_V)$ we can consider the following local functional
\ben
\int_\Sigma \<\beta, \mu(\gamma)\>_V .
\een 
The element $\mu$ determines a deformation of the complex structure of $V$, and we have prescribed an action functional encoding this deformation. 
We propose that this an appearance of the ordinary curved background in bosonic string theory from the perspective of the holomorphic model we work with.

There are interesting deformations that go outside of the world of cotangent theories. 
Consider the cohomological degree zero part of the weight 0 complex. 
There is a term of the form $H^1(V ; \Omega^2_{cl}(V))$.
It is shown in Part 2 Section 8.5 of \cite{ggw} how closed holomorphic two-forms determine local functionals of the $\beta\gamma$ system with curved target. 
A sketch of this construction goes as follows.
Locally we can write a closed holomorphic 2-form as $\d \theta$ for some holomorphic one-form $\theta \in \Omega^1(V)$. 
If $\gamma : \Sigma \to V$ is a map of the $\sigma$-model there is an induced map (when $\gamma$ satisfies the equations of motion) $\gamma^* : \Omega^1(V) \to \Omega^1(\Sigma)$. 
We can then integrate $\gamma^* \theta$ along any closed cycle $C$ in $\Sigma$ and one should think of this as a residue along $C$. 
In \cite{ggw} we write down a local functional that realizes this residue, and one can show that it only depends on the corresponding class in $H^1(V ; \Omega^2_{cl}(V))$. 
We posit that this is the appearance of the $B$-field deformation of the ordinary bosonic string. 



